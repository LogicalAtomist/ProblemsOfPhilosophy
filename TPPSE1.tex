\documentclass[oneside,letterpaper,12pt]{book}
\usepackage[utf8]{inputenc}
\usepackage{graphicx}
\usepackage{setspace}
\usepackage{quoting}
\usepackage{amssymb} 
\usepackage{amsmath}
\usepackage{amsthm} 
\usepackage[utf8]{inputenc}
\usepackage[T1]{fontenc}
\usepackage{times}
\usepackage{enumitem}
\usepackage{microtype}
%\usepackage[margin=1in]{geometry}
\usepackage[pdfencoding=unicode]{hyperref}
\usepackage[round]{natbib}
\usepackage{pifont} 
\usepackage{array}
\usepackage{forest}
%\usepackage{fancyhdr}
\newcommand\titem[1]{\item{\bfseries #1}\\}

\begin{document}
\title{The Problems of Philosophy: Student Edition}
\author{\doublespacing Bertrand Russell \and editing and student notes by Landon D. C. Elkind}
\date{\today}

\frontmatter

\maketitle

\chapter*{Editor's Introduction}
This is a very lightly edited open educational resource version of \href{https://archive.org/details/theproblemsofphi00russuoft}{\emph{The Problems of Philosophy}} by \href{https://bertrandrussellsociety.org/bertrand-russell/}{Bertrand Russell}. It is lightly edited to approach (without yet meeting) the ``truer'' text of \emph{The Problems of Philosophy} \href{https://mulpress.mcmaster.ca/russelljournal/article/view/4810}{indicated by Kenneth Blackwell in the journal \emph{Russell}}.\footnote{See ``A Truer Text of \emph{The Problems of Philosophy}, \emph{Russell} Vol. 41, No. 2, pp. 78-85.} Student notes from the editor are included to make the text more suitable for college (or high school) students. 

The basis for this edition is \href{https://www.gutenberg.org/cache/epub/5827/pg5827-images.html}{this 2004 version available in HTML} through \emph{Project Gutenberg}, produced by Gordon Keener and David Widger. After using \href{https://pandoc.org/}{Pandoc} to convert this HTML file into TeX, I made some aesthetic edits (like replacing Roman numerals by Arabic ones and using Arabic numerals for footnote markers). All the source files and the resulting PDF compiled in TeX are hosted on this GitHub repository and updates will be pushed there.

It is my plan to produce two descendants of this book. The first is an \emph{illustrated} (and hopefully thereby improved) student edition that includes black-and-white diagrams for the reader while retaining the instructor notes. This edition will also include excerpts from primary sources that Russell discusses in the text. 

The second is a \emph{truer} edition without my notes for the student or primary sources but with Russell's index (with updated page numbers) and following the paragraph and sentence breaks in the 1967 edition published by Oxford University Press. This will also be a ``truer'' text of \emph{Problems} than any other known to me. %Both editions (like this one) will follow Blackwell's suggested emendations.

This version of \emph{The Problems of Philosophy} is published under a Creative Commons license (\href{https://creativecommons.org/licenses/by-sa/4.0/}{CC BY-SA 4.0}). The work done to produce this book was generously supported by an Affordable Textbook Initiative Grant from Western Kentucky University's Libraries. I thank the \href{http://www.russfound.org/}{Bertrand Russell Peace Foundation}, who hold the copyright to \emph{The Problems of Philosophy}, for their permission to reproduce Russell's terrific introduction to philosophy.

\begin{flushright}
	Landon D. C. Elkind
\end{flushright} 

\hypertarget{preface}{%
	\chapter*{Preface}\label{preface}}

In the following pages I have confined myself in the main to those
problems of philosophy in regard to which I thought it possible to say
something positive and constructive, since merely negative criticism
seemed out of place. For this reason, theory of knowledge occupies a
larger space than metaphysics in the present volume, and some topics
much discussed by philosophers are treated very briefly, if at all.

I have derived valuable assistance from unpublished writings of G. E.
Moore and J. M. Keynes: from the former, as regards the relations of
sense-data to physical objects, and from the latter as regards
probability and induction. I have also profited greatly by the
criticisms and suggestions of Professor Gilbert Murray.

\bigskip 

1912

\section*{Note to the Seventeenth Impression}

With reference to certain 
statements on pages \pageref{China1}, \pageref{China2}, \pageref{Balfour1}, 
and \pageref{Balfour2}, it should be remarked that this book was 
written in the early part of 1912, when China was still an empire, and 
the name of the then late Prime Minister did begin with the letter B.

\bigskip

1943 
\begin{flushright}
	Bertrand Russell
\end{flushright} 

\tableofcontents

\mainmatter
\chapter{Appearance and Reality}\label{chapter-i.-appearance-and-reality}
Is there any knowledge in the world which is so certain that no
reasonable man could doubt it? This question, which at first sight might
not seem difficult, is really one of the most difficult that can be
asked. When we have realized the obstacles in the way of a
straightforward and confident answer, we shall be well launched on the
study of philosophy---for philosophy is merely the attempt to answer
such ultimate questions, not carelessly and dogmatically, as we do in
ordinary life and even in the sciences, but critically,\label{reasons} after exploring
all that makes such questions puzzling, and after realizing all the
vagueness and confusion that underlie our ordinary ideas.

In daily life, we assume as certain many things which, on a closer
scrutiny, are found to be so full of apparent contradictions that only a
great amount of thought enables us to know what it is that we really may
believe. In the search for certainty, it is natural to begin with our
present experiences, and in some sense, no doubt, knowledge is to be
derived from them. But any statement as to what it is that our immediate
experiences make us know is very likely to be wrong. It seems to me that
I am now sitting in a chair, at a table of a certain shape, on which I
see sheets of paper with writing or print. By turning my head I see out
of the window buildings and clouds and the sun. I believe that the sun
is about ninety-three million miles from the earth; that it is a hot
globe many times bigger than the earth; that, owing to the
earth's rotation, it rises every morning, and will
continue to do so for an indefinite time in the future. I believe that,
if any other normal person comes into my room, he will see the same
chairs and tables and books and papers as I see, and that the table
which I see is the same as the table which I feel pressing against my
arm. All this seems to be so evident as to be hardly worth stating,
except in answer to a man who doubts whether I know anything. Yet all
this may be reasonably doubted, and all of it requires much careful
discussion before we can be sure that we have stated it in a form that
is wholly true.

To make our difficulties plain, let us concentrate attention on the
table. To the eye it is oblong, brown and shiny, to the touch it is
smooth and cool and hard; when I tap it, it gives out a wooden sound.
Any one else who sees and feels and hears the table will agree with this
description, so that it might seem as if no difficulty would arise; but
as soon as we try to be more precise our troubles begin. Although I
believe that the table is `really' of
the same colour all over, the parts that reflect the light look much
brighter than the other parts, and some parts look white because of
reflected light. I know that, if I move, the parts that reflect the
light will be different, so that the apparent distribution of colours on
the table will change. It follows that if several people are looking at
the table at the same moment, no two of them will see exactly the same
distribution of colours, because no two can see it from exactly the same
point of view, and any change in the point of view makes some change in
the way the light is reflected.

For most practical purposes these differences are unimportant, but to
the painter they are all-important: the painter has to unlearn the habit
of thinking that things seem to have the colour which common sense says
they `really' have, and to learn the
habit of seeing things as they appear. Here we have already the
beginning of one of the distinctions that cause most trouble in
philosophy---the distinction between
`appearance' and
`reality', between what things seem to be
and what they are. The painter wants to know what things seem to be, the
practical man and the philosopher want to know what they are; but the
philosopher's wish to know this is stronger than the
practical man's, and is more troubled by knowledge as to
the difficulties of answering the question.

To return to the table. \label{evident} It is evident from what we have found, that
there is no colour which pre-eminently appears to be \emph{the} colour
of the table, or even of any one particular part of the table---it
appears to be of different colours from different points of view, and
there is no reason for regarding some of these as more really its colour
than others. And we know that even from a given point of view the colour
will seem different by artificial light, or to a colour-blind man, or to
a man wearing blue spectacles, while in the dark there will be no colour
at all, though to touch and hearing the table will be unchanged. This
colour is not something which is inherent in the table, but something
depending upon the table and the spectator and the way the light falls
on the table. When, in ordinary life, we speak of \emph{the} colour of
the table, we only mean the sort of colour which it will seem to have to
a normal spectator from an ordinary point of view under usual conditions
of light. But the other colours which appear under other conditions have
just as good a right to be considered real; and therefore, to avoid
favouritism, we are compelled to deny that, in itself, the table has any
one particular colour.

The same thing applies to the texture. With the naked eye one can see
the grain, but otherwise the table looks smooth and even. If we looked
at it through a microscope, we should see roughnesses and hills and
valleys, and all sorts of differences that are imperceptible to the
naked eye. Which of these is the `real'
table? We are naturally tempted to say that what we see through the
microscope is more real, but that in turn would be changed by a still
more powerful microscope. If, then, we cannot trust what we see with the
naked eye, why should we trust what we see through a microscope? Thus,
again, the confidence in our senses with which we began deserts us.

The shape of the table is no better. We are all in the habit of judging
as to the `real' shapes of things, and
we do this so unreflectingly that we come to think we actually see the
real shapes. But, in fact, as we all have to learn if we try to draw, a
given thing looks different in shape from every different point of view.
If our table is `really' rectangular,
it will look, from almost all points of view, as if it had two acute
angles and two obtuse angles. If opposite sides are parallel, they will
look as if they converged to a point away from the spectator; if they
are of equal length, they will look as if the nearer side were longer.
All these things are not commonly noticed in looking at a table, because
experience has taught us to construct the
`real' shape from the apparent shape,
and the `real' shape is what interests
us as practical men. But the `real'
shape is not what we see; it is something inferred from what we see. And
what we see is constantly changing in shape as we move about the room;
so that here again the senses seem not to give us the truth about the
table itself, but only about the appearance of the table.

Similar difficulties arise when we consider the sense of touch. It is
true that the table always gives us a sensation of hardness, and we feel
that it resists pressure. But the sensation we obtain depends upon how
hard we press the table and also upon what part of the body we press
with; thus the various sensations due to various pressures or various
parts of the body cannot be supposed to reveal \emph{directly} any
definite property of the table, but at most to be \emph{signs} of some
property which perhaps \emph{causes} all the sensations, but is not
actually apparent in any of them. And the same applies still more
obviously to the sounds which can be elicited by rapping the table.

Thus it becomes evident that the real table, if there is one, is not the
same as what we immediately experience by sight or touch or hearing. The
real table, if there is one, is not \emph{immediately} known to us at
all, but must be an inference from what is immediately known. Hence, two
very difficult questions at once arise; namely, (1) Is there a real
table at all? (2) If so, what sort of object can it be?

It will help us in considering these questions to have a few simple
terms of which the meaning is definite and clear. Let us give the name
of `sense-data' to the things that are
immediately known in sensation: such things as colours, sounds, smells,
hardnesses, roughnesses, and so on. We shall give the name
`sensation' to the experience of being
immediately aware of these things. Thus, whenever we see a colour, we
have a sensation \emph{of} the colour, but the colour itself is a
sense-datum, not a sensation. The colour is that \emph{of} which we are
immediately aware, and the awareness itself is the sensation. It is
plain that if we are to know anything about the table, it must be by
means of the sense-data---brown colour, oblong shape, smoothness,
etc.---which we associate with the table; but, for the reasons which
have been given, we cannot say that the table is the sense-data, or even
that the sense-data are directly properties of the table. Thus a problem
arises as to the relation of the sense-data to the real table, supposing
there is such a thing.

The real table, if it exists, we will call a `physical
object'. Thus we have to consider the relation of
sense-data to physical objects. The collection of all physical objects
is called `matter'. Thus our two
questions may be re-stated as follows: (1) Is there any such thing as
matter? (2) If so, what is its nature?

The philosopher who first brought prominently forward the reasons for
regarding the immediate objects of our senses as not existing
independently of us was \href{https://plato.stanford.edu/entries/berkeley/}
{Bishop Berkeley} (1685-1753). His \href{https://archive.org/details/threedialoguesbe00berkiala}
{\emph{Three Dialogues between Hylas and Philonous, in Opposition to Sceptics and Atheists}}, 
undertake to prove that there is no such thing as matter at
all, and that the world consists of nothing but minds and their ideas.
Hylas has hitherto believed in matter, but he is no match for Philonous,
who mercilessly drives him into contradictions and paradoxes, and makes
his own denial of matter seem, in the end, as if it were almost common
sense. The arguments employed are of very different value: some are
important and sound, others are confused or quibbling. But Berkeley
retains the merit of having shown that the existence of matter is
capable of being denied without absurdity, and that if there are any
things that exist independently of us they cannot be the immediate
objects of our sensations.

There are two different questions involved when we ask whether matter
exists, and it is important to keep them clear. We commonly mean by
`matter' something which is opposed to
`mind', something which we think of as
occupying space and as radically incapable of any sort of thought or
consciousness. It is chiefly in this sense that Berkeley denies matter;
that is to say, he does not deny that the sense-data which we commonly
take as signs of the existence of the table are really signs of the
existence of \emph{something} independent of us, but he does deny that
this something is non-mental, that it is neither mind nor ideas
entertained by some mind. He admits that there must be something which
continues to exist when we go out of the room or shut our eyes, and that
what we call seeing the table does really give us reason for believing
in something which persists even when we are not seeing it. But he
thinks that this something cannot be radically different in nature from
what we see, and cannot be independent of seeing altogether, though it
must be independent of \emph{our} seeing. He is thus led to regard the
`real' table as an idea in the mind of
God. Such an idea has the required permanence and independence of
ourselves, without being---as matter would otherwise be---something
quite unknowable, in the sense that we can only infer it, and can never
be directly and immediately aware of it.

Other philosophers since Berkeley have also held that, although the
table does not depend for its existence upon being seen by me, it does
depend upon being seen (or otherwise apprehended in sensation) by
\emph{some} mind---not necessarily the mind of God, but more often the
whole collective mind of the universe. This they hold, as Berkeley does,
chiefly because they think there can be nothing real---or at any rate
nothing known to be real except minds and their thoughts and feelings.
We might state the argument by which they support their view in some
such way as this: `Whatever can be thought of is an idea
in the mind of the person thinking of it; therefore nothing can be
thought of except ideas in minds; therefore anything else is
inconceivable, and what is inconceivable cannot exist.'

Such an argument, in my opinion, is fallacious; and of course those who
advance it do not put it so shortly or so crudely. But whether valid or
not, the argument has been very widely advanced in one form or another;
and very many philosophers, perhaps a majority, have held that there is
nothing real except minds and their ideas. Such philosophers are called
`idealists'. When they come to explaining
matter, they either say, like Berkeley, that matter is really nothing
but a collection of ideas, or they say, like 
\href{https://plato.stanford.edu/entries/leibniz/}{Leibniz} (1646-1716), that
what appears as matter is really a collection of more or less
rudimentary minds.

But these philosophers, though they deny matter as opposed to mind,
nevertheless, in another sense, admit matter. It will be remembered that
we asked two questions; namely, (1) Is there a real table at all? (2) If
so, what sort of object can it be? Now both Berkeley and Leibniz admit
that there is a real table, but Berkeley says it is certain ideas in the
mind of God, and Leibniz says it is a colony of souls. Thus both of them
answer our first question in the affirmative, and only diverge from the
views of ordinary mortals in their answer to our second question. \label{Q1} In
fact, almost all philosophers seem to be agreed that there is a real
table: they almost all agree that, however much our sense-data---colour,
shape, smoothness, etc.---may depend upon us, yet their occurrence is a
sign of something existing independently of us, something differing,
perhaps, completely from our sense-data, and yet to be regarded as
causing those sense-data whenever we are in a suitable relation to the
real table.

Now obviously this point in which the philosophers are agreed---the view
that there \emph{is} a real table, whatever its nature may be---is
vitally important, and it will be worth while to consider what reasons
there are for accepting this view before we go on to the further
question as to the nature of the real table. Our next chapter,
therefore, will be concerned with the reasons for supposing that there
is a real table at all.

Before we go farther it will be well to consider for a moment what it is
that we have discovered so far. It has appeared that, if we take any
common object of the sort that is supposed to be known by the senses,
what the senses \emph{immediately} tell us is not the truth about the
object as it is apart from us, but only the truth about certain
sense-data which, so far as we can see, depend upon the relations
between us and the object. Thus what we directly see and feel is merely
`appearance', which we believe to be a
sign of some `reality' behind. But if
the reality is not what appears, have we any means of knowing whether
there is any reality at all? And if so, have we any means of finding out
what it is like?

Such questions are bewildering, and it is difficult to know that even
the strangest hypotheses may not be true. Thus our familiar table, which
has roused but the slightest thoughts in us hitherto, has become a
problem full of surprising possibilities. The one thing we know about it
is that it is not what it seems. Beyond this modest result, so far, we
have the most complete liberty of conjecture. Leibniz tells us it is a
community of souls: Berkeley tells us it is an idea in the mind of God;
sober science, scarcely less wonderful, tells us it is a vast collection
of electric charges in violent motion.

Among these surprising possibilities, doubt suggests that perhaps there
is no table at all. Philosophy, if it cannot \emph{answer} so many
questions as we could wish, has at least the power of \emph{asking}
questions which increase the interest of the world, and show the
strangeness and wonder lying just below the surface even in the
commonest things of daily life.

\protect\hypertarget{link2HCH0002}{}{}

\pagebreak

\section{Notes for the Student}
\markboth{CHAPTER 1 NOTES}{CHAPTER 1 NOTES}
\section*{A bit about philosophers}
Philosophers, like everyone else, often use arguments to critically study things around them. In fact philosophers are expected to use arguments professionally. But of course! Everyone uses arguments in daily life and as a vital part of whatever career they are in. And philosophers, like everyone you would \emph{want} as a friend, are (supposed to be) curious and invested in figuring out which beliefs are true. In these ways philosophers are just like everyone else (is supposed to be).

But philosophers are not like everyone else in that they take no particular claim for granted, and they do this professionally. This means that philosophers in their practice can critically discuss \emph{any given premise} in a particular argument. They usually critically discuss some premise when the person they are talking to either does not believe it is true or does not understand \emph{why} we think it is true. The latter case actually occurs more often than the first one. Most philosophers do not reject all the claims that we ordinarily take for granted. But they are curious about what we should say if we were really pressed on the subject. This is because, as Russell says (page \pageref{reasons}), philosophers want to know what is the case \textit{with reasons}. 

In other words, we philosophize about what we believe already and it is not in serious doubt. But philosophizing about what we believe, even confidently, is still valuable for two reasons. First, we might end up rejecting what we thought was true after careful reflection. This has often happened (haven't you ever changed your mind before - maybe even more than once?). Second, even if we come back to our starting point, the philosophical journey around our beliefs will still have helped us understand what we believe and why more clearly. 

Philosophers' openness to questioning everything has caused people to form two false views about philosophers. The first is that philosophers are very destructive and question everything we believe, always asking another `why' question, until nothing is left. But this comes from a misunderstanding of what good philosophers do. Arguing well requires finding premises that another person will accept. Any child can (and some do) reject every premise you give them. That is not philosophizing and takes no skill. It takes philosophical skill to offer premises you take for granted and lead you, step by undeniable step, to a conclusion that you perhaps did not intend to accept, or even one that you find unbelievable. Critical arguments, which are the only ones worth having, are always a conversation, even when you have them with just yourself. Remember what who the conversational argument is with, what is valuable about the conversational argument to its participants, and whether the conversational argument is worth continuing to you.

The second false view about philosophers is that philosophers achieve nothing: they just negatively question everything that even a reasonable person would deny. This could not be further from the truth. Philosophizing is the beating heart of every scientific discipline and practice---political science, physics, biology, chemistry, computer science, engineering, psychology, economics, medicine, mathematics, and so on. Sea change almost always occurs in any scientific discipline because someone had a philosophical thought that they wanted to lift off the ground. Philosophy moreover touches upon humanistic questions that themselves spring from that same deep well of human experience and feeling which inspires all activism, art, music, literature, poetry, film, and theater that makes life worth living. As long as humans have been around, the world has been littered with the achievements of philosophers, including scientific disciplines themselves, and with timeless humanistic and social feats, from the grandiose like the invention of democracy to the mundane like an ordinary existential crisis. 

To take an example of philosophy's achievements that you can (and probably did just today) fit into your hand: we have the universal computer (of which your laptop and cell phones are instances) because of Alan Turing. Turing's achievement owes deep intellectually debts to philosophers, including this book's author. First, Turing's work on the foundations of mathematics and computation occurred in the Cambridge mathematical logic tradition that Russell advanced with his co-authored \emph{Principia Mathematica}. Turing's motivation for working on universal computers partly came from the Incompleteness Theorems of Kurt G\"odel, who initially proved his famous theorems by encoding Russell's co-authored \emph{Principia Mathematica} in the natural numbers. Second, Turing's model for computability, the \emph{Turing machine}, comes from a philosophical analysis of what human beings can compute. This philosophical analysis led to a conjecture we now call \emph{Turing's thesis} that all computable processes (in the philosophical sense of `computable') are computable by Turing machines (whether Turing's model of a universal computer or an equivalent one). We do not know if this thesis is true or not---we have no counterexamples yet---but we do know that Turing's philosophizing, his philosophical analysis of computability, is what gave us the universal computer. 

So from now on, every time your phone notification disturbs you, remember that you have a philosopher to thank (or curse). Likewise, every time you vote, exercise your human rights, consider whether the needs of the many outweigh the needs of the few, and so on, you have a philosopher to thank. Thus philosophy is highly productive and not just destructive. Let us continue to doing it.
\section*{Arguments}
The most vivid example of how philosophers take no particular claim for granted is the problem of perception. How do we know that when we look around the world, or listen around it (in polite company), or sniff around it (in impolite company), that what we experience is really there? When you are walking with a friend, neither you nor your friend is bewildered by your perceptions. You both just take it for granted that when you see a tree, it is not a scam by the government or a corporation; it is not a dream or a hallucination; it is just a plain old tree.

Almost nobody seriously denies it. But it becomes surprisingly hard to come up with a really knockdown argument for trees just being a tree. This is because it is always \emph{possible}---at least logically possible---that your perceptions are leading you into false beliefs about the world. Think about any vivid dream you had that was fantastically or just plain fictional. You may have seen, heard, tasted, touched, or smelled things in your dream that were not there. It may have even felt almost as vivid and real as waking life. So, given that you \emph{can} be wrong about your perceptions in any given case, what makes you \emph{sure} beyond a reasonable doubt that the ones in waking life are not a clever trick being played upon you?

These questions can feel bewildering, especially if you are seriously considering them for the first time. Russell's first chapter helps us begin to understand how to approach these questions as philosophers do---through arguments.

An \emph{argument} is just a collection of claims. One of these claims is a conclusion. The rest of the claims are premises offered in support of the conclusion. If I tell you, ``You should do your philosophy reading because you promised to do so,'' this would be an argument. The conclusion is that you should do your philosophy reading. The (explicit) premise is that you promised to do your philosophy reading. The unstated (implicit) premise is that if you promised to do your philosophy reading, then you should do your philosophy reading.

To help us study arguments, let us introduce two common patterns of argument. The first is \emph{modus ponens}, and says that given a \textit{conditional} (if-then claim) and its \textit{antecedent} (the left side or `if' side of the if-then claim) as premises, then its \textit{consequent} (the right side or `then' side of the if-then claim) is also given as a conclusion.
\begin{center}
\textbf{Modus Ponens}
\end{center}
\begin{enumerate}
	\item If P is true, then Q is true. \hfill Premise
	\item P is true. \hfill Premise
	\item Thus, Q is true. \hfill 1, 2 MP
\end{enumerate}
The second is \emph{modus tollens}, and says that given a conditional and the negation of its consequent as premises, then the negation of its antecedent is also given as a conclusion.
\begin{center}
	\textbf{Modus Tollens}
\end{center}
\begin{enumerate}
	\item If P is true, then Q is true. \hfill Premise
	\item Q is not true. \hfill Premise
	\item Thus, P is not true. \hfill 1, 2 MT
\end{enumerate}
In restating a philosopher's argument, it is helpful to put arguments in one of these forms. This is because these argument forms are \textit{deductively valid}: it is impossible (in a logical sense) for the premises to be true while the conclusion is false. In an argument that has the special property of being valid, any world in which the premises are true is automatically one in which the conclusion is true. So in arguments that are deductively valid, the truth of the premises is enough to guarantee (in a logical sense) the truth of the conclusion. 

Valid arguments are only good hypothetically. They are as good as their premises. Consider this valid argument:
\begin{enumerate}
	\item All humans are mammals.
	\item All mammals are animals.
	\item So, all humans are animals.
\end{enumerate}
This seems like a great argument, but validity is not enough to make a good argument. This valid argument only counts as a good one because the premises are highly likely to be true. We have good reason to think that all humans are mammals and that all mammals are animals (from a generalizations that are well supported by repeated observation).

Now consider the following argument:
\begin{enumerate}
	\item All humans are fish.
	\item All fish have gills.
	\item So, all humans have gills.
\end{enumerate}
This argument is valid just like the previous one. It follows the exact same pattern of reasoning. But we have good reason to think that premise (1) is false (by direct observation). So it is a bad argument.

The key to a good argument is that you want two features: (1) it is valid; (2) it has premises that are true. Condition (1) can be readily met once you get some practice. Condition (2) is harder to meet because it can often be controversial whether the premises are in fact true.

Consider the following argument as an example:
\begin{enumerate}
	\item Daniel Tiger is a cat.
	\item All cats are good.
	\item So, Daniel Tiger is good.
\end{enumerate}
This is another valid argument pattern. So it meets condition (1) for being a good argument. What about condition (2)? Are all the premises true? This is harder to judge. Daniel Tiger is definitely a cat because he is a tiger. Is it really the case, though, that all cats are good? I am sure that a quick internet search will provide some evidence to the contrary. (All dogs are good, though, not to worry.)

Let us return to our first argument: ``You should do your philosophy reading because you promised to do so.'' Putting this into \emph{modus ponens} format can help us understand and analyze this argument. 
\begin{enumerate}
	\item You promised to do your philosophy reading.
	\item If you promised to do your philosophy reading, then you should do your philosophy reading.
	\item So, you should do your philosophy reading.
\end{enumerate}
Remember, the second premise was unstated (implicit) but the person offering this argument likely meant to link premise (1) to the conclusion using \emph{some} such claim like premise (2). This argument is definitely valid because it follows the \emph{modus ponens} pattern. Does this argument meet condition (2)? Are its premises true?

We can verify or falsify premise (1). We can ask whether you promised, or see if there is a record of your promising (like if you promised to do your reading in a courtroom or if you signed a document promising to do the reading). 

It is harder to verify premise (2). Can we imagine a case where you promised to do something but we thought that you could (or maybe should) break your promise? I may promise to grade your homework within two weeks of the due date, but if I get hit by a campus bus and lapse into a coma, we might think it is excusable for me to break my promise in that case (I could hardly do anything else). There are lots of trying circumstance we could imagine life throwing at someone that causes them not to do the reading. And in a lot of these cases, we would not think you should do your philosophy reading because you need to take care of yourself first.

So premise (2) looks false. But what about this alternative? (2a): ``If you promised to do your philosophy reading \emph{and} there was no cause that would understandably absolve you of keeping your promise, then you should do your philosophy reading.'' This looks like it is on the right track. Then there are philosophical questions about what sorts of causes absolve you of promise-keeping. But now the premise looks at least plausibly true: if we replace premise (2) with premise (2a), we have a good argument.

There are three morals to take away from this brief discussion of arguments. First, a good argument follows a valid pattern and has true premises. (Actually, there are good arguments, which are common in daily life, that are not quite valid but there is still strong support between the premises and conclusion. We will talk about those again much later in the book.) Second, if you are confronted with an argument of this kind---one that is deductively valid---the only way to reject its conclusion (if you find it to be one that should be rejected) is to argue that its premises are untrue. Third, putting arguments into the \emph{modus ponens} pattern can help us understand the argument better, both by making our conversation partner state for us what premises they left unstated in their thinking and by helping us identify and attend to those premises we do not agree to just yet.

Again, in a valid argument, it is necessary that the conclusion is true \textit{if} the premises are all true. So if you think that the conclusion is not true, then you have to consider which of the premises is not true. And the benefit of putting another philosopher's argument in a deductively valid form is to make clear what the premises are, so that you can critically consider them. It helps you understand and evaluate an argument, and to do so \textit{critically}---as a philosopher should.
\section*{Back to the book you were reading}
Let us now apply this to Russell's argumentation in Chapter 1. In Chapter 1, having made the appearance-reality distinction, Russell argues that how things appear to us is distinct from how things are. What is his argument for this view? Now Russell says the following (\pageref{evident}):
\begin{quote}
	It is evident from what we have found, that there is no colour which preeminently appears to be \textit{the} colour of the table, or even of any one particular part of the table---it appears to be of different colours from different points of view, and there is no reason for regarding some of these as more really its colour than others. [...] When, in ordinary life, we speak of \textit{the} colour of the table, we only mean the sort of colour which it will seem to have to a normal spectator from an ordinary point of view under usual conditions of light. But the other colours which appear under other conditions have just as good a right to be considered real; and therefore, to avoid favouritism, we are compelled to deny that, in itself, the table has any one particular colour.
\end{quote}
Now this is a blob of text. Let us try to put it in one of the above argument forms.
\begin{enumerate}
	\item The table appears differently from different points of view. \hfill Premise
	\item If the table appears differently from different points of view, then there is no good reason for regarding any one appearance of the table as its real one. \hfill Premise
	\item Thus, there is no good reason for regarding any one appearance of the table as its real one. \hfill 1, 2 MP
\end{enumerate}
This is really nice! Now we know which premises to attack if we want to rationally reject the conclusion (although we might perhaps be convinced by Russell's argument of the need to separate the table's appearances from how it really is). Nobody reasonable will reject (1). \\
\par But what about (2)? Might that be rejected? Here is an attempt:
\begin{quote}
	Maybe we do not need to identify \textit{one} of the appearances as the real one. Why not say that \textit{many} of the appearances---perhaps \textit{all} of them---are equally real? %So the real table just is all of its appearances?
\end{quote}
Russell does not address this objection. And there is a good reason why not. It leads to a really undesirable conclusion in the following way:
\begin{enumerate}
	\item If the appearances of the table are all the one real table, then the appearances are the same object. \hfill Premise
	\item If the appearances are the same object, then the appearances have all the same properties. \hfill Premise (Leibniz)
	\item The apperances do not have all the same properties. \hfill Premise
	\item Thus, the appearances are not the same same object. \hfill 2, 3 MT
	\item Thus, the appearances of the table are all the one real table. \hfill 1, 4 MT
\end{enumerate}
This is nice! Now we know what premises to attack, if any. Presumably, no reasonable person will dispute (3), nor even (2). So one needs to consider (1) in evaluating this argument.

Notice also that this argument has \textit{two} inferences. It has an intermediate conclusion in (4) before the final conclusion in (5). A lot of philosophical arguments work that way.

\hypertarget{chapter-ii.-the-existence-of-matter}{%
\chapter{The Existence of Matter}\label{chapter-ii.-the-existence-of-matter}}

In this chapter we have to ask ourselves whether, in any sense at all,
there is such a thing as matter. Is there a table which has a certain
intrinsic nature, and continues to exist when I am not looking, or is
the table merely a product of my imagination, a dream-table in a very
prolonged dream? This question is of the greatest importance. For if we
cannot be sure of the independent existence of objects, we cannot be
sure of the independent existence of other people's
bodies, and therefore still less of other people's
minds, since we have no grounds for believing in their minds except such
as are derived from observing their bodies. Thus if we cannot be sure of
the independent existence of objects, we shall be left alone in a
desert---it may be that the whole outer world is nothing but a dream,
and that we alone exist. This is an uncomfortable possibility; but
although it cannot be strictly proved to be false, there is not the
slightest reason to suppose that it is true. \label{proof} In this chapter we have to
see why this is the case.

Before we embark upon doubtful matters, let us try to find some more or
less fixed point from which to start. Although we are doubting the
physical existence of the table, we are not doubting the existence of
the sense-data which made us think there was a table; we are not
doubting that, while we look, a certain colour and shape appear to us,
and while we press, a certain sensation of hardness is experienced by
us. All this, which is psychological, we are not calling in question. In
fact, whatever else may be doubtful, some at least of our immediate
experiences seem absolutely certain.

\href{https://plato.stanford.edu/entries/descartes/}{Descartes} (1596-1650), the founder of modern philosophy, invented a
method which may still be used with profit---the method of systematic
doubt. \label{doubt} He determined that he would believe nothing which he did not see
quite clearly and distinctly to be true. Whatever he could bring himself
to doubt, he would doubt, until he saw reason for not doubting it. By
applying this method he gradually became convinced that the only
existence of which he could be \emph{quite} certain was his own. He
imagined a deceitful demon, who presented unreal things to his senses in
a perpetual phantasmagoria; it might be very improbable that such a
demon existed, but still it was possible, and therefore doubt concerning
things perceived by the senses was possible.

But doubt concerning his own existence was not possible, for if he did
not exist, no demon could deceive him. If he doubted, he must exist; if
he had any experiences whatever, he must exist. Thus his own existence
was an absolute certainty to him. `I think, therefore I
am,' he said (\emph{Cogito, ergo sum}); and on the
basis of this certainty he set to work to build up again the world of
knowledge which his doubt had laid in ruins. By inventing the method of
doubt, and by showing that subjective things are the most certain,
Descartes performed a great service to philosophy, and one which makes
him still useful to all students of the subject.

But some care is needed in using Descartes' argument.
`I think, therefore I am' says rather
more than is strictly certain. It might seem as though we were quite
sure of being the same person to-day as we were yesterday, and this is
no doubt true in some sense. But the real Self is as hard to arrive at
as the real table, and does not seem to have that absolute, convincing
certainty that belongs to particular experiences. \label{self} When I look at my
table and see a certain brown colour, what is quite certain at once is
not '\emph{I} am seeing a brown colour',
but rather, `a brown colour is being
seen'. This of course involves something (or somebody)
which (or who) sees the brown colour; but it does not of itself involve
that more or less permanent person whom we call
`I'. So far as immediate certainty goes,
it might be that the something which sees the brown colour is quite
momentary, and not the same as the something which has some different
experience the next moment.

Thus it is our particular thoughts and feelings that have primitive
certainty. And this applies to dreams and hallucinations as well as to
normal perceptions: when we dream or see a ghost, we certainly do have
the sensations we think we have, but for various reasons it is held that
no physical object corresponds to these sensations. Thus the certainty
of our knowledge of our own experiences does not have to be limited in
any way to allow for exceptional cases. Here, therefore, we have, for
what it is worth, a solid basis from which to begin our pursuit of
knowledge.

The problem we have to consider is this: Granted that we are certain of
our own sense-data, have we any reason for regarding them as signs of
the existence of something else, which we can call the physical object?
When we have enumerated all the sense-data which we should naturally
regard as connected with the table, have we said all there is to say
about the table, or is there still something else---something not a
sense-datum, something which persists when we go out of the room? Common
sense unhesitatingly answers that there is. What can be bought and sold
and pushed about and have a cloth laid on it, and so on, cannot be a
\emph{mere} collection of sense-data. If the cloth completely hides the
table, we shall derive no sense-data from the table, and therefore, if
the table were merely sense-data, it would have ceased to exist, and the
cloth would be suspended in empty air, resting, by a miracle, in the
place where the table formerly was. This seems plainly absurd; but
whoever wishes to become a philosopher must learn not to be frightened
by absurdities.

One great reason why it is felt that we must secure a physical object in
addition to the sense-data, is that we want the same object for
different people. When ten people are sitting round a dinner-table, it
seems preposterous to maintain that they are not seeing the same
tablecloth, the same knives and forks and spoons and glasses. But the
sense-data are private to each separate person; what is immediately
present to the sight of one is not immediately present to the sight of
another: they all see things from slightly different points of view, and
therefore see them slightly differently. Thus, if there are to be public
neutral objects, which can be in some sense known to many different
people, there must be something over and above the private and
particular sense-data which appear to various people. What reason, then,
have we for believing that there are such public neutral objects?

The first answer that naturally occurs to one is that, although
different people may see the table slightly differently, still they all
see more or less similar things when they look at the table, and the
variations in what they see follow the laws of perspective and
reflection of light, so that it is easy to arrive at a permanent object
underlying all the different people's sense-data. I
bought my table from the former occupant of my room; I could not buy
\emph{his} sense-data, which died when he went away, but I could and did
buy the confident expectation of more or less similar sense-data. Thus
it is the fact that different people have similar sense-data, and that
one person in a given place at different times has similar sense-data,
which makes us suppose that over and above the sense-data there is a
permanent public object which underlies or causes the sense-data of
various people at various times.

Now in so far as the above considerations depend upon supposing that
there are other people besides ourselves, they beg the very question at
issue. Other people are represented to me by certain sense-data, such as
the sight of them or the sound of their voices, and if I had no reason
to believe that there were physical objects independent of my
sense-data, I should have no reason to believe that other people exist
except as part of my dream. Thus, when we are trying to show that there
must be objects independent of our own sense-data, we cannot appeal to
the testimony of other people, since this testimony itself consists of
sense-data, and does not reveal other people's
experiences unless our own sense-data are signs of things existing
independently of us. We must therefore, if possible, find, in our own
purely private experiences, characteristics which show, or tend to show,
that there are in the world things other than ourselves and our private
experiences.

In one sense it must be admitted that we can never prove the existence
of things other than ourselves and our experiences. No logical absurdity
results from the hypothesis that the world consists of myself and my
thoughts and feelings and sensations, and that everything else is mere
fancy. In dreams a very complicated world may seem to be present, and
yet on waking we find it was a delusion; that is to say, we find that
the sense-data in the dream do not appear to have corresponded with such
physical objects as we should naturally infer from our sense-data. (It
is true that, when the physical world is assumed, it is possible to find
physical causes for the sense-data in dreams: a door banging, for
instance, may cause us to dream of a naval engagement. \label{navy} But although, in
this case, there is a physical cause for the sense-data, there is not a
physical object corresponding to the sense-data in the way in which an
actual naval battle would correspond.) There is no logical impossibility
in the supposition that the whole of life is a dream, in which we
ourselves create all the objects that come before us. But although this
is not logically impossible, there is no reason whatever to suppose that
it is true; and it is, in fact, a less simple hypothesis, viewed as a
means of accounting for the facts of our own life, than the common-sense
hypothesis that there really are objects independent of us, whose action
on us causes our sensations.

The way in which simplicity comes in from supposing that there really
are physical objects is easily seen. \label{kitty} If the cat appears at one moment in
one part of the room, and at another in another part, it is natural to
suppose that it has moved from the one to the other, passing over a
series of intermediate positions. But if it is merely a set of
sense-data, it cannot have ever been in any place where I did not see
it; thus we shall have to suppose that it did not exist at all while I
was not looking, but suddenly sprang into being in a new place. If the
cat exists whether I see it or not, we can understand from our own
experience how it gets hungry between one meal and the next; but if it
does not exist when I am not seeing it, it seems odd that appetite
should grow during non-existence as fast as during existence. And if the
cat consists only of sense-data, it cannot be hungry, since no hunger
but my own can be a sense-datum to me. Thus the behaviour of the
sense-data which represent the cat to me, though it seems quite natural
when regarded as an expression of hunger, becomes utterly inexplicable
when regarded as mere movements and changes of patches of colour, which
are as incapable of hunger as a triangle is of playing football.

But the difficulty in the case of the cat is nothing compared to the
difficulty in the case of human beings. When human beings speak---that
is, when we hear certain noises which we associate with ideas, and
simultaneously see certain motions of lips and expressions of face---it
is very difficult to suppose that what we hear is not the expression of
a thought, as we know it would be if we emitted the same sounds. Of
course similar things happen in dreams, where we are mistaken as to the
existence of other people. But dreams are more or less suggested by what
we call waking life, and are capable of being more or less accounted for
on scientific principles if we assume that there really is a physical
world. Thus every principle of simplicity urges us to adopt the natural
view, that there really are objects other than our selves and our
sense-data which have an existence not dependent upon our perceiving
them.

Of course it is not by argument that we originally come by our belief in
an independent external world. We find this belief ready in ourselves as
soon as we begin to reflect: it is what may be called an
\emph{instinctive} belief. \label{instinctive} We should never have been led to question
this belief but for the fact that, at any rate in the case of sight, it
seems as if the sense-datum itself were instinctively believed to be the
independent object, whereas argument shows that the object cannot be
identical with the sense-datum. This discovery, however---which is not
at all paradoxical in the case of taste and smell and sound, and only
slightly so in the case of touch---leaves undiminished our instinctive
belief that there \emph{are} objects \emph{corresponding} to our
sense-data. Since this belief does not lead to any difficulties, but on
the contrary tends to simplify and systematize our account of our
experiences, there seems no good reason for rejecting it. We may
therefore admit---though with a slight doubt derived from dreams---that
the external world does really exist, and is not wholly dependent for
its existence upon our continuing to perceive it.

The argument which has led us to this conclusion is doubtless less
strong than we could wish, but it is typical of many philosophical
arguments, and it is therefore worth while to consider briefly its
general character and validity. All knowledge, we find, must be built up
upon our instinctive beliefs, and if these are rejected, nothing is
left. But among our instinctive beliefs some are much stronger than
others, while many have, by habit and association, become entangled with
other beliefs, not really instinctive, but falsely supposed to be part
of what is believed instinctively.

Philosophy should show us the hierarchy of our instinctive beliefs,
beginning with those we hold most strongly, and presenting each as much
isolated and as free from irrelevant additions as possible. It should
take care to show that, in the form in which they are finally set forth,
our instinctive beliefs do not clash, but form a harmonious system.
There can never be any reason for rejecting one instinctive belief
except that it clashes with others; thus, if they are found to
harmonize, the whole system becomes worthy of acceptance.

It is of course \emph{possible} that all or any of our beliefs may be
mistaken, and therefore all ought to be held with at least some slight
element of doubt. But we cannot have \emph{reason} to reject a belief
except on the ground of some other belief. Hence, by organizing our
instinctive beliefs and their consequences, by considering which among
them it is most possible, if necessary, to modify or abandon, we can
arrive, on the basis of accepting as our sole data what we instinctively
believe, at an orderly systematic organization of our knowledge, in
which, though the \emph{possibility} of error remains, its likelihood is
diminished by the interrelation of the parts and by the critical
scrutiny which has preceded acquiescence.

This function, at least, philosophy can perform. Most philosophers,
rightly or wrongly, believe that philosophy can do much more than
this---that it can give us knowledge, not otherwise attainable,
concerning the universe as a whole, and concerning the nature of
ultimate reality. Whether this be the case or not, the more modest
function we have spoken of can certainly be performed by philosophy, and
certainly suffices, for those who have once begun to doubt the adequacy
of common sense, to justify the arduous and difficult labours that
philosophical problems involve.

\protect\hypertarget{link2HCH0003}{}{}

\pagebreak
\section{Notes for the Student} 
\markboth{CHAPTER 2 NOTES}{CHAPTER 2 NOTES}
\par Recall Russell's argument, which we put into the \emph{modus ponens} pattern below, for an object's appearances being distinct from its underlying reality:
\begin{enumerate}
	\item The table appears differently from different points of view.\hfill Premise
	\item If the table appears differently from different points of view, then there is no good reason for regarding any one appearance of the table as its real one.\hfill Premise
	\item Thus, there is no good reason for regarding any one appearance of the table as its real one. \hfill 1, 2 MP
\end{enumerate}
Russell now asks us to consider whether there is any underlying reality at all. This is his question (1) Is there any such thing as matter? Now most philosophers, like most people, believe that there is an underlying reality. They just disagree over what it is like (\pageref{Q1}).
\par Russell praises Descrates for ``the method of systematic doubt" because, besides being a great method, it shows ``that the subjective things are the most certain" (\pageref{doubt}).
\par What is this method, then? Here is Descrates in the \textit{Meditations}:
\begin{quote}
	Yet I have found that these senses sometimes deceive me, and it is a matter of prudence never to confide completely in those who have deceived us even once...I am forced, finally, to concede that of the things which I once held to be true there is none that it would not be [rationally] permitted to doubt...And therefore I am forced to concede that from now on assent is accurately to be withheld from the same things too no less than from the overtly false ones, if I would want to find something certain. (\textit{First Meditation})
\end{quote}
We can restate \emph{Descartes' dream argument}, in the valid \textit{modus ponens} argument-form, as follows:
\begin{enumerate}
	\item If what my senses tell me is sometimes wrong, then anything my senses tell me can be reasonably doubted. \hfill Premise
	\item What my senses tell me is sometimes wrong [as in illusions, etc.]. \hfill Premise
	\item Thus, anything my senses tell me can be reasonably doubted. \hfill 1, 2 MP
	\item If anything my senses tell me can be reasonably doubted, then nothing my senses tell me is certain. \hfill Premise
	\item Thus, nothing my senses tell me is certain. \hfill 3, 4 MP
\end{enumerate}
This conclusion teeters towards a \textit{global skepticism}, on which there are no justified beliefs. Descartes escapes global skepticism through the subjective experiences that are most certain:
\begin{quote}
	No, if I was persuading myself of something, then certainly \textit{I} was...Here I find: it is cogitations: this alone cannot be rent from me. \textit{I} am, \textit{I} exist; it is certain. But for how long? So long as I am cogitating, of course...But what, then, am I? A cogitating thing. What is that? A thing doubting, understanding, affirming, denying, willing, not willing, also imagining, of course. (\textit{Second Meditation})
\end{quote}
Russell agrees with Descartes so far. But then Descartes infers that in all those subjective experiences, it is the same thing---the same \textit{I}---that undergoes those experiences:
\begin{quote}
	What is there of these things that might be distinguished from my cogitation? What is there of these things that could be called ``separate" from me myself?...But truly \textit{I} am also the same one who imagines...\textit{I} am the same one who senses...I seem to see, I seem to hear, I seem to be warmed. (\textit{Second Meditation})
\end{quote}
Russell does not agree with Descartes' argument. We can restate the reasoning as follows:
\begin{enumerate}
	\item If there is cogitation, then there is a thing that cogitates. \hfill Premise
	\item There is cogitation [doubting, thinking, etc.]. \hfill Premise
	\item Thus, there is a thing that cogitates. \hfill 1, 2 MP
	\item The word `I' refers to a thing that cogitates on different occasions. \hfill Premise
	\item If there is a thing that cogitates and the word `I' refers to it on different occasions, then a single self has one's cogitations. \hfill Premise
	\item Thus, a single self has one's cogitations. \hfill 3\&4, 5 MP
\end{enumerate}
And (4) is support by different, seemingly correct uses of `I' to pick out the same entity. But Russell argues that (4) is not supported by the different, seemingly correct uses of `I':
\begin{quote}
	When I look at my table and see a certain brown colour, what is quite certain at once is not `\textit{I} am seeing a brown colour', but rather, `a brown colour is being seen'. This of course involves something (or somebody) which (or who) sees the brown colour; but it does not itself involve that more or less permanent person whom we call `I'. So far as immediate certainty goes, it might be that the something which sees the brown colour is quite momentary, and not the same as the something which has some different experience the next moment. (\pageref{self})
\end{quote}
Buddhists like Vasubandhu (400s CE) gave a similar argument some centuries before Russell:
%\par Notice also that this argument has \textit{two} inferences. It has an intermediate conclusion in (4) before the final conclusion in (5). A lot of philosophical arguments work that way.
\begin{quote}
	It is known that the expression, ``self," refers to a continuum of aggregates and not to anything else because there is no direct perception or correct inference...to a self. Therefore, there is no self. (\textit{Refutation of the Theory of a Self}, \S1)
\end{quote}
Exploring these theories of self is a great paper topic! You can try to answer, `Is there a permanent self underlying my experiences?' and discuss some of these varying views.

\hypertarget{chapter-iii.-the-nature-of-matter}{%
\chapter{The Nature of Matter}\label{chapter-iii.-the-nature-of-matter}}

In the preceding chapter we agreed, though without being able to find
demonstrative reasons, that it is rational to believe that our
sense-data---for example, those which we regard as associated with my
table---are really signs of the existence of something independent of us
and our perceptions. That is to say, over and above the sensations of
colour, hardness, noise, and so on, which make up the appearance of the
table to me, I assume that there is something else, of which these
things are appearances. The colour ceases to exist if I shut my eyes,
the sensation of hardness ceases to exist if I remove my arm from
contact with the table, the sound ceases to exist if I cease to rap the
table with my knuckles. But I do not believe that when all these things
cease the table ceases. On the contrary, I believe that it is because
the table exists continuously that all these sense-data will reappear
when I open my eyes, replace my arm, and begin again to rap with my
knuckles. The question we have to consider in this chapter is: What is
the nature of this real table, which persists independently of my
perception of it?

To this question physical science gives an answer, somewhat incomplete
it is true, and in part still very hypothetical, but yet deserving of
respect so far as it goes. Physical science, more or less unconsciously,
has drifted into the view that all natural phenomena ought to be reduced
to motions. Light and heat and sound are all due to wave-motions, which
travel from the body emitting them to the person who sees light or feels
heat or hears sound. That which has the wave-motion is either aether or
`gross matter', but in either case is
what the philosopher would call matter. The only properties which
science assigns to it are position in space, and the power of motion
according to the laws of motion. Science does not deny that it
\emph{may} have other properties; but if so, such other properties are
not useful to the man of science, and in no way assist him in explaining
the phenomena.

It is sometimes said that `light \emph{is} a form of
wave-motion', but this is misleading, for the light which
we immediately see, which we know directly by means of our senses, is
\emph{not} a form of wave-motion, but something quite
different---something which we all know if we are not blind, though we
cannot describe it so as to convey our knowledge to a man who is blind.
A wave-motion, on the contrary, could quite well be described to a blind
man, since he can acquire a knowledge of space by the sense of touch;
and he can experience a wave-motion by a sea voyage almost as well as we
can. But this, which a blind man can understand, is not what we mean by
\emph{light}: we mean by \emph{light} just that which a blind man can
never understand, and which we can never describe to him.

Now this something, which all of us who are not blind know, is not,
according to science, really to be found in the outer world: it is
something caused by the action of certain waves upon the eyes and nerves
and brain of the person who sees the light. When it is said that light
\emph{is} waves, what is really meant is that waves are the physical
cause of our sensations of light. But light itself, the thing which
seeing people experience and blind people do not, is not supposed by
science to form any part of the world that is independent of us and our
senses. \label{light} And very similar remarks would apply to other kinds of
sensations.

It is not only colours and sounds and so on that are absent from the
scientific world of matter, but also \emph{space} as we get it through
sight or touch. It is essential to science that its matter should be in
\emph{a} space, but the space in which it is cannot be exactly the space
we see or feel. To begin with, space as we see it is not the same as
space as we get it by the sense of touch; it is only by experience in
infancy that we learn how to touch things we see, or how to get a sight
of things which we feel touching us. But the space of science is neutral
as between touch and sight; thus it cannot be either the space of touch
or the space of sight.

Again, different people see the same object as of different shapes,
according to their point of view. A circular coin, for example, though
we should always \emph{judge} it to be circular, will \emph{look} oval
unless we are straight in front of it. When we judge that it \emph{is}
circular, we are judging that it has a real shape which is not its
apparent shape, but belongs to it intrinsically apart from its
appearance. But this real shape, which is what concerns science, must be
in a real space, not the same as anybody's
\emph{apparent} space. The real space is public, the apparent space is
private to the percipient. In different people's
\emph{private} spaces the same object seems to have different shapes;
thus the real space, in which it has its real shape, must be different
from the private spaces. The space of science, therefore, though
\emph{connected} with the spaces we see and feel, is not identical with
them, and the manner of its connexion requires investigation.

We agreed provisionally that physical objects cannot be quite like our
sense-data, but may be regarded as \emph{causing} our sensations. These
physical objects are in the space of science, which we may call
`physical' space. It is important to
notice that, if our sensations are to be caused by physical objects,
there must be a physical space containing these objects and our
sense-organs and nerves and brain. We get a sensation of touch from an
object when we are in contact with it; that is to say, when some part of
our body occupies a place in physical space quite close to the space
occupied by the object. We see an object (roughly speaking) when no
opaque body is between the object and our eyes in physical space.
Similarly, we only hear or smell or taste an object when we are
sufficiently near to it, or when it touches the tongue, or has some
suitable position in physical space relatively to our body. We cannot
begin to state what different sensations we shall derive from a given
object under different circumstances unless we regard the object and our
body as both in one physical space, for it is mainly the relative
positions of the object and our body that determine what sensations we
shall derive from the object.

\label{private} Now our sense-data are situated in our private spaces, either the space
of sight or the space of touch or such vaguer spaces as other senses may
give us. \label{onespace} If, as science and common sense assume, there is one public
all-embracing physical space in which physical objects are, the relative
positions of physical objects in physical space must more or less
correspond to the relative positions of sense-data in our private
spaces. There is no difficulty in supposing this to be the case. \label{distance} If we
see on a road one house nearer to us than another, our other senses will
bear out the view that it is nearer; for example, it will be reached
sooner if we walk along the road. Other people will agree that the house
which looks nearer to us is nearer; the ordnance map will take the same
view; and thus everything points to a spatial relation between the
houses corresponding to the relation between the sense-data which we see
when we look at the houses. Thus we may assume that there is a physical
space in which physical objects have spatial relations corresponding to
those which the corresponding sense-data have in our private spaces. It
is this physical space which is dealt with in geometry and assumed in
physics and astronomy.

Assuming that there is physical space, and that it does thus correspond
to private spaces, what can we know about it? We can know \emph{only}
what is required in order to secure the correspondence. That is to say,
we can know nothing of what it is like in itself, but we can know the
sort of arrangement of physical objects which results from their spatial
relations. We can know, for example, that the earth and moon and sun are
in one straight line during an eclipse, though we cannot know what a
physical straight line is in itself, as we know the look of a straight
line in our visual space. Thus we come to know much more about the
\emph{relations} of distances in physical space than about the distances
themselves; we may know that one distance is greater than another, or
that it is along the same straight line as the other, but we cannot have
that immediate acquaintance with physical distances that we have with
distances in our private spaces, or with colours or sounds or other
sense-data. We can know all those things about physical space which a
man born blind might know through other people about the space of sight;
but the kind of things which a man born blind could never know about the
space of sight we also cannot know about physical space. We can know the
properties of the relations required to preserve the correspondence with
sense-data, but we cannot know the nature of the terms between which the
relations hold.

\label{timeorder} With regard to time, our \emph{feeling} of duration or of the lapse of
time is notoriously an unsafe guide as to the time that has elapsed by
the clock. Times when we are bored or suffering pain pass slowly, times
when we are agreeably occupied pass quickly, and times when we are
sleeping pass almost as if they did not exist. Thus, in so far as time
is constituted by duration, there is the same necessity for
distinguishing a public and a private time as there was in the case of
space. But in so far as time consists in an \emph{order} of before and
after, there is no need to make such a distinction; the time-order which
events seem to have is, so far as we can see, the same as the time-order
which they do have. At any rate no reason can be given for supposing
that the two orders are not the same. The same is usually true of space:
if a regiment of men are marching along a road, the shape of the
regiment will look different from different points of view, but the men
will appear arranged in the same order from all points of view. Hence we
regard the order as true also in physical space, whereas the shape is
only supposed to correspond to the physical space so far as is required
for the preservation of the order.

In saying that the time-order which events seem to have is the same as
the time-order which they really have, it is necessary to guard against
a possible misunderstanding. It must not be supposed that the various
states of different physical objects have the same time-order as the
sense-data which constitute the perceptions of those objects. \label{thunder} Considered
as physical objects, the thunder and lightning are simultaneous; that is
to say, the lightning is simultaneous with the disturbance of the air in
the place where the disturbance begins, namely, where the lightning is.
But the sense-datum which we call hearing the thunder does not take
place until the disturbance of the air has travelled as far as to where
we are. Similarly, it takes about eight minutes for the
sun's light to reach us; thus, when we see the sun we
are seeing the sun of eight minutes ago. So far as our sense-data afford
evidence as to the physical sun they afford evidence as to the physical
sun of eight minutes ago; if the physical sun had ceased to exist within
the last eight minutes, that would make no difference to the sense-data
which we call `seeing the sun'. This
affords a fresh illustration of the necessity of distinguishing between
sense-data and physical objects.

What we have found as regards space is much the same as what we find in
relation to the correspondence of the sense-data with their physical
counterparts. \label{similarity} If one object looks blue and another red, we may
reasonably presume that there is some corresponding difference between
the physical objects; if two objects both look blue, we may presume a
corresponding similarity. But we cannot hope to be acquainted directly
with the quality in the physical object which makes it look blue or red.
Science tells us that this quality is a certain sort of wave-motion, and
this sounds familiar, because we think of wave-motions in the space we
see. But the wave-motions must really be in physical space, with which
we have no direct acquaintance; thus the real wave-motions have not that
familiarity which we might have supposed them to have. And what holds
for colours is closely similar to what holds for other sense-data. \label{relations} Thus
we find that, although the \emph{relations} of physical objects have all
sorts of knowable properties, derived from their correspondence with the
relations of sense-data, the physical objects themselves remain unknown
in their intrinsic nature, so far at least as can be discovered by means
of the senses. The question remains whether there is any other method of
discovering the intrinsic nature of physical objects.

The most natural, though not ultimately the most defensible, hypothesis
to adopt in the first instance, at any rate as regards visual
sense-data, would be that, though physical objects cannot, for the
reasons we have been considering, be \emph{exactly} like sense-data, yet
they may be more or less like. According to this view, physical objects
will, for example, really have colours, and we might, by good luck, see
an object as of the colour it really is. The colour which an object
seems to have at any given moment will in general be very similar,
though not quite the same, from many different points of view; we might
thus suppose the `real' colour to be a
sort of medium colour, intermediate between the various shades which
appear from the different points of view.

Such a theory is perhaps not capable of being definitely refuted, but it
can be shown to be groundless. To begin with, it is plain that the
colour we see depends only upon the nature of the light-waves that
strike the eye, and is therefore modified by the medium intervening
between us and the object, as well as by the manner in which light is
reflected from the object in the direction of the eye. The intervening
air alters colours unless it is perfectly clear, and any strong
reflection will alter them completely. Thus the colour we see is a
result of the ray as it reaches the eye, and not simply a property of
the object from which the ray comes. Hence, also, provided certain waves
reach the eye, we shall see a certain colour, whether the object from
which the waves start has any colour or not. Thus it is quite gratuitous
to suppose that physical objects have colours, and therefore there is no
justification for making such a supposition. Exactly similar arguments
will apply to other sense-data.

It remains to ask whether there are any general philosophical arguments
enabling us to say that, if matter is real, it must be of such and such
a nature. As explained above, very many philosophers, perhaps most, have
held that whatever is real must be in some sense mental, or at any rate
that whatever we can know anything about must be in some sense mental.
Such philosophers are called `idealists'.
Idealists tell us that what appears as matter is really something
mental; namely, either (as Leibniz held) more or less rudimentary minds,
or (as Berkeley contended) ideas in the minds which, as we should
commonly say, `perceive' the matter.
Thus idealists deny the existence of matter as something intrinsically
different from mind, though they do not deny that our sense-data are
signs of something which exists independently of our private sensations.
In the following chapter we shall consider briefly the reasons---in my
opinion fallacious---which idealists advance in favour of their theory.

\protect\hypertarget{link2HCH0004}{}{}
\pagebreak
\section{Notes for the Student}
\markboth{CHAPTER 3 NOTES}{CHAPTER 3 NOTES}
\par Russell argued in Chapter 2 that matter exists. He notes that it is \textit{logically consistent} with all of our appearances that there is no underlying reality to the table, and that the world consists solely of one's own fleeting appearances (and no other minds with their own appearances!). This is why we cannot \textit{prove} that matter exists (\pageref{self}).
\par But saying something is logically consistent is a weak claim: my being a god is \textit{logically consistent} with everything that you know about me. But it is highly unlikely.
\par Now recall Russell's argument from kitties for holding that matter exists (\pageref{kitty}):
\begin{enumerate}%[Ch 2: Matter Exists]
	\item If appearances are all there is, then hungry kitty-appearances are just inexplicable [for they are unconnected to other kitty-appearances]. \hfill Premise
	\item It is not true that hungry kitty-appearances are just inexplicable. \hfill Premise
	\item Thus, appearances are not all there is. \hfill 1, 2 MT
	\item If appearances are not all there is, then there is also matter. \hfill Premise
	\item Thus, there is also matter. \hfill 3, 4 MP
\end{enumerate}
So the existence of matter is an explanation for what would otherwise lack any explanation: how a kitty gets hungry after a while. We would otherwise have to say, `My kitty-appearances are hungry now' without seeing why this would be the case (the appearances just came into existence, and they do not exist a long time---why should they be hungry (or satiated!)?).
\par This same sort of argument works for all kinds of \textit{instinctive beliefs}, as Russell calls them (\pageref{instinctive}). Before we do philosophy, we have all sorts of beliefs about the reliability of our senses, the existence of matter, our persistence through time, and so on. Absent a good argument \textit{against} those beliefs, we can still hold them. But notice that we are only rationally entitled to do this \textit{after} we study these instinctive beliefs closely. For once we consider them philosophically, they might undergo a lot of revision or change their meaning entirely. 
\par So matter exists. We answer Russell's first question in the affirmative. But what is matter like? This is the more controversial question, and is taken up in Chapters 3-4. 
\par First, Russell argues that matter does not belong to our private sensory spaces of sight, touch, etc. (\pageref{light}) The public or scientific space of matter is \textit{correlated} with these private spaces, and we construct it using the data of our private space plus this \textit{correlation}.
\par First, Russell assumes that there is \textit{one} physical (public, scientific) space, and then he argues that physical objects in it \textit{correlate} with sense-data in private spaces:
\begin{quote}
	If, as science and common-sense assume, there is one public all-embracing physical space in which physical objects are, the relative positions of physical objects must more or less correspond to the relative positions of sense-data in our private spaces. There is no difficulty in supposing this...everything points to a spatial relation between the houses corresponding to the relation between the sense-data which we have when we look at the houses. Thus we may assume that there is a physical space in which physical objects have spatial relations corresponding to those which the corresponding sense-data have in our private spaces. \pageref{onespace}
\end{quote}
If each private space is correlated with a different physical space, and then we have \textit{no way} of correlating our respective physical spaces \textit{with each other}. But we make such correlations all the time! This is \textit{best explained} by \textit{one} physical space. And given this assumption, we can correlate our private spaces with that one space, or so Russell argues as follows:
\begin{enumerate} %There is just one physical space.
	\item If there is one physical space with which we correlate our private spaces, then the relative positions of physical objects in physical space are more or less correlated to those of sense-data in each private space. \hfill Premise
	\item There is one physical space with which we correlate our private spaces. \hfill Premise
	\item Thus, the relative positions of physical objects in physical space are more or less correlated to those of sense-data in each private space. \hfill 1, 2 MP
\end{enumerate}
So we have shown that each private space is correlated with the one public, physical space.
\par Given that there is this correspondence, what can we know about this physical space that we never see, touch, hear, taste, or smell? Russell's answer is that we can know about it whatever is assumed in its corresponding to private space. Take two appearances: as a rule, whatever relational structure (of quality similarity, spatial distance, or time-order) they have, the physical objects also have whatever features are required to preserve this correspondence. More precisely:
\begin{itemize}
	\item If $X$ is closer than $Y$, then we know that the physical object corresponding to $X$ is closer than the one that corresponds to $Y$ (\pageref{distance}). 
	\item Likewise, if $X$ appears at a time earlier than $Y$, then the physical object corresponding to $X$ causes the appearance earlier than the one that corresponds to $Y$ (\pageref{timeorder}).
	\item If two appearances $X$ and $Y$ are qualitatively different, we know the physical objects that cause them are similarly different from each other (though we do not know what the physical objects are like (\pageref{similarity}).
\end{itemize}
Now these rules have exceptions. Sometimes an appearance, like a hallucination, does not correspond in the usual way to a physical object, and we can dream about a sea battle if we hear a door shut while asleep (\pageref{navy}). The time-order of appearances do not always correspond to the time-order in physical space, as when something feels like it lasts forever but does not, or as when lightning appears before thunder but both are caused by the same event (\pageref{thunder}). But there is some order in physical space that corresponds to the order in each of our private spaces, and we usually take disruptions of the order in appearances as reason to investigate further, with more appearances, to determine what the `right' order is. 
\par To summarize all this: the \textit{relational structure} of the physical world is what we can infer from appearances: differences, distances, and durations among appearances are correlated with differences, distances, and durations in the physical world:
\begin{quote} 
	Thus we find that, although the \textit{relations} of physical objects have all sorts of knowable properties, derived from their correspondence with the relations of sense-data, the physical objects themselves remain unknown in their intrinsic nature, so far at least as can be discovered by means of the senses. (\pageref{relations})
\end{quote}

\hypertarget{chapter-iv.-idealism}{%
\chapter{Idealism}\label{chapter-iv.-idealism}}

The word `idealism' is used by
different philosophers in somewhat different senses. \label{idealism} We shall understand
by it the doctrine that whatever exists, or at any rate whatever can be
known to exist, must be in some sense mental. This doctrine, which is
very widely held among philosophers, has several forms, and is advocated
on several different grounds. The doctrine is so widely held, and so
interesting in itself, that even the briefest survey of philosophy must
give some account of it.

Those who are unaccustomed to philosophical speculation may be inclined
to dismiss such a doctrine as obviously absurd. There is no doubt that
common sense regards tables and chairs and the sun and moon and material
objects generally as something radically different from minds and the
contents of minds, and as having an existence which might continue if
minds ceased. We think of matter as having existed long before there
were any minds, and it is hard to think of it as a mere product of
mental activity. But whether true or false, idealism is not to be
dismissed as obviously absurd.

We have seen that, even if physical objects do have an independent
existence, they must differ very widely from sense-data, and can only
have a \emph{correspondence} with sense-data, in the same sort of way in
which a catalogue has a correspondence with the things catalogued. Hence
common sense leaves us completely in the dark as to the true intrinsic
nature of physical objects, and if there were good reason to regard them
as mental, we could not legitimately reject this opinion merely because
it strikes us as strange. The truth about physical objects \emph{must}
be strange. It may be unattainable, but if any philosopher believes that
he has attained it, the fact that what he offers as the truth is strange
ought not to be made a ground of objection to his opinion.

The grounds on which idealism is advocated are generally grounds derived
from the theory of knowledge, that is to say, from a discussion of the
conditions which things must satisfy in order that we may be able to
know them. The first serious attempt to establish idealism on such
grounds was that of Bishop Berkeley. He proved first, by arguments which
were largely valid, that our sense-data cannot be supposed to have an
existence independent of us, but must be, in part at least,
`in' the mind, in the sense that their
existence would not continue if there were no seeing or hearing or
touching or smelling or tasting. So far, his contention was almost
certainly valid, even if some of his arguments were not so. But he went
on to argue that sense-data were the only things of whose existence our
perceptions could assure us; and that to be known is to be
`in' a mind, and therefore to be
mental. Hence he concluded that nothing can ever be known except what is
in some mind, and that whatever is known without being in my mind must
be in some other mind.

In order to understand his argument, it is necessary to understand his
use of the word `idea'. \label{ideas} He gives the name
`idea' to anything which is
\emph{immediately} known, as, for example, sense-data are known. Thus a
particular colour which we see is an idea; so is a voice which we hear,
and so on. But the term is not wholly confined to sense-data. There will
also be things remembered or imagined, for with such things also we have
immediate acquaintance at the moment of remembering or imagining. All
such immediate data he calls `ideas'.

He then proceeds to consider common objects, such as a tree, for
instance. He shows that all we know immediately when we
`perceive' the tree consists of ideas
in his sense of the word, and he argues that there is not the slightest
ground for supposing that there is anything real about the tree except
what is perceived. Its being, he says, consists in being perceived: in
the Latin of the schoolmen its
'\emph{esse}' is
'\emph{percipi}'. He fully admits that
the tree must continue to exist even when we shut our eyes or when no
human being is near it. But this continued existence, he says, is due to
the fact that God continues to perceive it; the
`real' tree, which corresponds to what
we called the physical object, consists of ideas in the mind of God,
ideas more or less like those we have when we see the tree, but
differing in the fact that they are permanent in God's
mind so long as the tree continues to exist. All our perceptions,
according to him, consist in a partial participation in
God's perceptions, and it is because of this
participation that different people see more or less the same tree. Thus
apart from minds and their ideas there is nothing in the world, nor is
it possible that anything else should ever be known, since whatever is
known is necessarily an idea.

There are in this argument a good many fallacies which have been
important in the history of philosophy, and which it will be as well to
bring to light. In the first place, there is a confusion engendered by
the use of the word `idea'. We think of
an idea as essentially something in somebody's mind, and
thus when we are told that a tree consists entirely of ideas, it is
natural to suppose that, if so, the tree must be entirely in minds. But
the notion of being `in' the mind is
ambiguous. We speak of bearing a person in mind, not meaning that the
person is in our minds, but that a thought of him is in our minds. When
a man says that some business he had to arrange went clean out of his
mind, he does not mean to imply that the business itself was ever in his
mind, but only that a thought of the business was formerly in his mind,
but afterwards ceased to be in his mind. And so when Berkeley says that
the tree must be in our minds if we can know it, all that he really has
a right to say is that a thought of the tree must be in our minds. To
argue that the tree itself must be in our minds is like arguing that a
person whom we bear in mind is himself in our minds. This confusion may
seem too gross to have been really committed by any competent
philosopher, but various attendant circumstances rendered it possible.
In order to see how it was possible, we must go more deeply into the
question as to the nature of ideas.

Before taking up the general question of the nature of ideas, we must
disentangle two entirely separate questions which arise concerning
sense-data and physical objects. We saw that, for various reasons of
detail, Berkeley was right in treating the sense-data which constitute
our perception of the tree as more or less subjective, in the sense that
they depend upon us as much as upon the tree, and would not exist if the
tree were not being perceived. But this is an entirely different point
from the one by which Berkeley seeks to prove that whatever can be
immediately known must be in a mind. For this purpose arguments of
detail as to the dependence of sense-data upon us are useless. It is
necessary to prove, generally, that by being known, things are shown to
be mental. This is what Berkeley believes himself to have done. It is
this question, and not our previous question as to the difference
between sense-data and the physical object, that must now concern us.

Taking the word `idea' in
Berkeley's sense, there are two quite distinct things to
be considered whenever an idea is before the mind. \label{act-object}There is on the one
hand the thing of which we are aware---say the colour of my table---and
on the other hand the actual awareness itself, the mental act of
apprehending the thing. The mental act is undoubtedly mental, but is
there any reason to suppose that the thing apprehended is in any sense
mental? Our previous arguments concerning the colour did not prove it to
be mental; they only proved that its existence depends upon the relation
of our sense organs to the physical object---in our case, the table.
That is to say, they proved that a certain colour will exist, in a
certain light, if a normal eye is placed at a certain point relatively
to the table. They did not prove that the colour is in the mind of the
percipient.

Berkeley's view, that obviously the colour must be in
the mind, seems to depend for its plausibility upon confusing the thing
apprehended with the act of apprehension. Either of these might be
called an `idea'; probably either would
have been called an idea by Berkeley. The act is undoubtedly in the
mind; hence, when we are thinking of the act, we readily assent to the
view that ideas must be in the mind. Then, forgetting that this was only
true when ideas were taken as acts of apprehension, we transfer the
proposition that `ideas are in the
mind' to ideas in the other sense, i.e. to the things
apprehended by our acts of apprehension. \label{apprehension} Thus, by an unconscious
equivocation, we arrive at the conclusion that whatever we can apprehend
must be in our minds. This seems to be the true analysis of
Berkeley's argument, and the ultimate fallacy upon which
it rests.

This question of the distinction between act and object in our
apprehending of things is vitally important, since our whole power of
acquiring knowledge is bound up with it. The faculty of being acquainted
with things other than itself is the main characteristic of a mind.
Acquaintance with objects essentially consists in a relation between the
mind and something other than the mind; it is this that constitutes the
mind's power of knowing things. If we say that the
things known must be in the mind, we are either unduly limiting the
mind's power of knowing, or we are uttering a mere
tautology. We are uttering a mere tautology if we mean by
'\emph{in} the mind' the same as by
'\emph{before} the mind', i.e. if we
mean merely being apprehended by the mind. But if we mean this, we shall
have to admit that what, \emph{in this sense}, is in the mind, may
nevertheless be not mental. Thus when we realize the nature of
knowledge, Berkeley's argument is seen to be wrong in
substance as well as in form, and his grounds for supposing that
`ideas'---i.e. the objects
apprehended---must be mental, are found to have no validity whatever.
Hence his grounds in favour of idealism may be dismissed. It remains to
see whether there are any other grounds.

It is often said, as though it were a self-evident truism, that we
cannot know that anything exists which we do not know. It is inferred
that whatever can in any way be relevant to our experience must be at
least capable of being known by us; whence it follows that if matter
were essentially something with which we could not become acquainted,
matter would be something which we could not know to exist, and which
could have for us no importance whatever. It is generally also implied,
for reasons which remain obscure, that what can have no importance for
us cannot be real, and that therefore matter, if it is not composed of
minds or of mental ideas, is impossible and a mere chimaera.

To go into this argument fully at our present stage would be impossible,
since it raises points requiring a considerable preliminary discussion;
but certain reasons for rejecting the argument may be noticed at once.
To begin at the end: there is no reason why what cannot have any
\emph{practical} importance for us should not be real. It is true that,
if \emph{theoretical} importance is included, everything real is of
\emph{some} importance to us, since, as persons desirous of knowing the
truth about the universe, we have some interest in everything that the
universe contains. But if this sort of interest is included, it is not
the case that matter has no importance for us, provided it exists even
if we cannot know that it exists. We can, obviously, suspect that it may
exist, and wonder whether it does; hence it is connected with our desire
for knowledge, and has the importance of either satisfying or thwarting
this desire.

Again, it is by no means a truism, and is in fact false, that we cannot
know that anything exists which we do not know. The word
`know' is here used in two different
senses. (1) In its first use it is applicable to the sort of knowledge
which is opposed to error, the sense in which what we know is
\emph{true}, the sense which applies to our beliefs and convictions,
i.e. to what are called \emph{judgements}. In this sense of the word we
know \emph{that} something is the case. This sort of knowledge may be
described as knowledge of \emph{truths}. (2) In the second use of the
word `know' above, the word applies to
our knowledge of \emph{things}, which we may call \emph{acquaintance}.
This is the sense in which we know sense-data. (The distinction involved
is roughly that between \emph{savoir} and \emph{connaître} in French, or
between \emph{wissen} and \emph{kennen} in German.)

Thus the statement which seemed like a truism becomes, when re-stated,
the following: `We can never truly judge that something
with which we are not acquainted exists.' This is by no
means a truism, but on the contrary a palpable falsehood. I have not the
honour to be acquainted with the Emperor of \label{China1}China, but I truly judge
that he exists. It may be said, of course, that I judge this because of
other people's acquaintance with him. This, however,
would be an irrelevant retort, since, if the principle were true, I
could not know that any one else is acquainted with him. But further:
there is no reason why I should not know of the existence of something
with which nobody is acquainted. This point is important, and demands
elucidation.

If I am acquainted with a thing which exists, my acquaintance gives me
the knowledge that it exists. But it is not true that, conversely,
whenever I can know that a thing of a certain sort exists, I or some one
else must be acquainted with the thing. What happens, in cases where I
have true judgement without acquaintance, is that the thing is known to
me by \emph{description}, and that, in virtue of some general principle,
the existence of a thing answering to this description can be inferred
from the existence of something with which I am acquainted. In order to
understand this point fully, it will be well first to deal with the
difference between knowledge by acquaintance and knowledge by
description, and then to consider what knowledge of general principles,
if any, has the same kind of certainty as our knowledge of the existence
of our own experiences. These subjects will be dealt with in the
following chapters.

\protect\hypertarget{link2HCH0005}{}{}
\pagebreak
\section{Notes for the Student}
\markboth{CHAPTER 4 NOTES}{CHAPTER 4 NOTES}
As we open Ch 4, let us recall what we agreed to so far:
\begin{itemize}%[itemsep=0ex]
	\item We must distinguish appearances (sense-data) from reality (physical objects).
	\item It is simpler to assume that physical objects exist and cause sense-data.
	\item The relational structure of the physical world is what we know using our appearances.
\end{itemize}
But what about the nature of matter in itself? Is it possible that physical objects are \textit{mental}? \textit{Idealists} hold that physical objects are mental. An \textit{Idealist} in Russell's sense holds that whatever exists (or can be known to exist) is mental (\pageref{idealism}).

Now Idealists \textit{do not deny that matter exists}. Matter is just whatever quasi-permanent stuff causes our appearances. That stuff could be mental: matter is consistent with Idealism. %Many philosophers, from Buddha, Parmenides, and Plato, to the scientific community of the present day, have held a view on which reality is different from how we instinctively view it. Each is a philosophical views, deserving to be considered on its merits (Ch 4, 38).

Turning now to Berkeley's \textit{Treatise Concerning the Principles of Human Knowledge} (\S I.3)
\begin{quote}
	That neither our thoughts, nor passions, nor ideas formed by the imagination, exist without the mind, is what everybody will allow...The table I write on I say exists, that is, I see and feel it; and if I were out of my study I should say it existed---meaning thereby that if I was in my study I might perceive it, or that some other spirit actually does perceive it...That is all that I can understand by these and the like expressions. For as to what is said by the absolute existence of unthinking things without any relation to their being perceived, that seems perfectly unintelligible. Their \textit{esse} is \textit{percipi}, nor is it possible that they should have any existence out of the minds of thinking things which perceive them.
\end{quote}
Berkeley argues that we only know of stuff by perceiving it (a natural enough suggestion). As such, we only have knowledge of ideas, so that even physical objects are ideas (in God's mind, since they are not directly perceived by us when, say, we are not looking at them):\footnote{`HS' is the valid argument-form \textit{hypothetical syllogism}: (Premise 1) if $p$, then $q$; (Premise 2) if $q$, then $r$; (Conclusion) thus, if $p$, then $r$.}
\begin{enumerate}%[Ch 4: Berkeley's argument for everything being mental.]
	\item Sense-data are \textit{ideas} [immediately known (\pageref{ideas})]. \hfill Premise
	\item If sense-data are ideas, then sense-data are in a mind. \hfill Premise
	\item Thus, sense-data are in a mind. \hfill 1 2 MP
	\item If something is immediately known to exist, then it is perceived. \hspace{.25cm} Premise
	\item If something is perceived, then it is sense-data [clearly!]. \hfill Premise
	\item So, if something is immediately known to exist, then it is sense-data. \\ \text{ } \hfill 4, 5 HS
	%\item So, whatever is known to exist is an idea.}{By 4, 5}
	\item So, if something is immediately known to exist, then it is in a mind. \\ \text{ } \hfill  3, 6 HS
	\item Matter is immediately known to exist [by Ch 2!]. \hfill Premise
	%\item So, matter is perceived.}{4, 7 MP}
	%\item So, matter is sense-data.}{5, 8 MP}
	%\item So, matter is sense-data and sense-data are in the mind.}{By 3, 9}
	%\item What is known to exist is what known by perception. \hfill Premise}
	%\item What is known by perception is sense-data. \hfill Premise}
	%\item So, what is known to exist is sense-data. \hfill Premise}
	%\item So, what is known to exist is in the mind.}{3, 6}
	\item Thus, matter is in a mind. \hfill 7, 8 MP
	%\item Whatever is perceived is an idea [immediately known]. \hfill Premise}
	%\item So, if matter is perceived, then matter is an idea.}{From 8}
	%\item If our perceptions only give us knowledge of our sense-data, then our perceptions only give us knowledge of ideas. \hfill Premise}
	%\item Thus, our perception only gives us knowledge of ideas.}{4, 5 MP}
	%\item If matter is known to exist, then matter is perceived by some mind.}{3, 4 MP}
	%\item All things that are immediately known are in the mind.}{}
\end{enumerate}
Russell agrees with much of this argument (and we should, too!). He agrees with Berkeley that sense-data are largely dependent on us, so that sense-data exist `in' our minds in some sense (that is, they are mind-dependent entities). So he accepts (1) and, in some sense, (2). He definitely accepts (5). But Russell rejects the conclusion. As we said, one has to contest the premises of a valid argument if one wants to rationally reject the conclusion. 

So what is Russell's response to this? Russell wants to distinguish two senses of `know' and also two senses of `idea'. First, the word `idea' may pick out a mental \textit{act} of apprehension or the \textit{object} apprehended (\pageref{act-object}). Russell maintains that though all apprehension is mental, we do apprehend non-mental objects (like universals, as we will see in Ch IX).

Once we distinguish these two senses of the word `idea', we can agree at once that whatever is immediately known to exist is apprehended by some mental act; but the object apprehended is \textit{not thereby shown} to be mental (\ref{apprehension}). An independent argument for this is needed. For example, sense-data do seem mind-dependent because how they are depends on our minds and bodies. This does not generalize to all objects of awareness.

Second, in Ch 5 Russell introduces a distinction between knowledge by direct awareness, or \textit{knowledge by acquaintance}, and indirect knowledge, or \textit{knowledge by description}. 

We can know objects in these two distinct ways. We know something by acquaintance when we are directly aware of, or acquainted with, a thing, as when we perceive a sense-datum, remember a memory-datum, or introspect on a mental-datum. 

We know something by description when we know some truth about it. If you have never been to, or flown over, Mongolia, then you can only know about Mongolia by description, and cannot know it by acquaintance. I have been there, so I know Mongolia by acquaintance.

Now you can have knowledge of something by description and knowledge by acquaintance. You have acquaintance with the United States, and with many parts of it. But you do not know \textit{every} part of the United States by acquaintance (it is much too big!).

We will discuss these two kinds of knowledge in the next chapter. But let us return to Berkeley. Does Russell have the last word, or might Berkeley reply to Russell?

Berkeley actually anticipates the sort of thing that Russell says. Russell's reply is essentially that, while we do not have \textit{direct} (acquaintance) knowledge of matter, we do have \textit{indirect} (descrptive) knowledge of matter. But Berkeley criticizes the very idea of indirect knowledge, or knowledge that is separate from our ideas. Berkeley writes (\textit{Treatise}, \S I.8):
\begin{quote}
	But say you, though the ideas themselves do not exist without the mind, yet there may be things like them, whereof they are copies or resemblances...I answer, an idea can be like nothing but an idea; a colour or figure can be like nothing but another colour or figure. Again, I ask whether those supposed originals or external things, of which our ideas are the pictures or representations, be themselves perceivable or no? If they are, then they are ideas and we have gained our point; but if you say they are not, I appeal to any one whether it be sense to assert a colour is like something which is invisible; hard or soft, like something which is intangible; and so of the rest. [see the \textit{master argument}, \S I.23]
\end{quote}
This is a devastating argument! Berkeley is saying that it violates the view that knowledge comes from perception to say that there are unperceived physical objects `similar' to percepts. %And he further attacks the argument with 

\hypertarget{chapter-v.-knowledge-by-acquaintance-and-knowledge-by-description}{%
\chapter{Knowledge by Acquaintance and Knowledge by Description}\label{chapter-v.-knowledge-by-acquaintance-and-knowledge-by-description}}

\chaptermark{Knowledge by Acquaintance and ...}

In the preceding chapter we saw that there are two sorts of knowledge:
knowledge of things, and knowledge of truths. In this chapter we shall
be concerned exclusively with knowledge of things, of which in turn we
shall have to distinguish two kinds. Knowledge of things, when it is of
the kind we call knowledge by \emph{acquaintance}, is essentially
simpler than any knowledge of truths, and logically independent of
knowledge of truths, though it would be rash to assume that human beings
ever, in fact, have acquaintance with things without at the same time
knowing some truth about them. Knowledge of things by
\emph{description}, on the contrary, always involves, as we shall find
in the course of the present chapter, some knowledge of truths as its
source and ground. But first of all we must make clear what we mean by
`acquaintance' and what we mean by
`description'.

We shall say that we have \emph{acquaintance} with anything of which we
are directly aware, without the intermediary of any process of inference
or any knowledge of truths. Thus in the presence of my table I am
acquainted with the sense-data that make up the appearance of my
table---its colour, shape, hardness, smoothness, etc.; all these are
things of which I am immediately conscious when I am seeing and touching
my table. The particular shade of colour that I am seeing may have many
things said about it---I may say that it is brown, that it is rather
dark, and so on. But such statements, though they make me know truths
about the colour, do not make me know the colour itself any better than
I did before so far as concerns knowledge of the colour itself, as
opposed to knowledge of truths about it, I know the colour perfectly and
completely when I see it, and no further knowledge of it itself is even
theoretically possible. Thus the sense-data which make up the appearance
of my table are things with which I have acquaintance, things
immediately known to me just as they are.

My knowledge of the table as a physical object, on the contrary, is not
direct knowledge. Such as it is, it is obtained through acquaintance
with the sense-data that make up the appearance of the table. We have
seen that it is possible, without absurdity, to doubt whether there is a
table at all, whereas it is not possible to doubt the sense-data. My
knowledge of the table is of the kind which we shall call
`knowledge by description'. The table is
`the physical object which causes such-and-such
sense-data'. This describes the table by means of the
sense-data. In order to know anything at all about the table, we must
know truths connecting it with things with which we have acquaintance:
we must know that `such-and-such sense-data are caused
by a physical object'. There is no state of mind in which
we are directly aware of the table; all our knowledge of the table is
really knowledge of truths, and the actual thing which is the table is
not, strictly speaking, known to us at all. We know a description, and
we know that there is just one object to which this description applies,
though the object itself is not directly known to us. In such a case, we
say that our knowledge of the object is knowledge by description.

\label{foundationalism} All our knowledge, both knowledge of things and knowledge of truths,
rests upon acquaintance as its foundation. It is therefore important to
consider what kinds of things there are with which we have acquaintance.

Sense-data, as we have already seen, are among the things with which we
are acquainted; in fact, they supply the most obvious and striking
example of knowledge by acquaintance. But if they were the sole example,
our knowledge would be very much more restricted than it is. We should
only know what is now present to our senses: we could not know anything
about the past---not even that there was a past---nor could we know any
truths about our sense-data, for all knowledge of truths, as we shall
show, demands acquaintance with things which are of an essentially
different character from sense-data, the things which are sometimes
called `abstract ideas', but which we
shall call `universals'. We have
therefore to consider acquaintance with other things besides sense-data
if we are to obtain any tolerably adequate analysis of our knowledge.

The first extension beyond sense-data to be considered is acquaintance
by \emph{memory}. It is obvious that we often remember what we have seen
or heard or had otherwise present to our senses, and that in such cases
we are still immediately aware of what we remember, in spite of the fact
that it appears as past and not as present. This immediate knowledge by
memory is the source of all our knowledge concerning the past: without
it, there could be no knowledge of the past by inference, since we
should never know that there was anything past to be inferred.

The next extension to be considered is acquaintance by
\emph{introspection}. We are not only aware of things, but we are often
aware of being aware of them. When I see the sun, I am often aware of my
seeing the sun; thus `my seeing the
sun' is an object with which I have acquaintance. When
I desire food, I may be aware of my desire for food; thus
`my desiring food' is an object with
which I am acquainted. Similarly we may be aware of our feeling pleasure
or pain, and generally of the events which happen in our minds. This
kind of acquaintance, which may be called self-consciousness, is the
source of all our knowledge of mental things. It is obvious that it is
only what goes on in our own minds that can be thus known immediately.
What goes on in the minds of others is known to us through our
perception of their bodies, that is, through the sense-data in us which
are associated with their bodies. But for our acquaintance with the
contents of our own minds, we should be unable to imagine the minds of
others, and therefore we could never arrive at the knowledge that they
have minds. It seems natural to suppose that self-consciousness is one
of the things that distinguish men from animals: animals, we may
suppose, though they have acquaintance with sense-data, never become
aware of this acquaintance, and thus never know of their own existence. 
I do not mean that they \emph{doubt} whether
they exist, but that they have never become conscious of the fact that
they have sensations and feelings, nor therefore of the fact that they,
the subjects of their sensations and feelings, exist.

We have spoken of acquaintance with the contents of our minds as
\emph{self}-consciousness, but it is not, of course, consciousness of
our \emph{self}: it is consciousness of particular thoughts and
feelings. The question whether we are also acquainted with our bare
selves, as opposed to particular thoughts and feelings, is a very
difficult one, upon which it would be rash to speak positively. When we
try to look into ourselves we always seem to come upon some particular
thought or feeling, and not upon the
`I' which has the thought or feeling.
Nevertheless there are some reasons for thinking that we are acquainted
with the `I', though the acquaintance is
hard to disentangle from other things. To make clear what sort of reason
there is, let us consider for a moment what our acquaintance with
particular thoughts really involves.

When I am acquainted with `my seeing the
sun', it seems plain that I am acquainted with two
different things in relation to each other. On the one hand there is the
sense-datum which represents the sun to me, on the other hand there is
that which sees this sense-datum. All acquaintance, such as my
acquaintance with the sense-datum which represents the sun, seems
obviously a relation between the person acquainted and the object with
which the person is acquainted. When a case of acquaintance is one with
which I can be acquainted (as I am acquainted with my acquaintance with
the sense-datum representing the sun), it is plain that the person
acquainted is myself. Thus, when I am acquainted with my seeing the sun,
the whole fact with which I am acquainted is
`Self-acquainted-with-sense-datum'.

Further, we know the truth `I am acquainted with this
sense-datum'. It is hard to see how we could know this
truth, or even understand what is meant by it, unless we were acquainted
with something which we call `I'. It does
not seem necessary to suppose that we are acquainted with a more or less
permanent person, the same to-day as yesterday, but it does seem as
though we must be acquainted with that thing, whatever its nature, which
sees the sun and has acquaintance with sense-data. Thus, in some sense
it would seem we must be acquainted with our Selves as opposed to our
particular experiences. But the question is difficult, and complicated
arguments can be adduced on either side. Hence, although acquaintance
with ourselves seems \emph{probably} to occur, it is not wise to assert
that it undoubtedly does occur.

\label{acquaintance} We may therefore sum up as follows what has been said concerning
acquaintance with things that exist. We have acquaintance in sensation
with the data of the outer senses, and in introspection with the data of
what may be called the inner sense---thoughts, feelings, desires, etc.;
we have acquaintance in memory with things which have been data either
of the outer senses or of the inner sense. Further, it is probable,
though not certain, that we have acquaintance with Self, as that which
is aware of things or has desires towards things.

In addition to our acquaintance with particular existing things, we also
have acquaintance with what we shall call \emph{universals}, that is to
say, general ideas, such as \emph{whiteness}, \emph{diversity},
\emph{brotherhood}, and so on. Every complete sentence must contain at
least one word which stands for a universal, since all verbs have a
meaning which is universal. We shall return to universals later on, in
Chapter IX; for the present, it is only necessary to guard against the
supposition that whatever we can be acquainted with must be something
particular and existent. Awareness of universals is called
\emph{conceiving}, and a universal of which we are aware is called a
\emph{concept}.

It will be seen that among the objects with which we are acquainted are
not included physical objects (as opposed to sense-data), nor other
people's minds. These things are known to us by what I
call `knowledge by description', which we
must now consider.

By a `description' I mean any phrase of
the form `a so-and-so' or
`the so-and-so'. A phrase of the form
`a so-and-so' I shall call an
`ambiguous' description; a phrase of
the form `the so-and-so' (in the
singular) I shall call a `definite'
description. Thus `a man' is an
ambiguous description, and `the man with the iron
mask' is a definite description. There are various
problems connected with ambiguous descriptions, but I pass them by,
since they do not directly concern the matter we are discussing, which
is the nature of our knowledge concerning objects in cases where we know
that there is an object answering to a definite description, though we
are not acquainted with any such object. This is a matter which is
concerned exclusively with definite descriptions. I shall therefore, in
the sequel, speak simply of
`descriptions' when I mean
`definite descriptions'. Thus a
description will mean any phrase of the form `the
so-and-so' in the singular.

We shall say that an object is `known by
description' when we know that it is
`\emph{the} so-and-so', i.e. when we know that
there is one object, and no more, having a certain property; and it will
generally be implied that we do not have knowledge of the same object by
acquaintance. We know that the man with the iron mask existed, and many
propositions are known about him; but we do not know who he was. We know
that the candidate who gets the most votes will be elected, and in this
case we are very likely also acquainted (in the only sense in which one
can be acquainted with some one else) with the man who is, in fact, the
candidate who will get most votes; but we do not know which of the
candidates he is, i.e. we do not know any proposition of the form
`A is the candidate who will get most
votes' where A is one of the candidates by name. We
shall say that we have `merely descriptive
knowledge' of the so-and-so when, although we know that
the so-and-so exists, and although we may possibly be acquainted with
the object which is, in fact, the so-and-so, yet we do not know any
proposition '\emph{a} is the so-and-so',
where \emph{a} is something with which we are acquainted.

When we say `the so-and-so exists', we
mean that there is just one object which is the so-and-so. The
proposition '\emph{a} is the
so-and-so' means that \emph{a} has the property
so-and-so, and nothing else has. `Mr. A. is the Unionist
candidate for this constituency' means
`Mr. A. is a Unionist candidate for this constituency,
and no one else is'. `The Unionist
candidate for this constituency exists' means
`some one is a Unionist candidate for this constituency,
and no one else is'. Thus, when we are acquainted with an
object which is the so-and-so, we know that the so-and-so exists; but we
may know that the so-and-so exists when we are not acquainted with any
object which we know to be the so-and-so, and even when we are not
acquainted with any object which, in fact, is the so-and-so.

Common words, even proper names, are usually really descriptions. That
is to say, the thought in the mind of a person using a proper name
correctly can generally only be expressed explicitly if we replace the
proper name by a description. Moreover, the description required to
express the thought will vary for different people, or for the same
person at different times. The only thing constant (so long as the name
is rightly used) is the object to which the name applies. But so long as
this remains constant, the particular description involved usually makes
no difference to the truth or falsehood of the proposition in which the
name appears.

Let us take some illustrations. Suppose some statement made about
Bismarck. Assuming that there is such a thing as direct acquaintance
with oneself, Bismarck himself might have used his name directly to
designate the particular person with whom he was acquainted. In this
case, if he made a judgement about himself, he himself might be a
constituent of the judgement. Here the proper name has the direct use
which it always wishes to have, as simply standing for a certain object,
and not for a description of the object. But if a person who knew
Bismarck made a judgement about him, the case is different. What this
person was acquainted with were certain sense-data which he connected
(rightly, we will suppose) with Bismarck's body. His
body, as a physical object, and still more his mind, were only known as
the body and the mind connected with these sense-data. That is, they
were known by description. It is, of course, very much a matter af
chance which characteristics of a man's appearance will
come into a friend's mind when he thinks of him; thus
the description actually in the friend's mind is
accidental. The essential point is that he knows that the various
descriptions all apply to the same entity, in spite of not being
acquainted with the entity in question.

When we, who did not know Bismarck, make a judgement about him, the
description in our minds will probably be some more or less vague mass
of historical knowledge---far more, in most cases, than is required to
identify him. But, for the sake of illustration, let us assume that we
think of him as `the first Chancellor of the German
Empire'. Here all the words are abstract except
`German'. The word
`German' will, again, have different
meanings for different people. To some it will recall travels in
Germany, to some the look of Germany on the map, and so on. But if we
are to obtain a description which we know to be applicable, we shall be
compelled, at some point, to bring in a reference to a particular with
which we are acquainted. Such reference is involved in any mention of
past, present, and future (as opposed to definite dates), or of here and
there, or of what others have told us. Thus it would seem that, in some
way or other, a description known to be applicable to a particular must
involve some reference to a particular with which we are acquainted, if
our knowledge about the thing described is not to be merely what follows
\emph{logically} from the description. For example, `the
most long-lived of men' is a description involving only
universals, which must apply to some man, but we can make no judgements
concerning this man which involve knowledge about him beyond what the
description gives. If, however, we say, `The first
Chancellor of the German Empire was an astute
diplomatist', we can only be assured of the truth of our
judgement in virtue of something with which we are acquainted---usually
a testimony heard or read. Apart from the information we convey to
others, apart from the fact about the actual Bismarck, which gives
importance to our judgement, the thought we really have contains the one
or more particulars involved, and otherwise consists wholly of concepts.

All names of places---London, England, Europe, the Earth, the Solar
System---similarly involve, when used, descriptions which start from
some one or more particulars with which we are acquainted. I suspect
that even the Universe, as considered by metaphysics, involves such a
connexion with particulars. In logic, on the contrary, where we are
concerned not merely with what does exist, but with whatever might or
could exist or be, no reference to actual particulars is involved.

It would seem that, when we make a statement about something only known
by description, we often \emph{intend} to make our statement, not in the
form involving the description, but about the actual thing described.
That is to say, when we say anything about Bismarck, we should like, if
we could, to make the judgement which Bismarck alone can make, namely,
the judgement of which he himself is a constituent. In this we are
necessarily defeated, since the actual Bismarck is unknown to us. But we
know that there is an object B, called Bismarck, and that B was an
astute diplomatist. We can thus \emph{describe} the proposition we
should like to affirm, namely, `B was an astute
diplomatist', where B is the object which was Bismarck.
If we are describing Bismarck as `the first Chancellor
of the German Empire', the proposition we should like to
affirm may be described as `the proposition asserting,
concerning the actual object which was the first Chancellor of the
German Empire, that this object was an astute
diplomatist'. What enables us to communicate in spite of
the varying descriptions we employ is that we know there is a true
proposition concerning the actual Bismarck, and that however we may vary
the description (so long as the description is correct) the proposition
described is still the same. This proposition, which is described and is
known to be true, is what interests us; but we are not acquainted with
the proposition itself, and do not know it, though we know it is true.

It will be seen that there are various stages in the removal from
acquaintance with particulars: there is Bismarck to people who knew him;
Bismarck to those who only know of him through history; the man with the
iron mask; the longest-lived of men. These are progressively further
removed from acquaintance with particulars; the first comes as near to
acquaintance as is possible in regard to another person; in the second,
we shall still be said to know `who Bismarck
was'; in the third, we do not know who was the man with
the iron mask, though we can know many propositions about him which are
not logically deducible from the fact that he wore an iron mask; in the
fourth, finally, we know nothing beyond what is logically deducible from
the definition of the man. There is a similar hierarchy in the region of
universals. Many universals, like many particulars, are only known to us
by description. But here, as in the case of particulars, knowledge
concerning what is known by description is ultimately reducible to
knowledge concerning what is known by acquaintance.

\label{fundamental} The fundamental principle in the analysis of propositions containing
descriptions is this: \emph{Every proposition which we can understand
must be composed wholly of constituents with which we are acquainted}.

We shall not at this stage attempt to answer all the objections which
may be urged against this fundamental principle. For the present, we
shall merely point out that, in some way or other, it must be possible
to meet these objections, for it is scarcely conceivable that we can
make a judgement or entertain a supposition without knowing what it is
that we are judging or supposing about. We must attach \emph{some}
meaning to the words we use, if we are to speak significantly and not
utter mere noise; and the meaning we attach to our words must be
something with which we are acquainted. Thus when, for example, we make
a statement about Julius Caesar, it is plain that Julius Caesar himself
is not before our minds, since we are not acquainted with him. We have
in mind some description of Julius Caesar: `the man who
was assassinated on the Ides of March',
`the founder of the Roman Empire', or,
perhaps, merely `the man whose name was \emph{Julius
Caesar}'. (In this last description, \emph{Julius Caesar}
is a noise or shape with which we are acquainted.) Thus our statement
does not mean quite what it seems to mean, but means something
involving, instead of Julius Caesar, some description of him which is
composed wholly of particulars and universals with which we are
acquainted.

The chief importance of knowledge by description is that it enables us
to pass beyond the limits of our private experience. In spite of the
fact that we can only know truths which are wholly composed of terms
which we have experienced in acquaintance, we can yet have knowledge by
description of things which we have never experienced. In view of the
very narrow range of our immediate experience, this result is vital, and
until it is understood, much of our knowledge must remain mysterious and
therefore doubtful.

\protect\hypertarget{link2HCH0006}{}{}
\pagebreak 
\section{Notes for the Student}
\markboth{CHAPTER 5 NOTES}{CHAPTER 5 NOTES}
In Ch 5, we will consider the sorts of things we know, and decide whether how we know them falls into our \textit{knowledge by acquaintance} or \textit{knowledge by description} (or both!). 
Here are the various kinds of knowledge, on Russell's account, in a tree-format:
\begin{center}
	\begin{forest}
		[Knowledge
		[of Things
		[by Acquaintance]
		[by Description]
		]
		[of Truths
		[by Intuition]
		[by Inference]
		]
		]
	\end{forest}
\end{center}
Let me also give you a table to clarify his views on immediate and derivative knowledge:
\begin{center}
	\begin{tabular}{|c|c|c|} \hline
		\textbf{Kinds of Knowledge}& \textbf{Immediate} & \textbf{Derivative} \\ \hline
		\textbf{of Things} & Acquaintance & Description \\ \hline
		\textbf{of Truths} & Intuition & Inference \\ \hline
	\end{tabular}
\end{center}
By \textit{immediate knowledge}, Russell just means that we are either aware of, or \textit{acquainted with}, the thing that is known, or that we know a truth without inferring it from other truths (\ref{immediate}). Russell suggests that such immediate knowledge of truths is due to \textit{intuitions} by which they are seen to be self-evident. Now self-evident here does not mean \textit{certain} or \textit{impossible to refute}; they just enjoy our \textit{confidence}, which comes in degrees (\ref{degrees}).

Here are the sorts of things that we are and are not acquainted with (\pageref{acquaintance}):
\begin{center}
	\begin{tabular}{|c|c|} \hline
		\textbf{Knowable by Acquaintance}& \textbf{Not Knowable by Acquaintance} \\ \hline
		Sense-data & Physical Objects \\ \hline
		Memory-data & Other folks' minds \\ \hline
		Introspective-data &  \\ \hline
		Universals & \\ \hline
		\textbf{the Self?} & \\ \hline
	\end{tabular}
\end{center}
This issue of whether we have acquaintance with the Self is great for a term paper! But I want to highlight one other point from this chapter.
\subsection*{Foundationalism}
Russell endorses \textit{foundationalism}. He writes:
\begin{quote}
	All our knowledge, both knowledge of things and knowledge of truths, rests upon acquaintance as its foundation. (\pageref{foundationalism})
\end{quote} 
Russell is saying that all justification is traced back to immediate knowledge. (The way he endorses foundationalism comes close to what Berkeley says in arguing for Idealism.)

This is related to Russell's case for his ``fundamental principle" in analysis that traces all analysis to entities with which we have acquaintance (\pageref{fundamental}).

\hypertarget{chapter-vi.-on-induction}{%
\chapter{ON INDUCTION}\label{chapter-vi.-on-induction}}

In almost all our previous discussions we have been concerned in the
attempt to get clear as to our data in the way of knowledge of
existence. What things are there in the universe whose existence is
known to us owing to our being acquainted with them? So far, our answer
has been that we are acquainted with our sense-data, and, probably, with
ourselves. These we know to exist. And past sense-data which are
remembered are known to have existed in the past. This knowledge
supplies our data.

But if we are to be able to draw inferences from these data---if we are
to know of the existence of matter, of other people, of the past before
our individual memory begins, or of the future, we must know general
principles of some kind by means of which such inferences can be drawn.
It must be known to us that the existence of some one sort of thing, A,
is a sign of the existence of some other sort of thing, B, either at the
same time as A or at some earlier or later time, as, for example,
thunder is a sign of the earlier existence of lightning. If this were
not known to us, we could never extend our knowledge beyond the sphere
of our private experience; and this sphere, as we have seen, is
exceedingly limited. The question we have now to consider is whether
such an extension is possible, and if so, how it is effected.

Let us take as an illustration a matter about which none of us, in fact,
feel the slightest doubt. \label{sun} We are all convinced that the sun will rise
to-morrow. Why? Is this belief a mere blind outcome of past experience,
or can it be justified as a reasonable belief? It is not easy to find a
test by which to judge whether a belief of this kind is reasonable or
not, but we can at least ascertain what sort of general beliefs would
suffice, if true, to justify the judgement that the sun will rise
to-morrow, and the many other similar judgements upon which our actions
are based.

It is obvious that if we are asked why we believe that the sun will rise
to-morrow, we shall naturally answer `Because it always
has risen every day'. We have a firm belief that it will
rise in the future, because it has risen in the past. If we are
challenged as to why we believe that it will continue to rise as
heretofore, we may appeal to the laws of motion: the earth, we shall
say, is a freely rotating body, and such bodies do not cease to rotate
unless something interferes from outside, and there is nothing outside
to interfere with the earth between now and to-morrow. Of course it
might be doubted whether we are quite certain that there is nothing
outside to interfere, but this is not the interesting doubt. The
interesting doubt is as to whether the laws of motion will remain in
operation until to-morrow. If this doubt is raised, we find ourselves in
the same position as when the doubt about the sunrise was first raised.

The \emph{only} reason for believing that the laws of motion will remain
in operation is that they have operated hitherto, so far as our
knowledge of the past enables us to judge. It is true that we have a
greater body of evidence from the past in favour of the laws of motion
than we have in favour of the sunrise, because the sunrise is merely a
particular case of fulfilment of the laws of motion, and there are
countless other particular cases. But the real question is: Do
\emph{any} number of cases of a law being fulfilled in the past afford
evidence that it will be fulfilled in the future? If not, it becomes
plain that we have no ground whatever for expecting the sun to rise
to-morrow, or for expecting the bread we shall eat at our next meal not
to poison us, or for any of the other scarcely conscious expectations
that control our daily lives. It is to be observed that all such
expectations are only \emph{probable}; thus we have not to seek for a
proof that they \emph{must} be fulfilled, but only for some reason in
favour of the view that they are \emph{likely} to be fulfilled.

Now in dealing with this question we must, to begin with, make an
important distinction, without which we should soon become involved in
hopeless confusions. Experience has shown us that, hitherto, the
frequent repetition of some uniform succession or coexistence has been a
\emph{cause} of our expecting the same succession or coexistence on the
next occasion. Food that has a certain appearance generally has a
certain taste, and it is a severe shock to our expectations when the
familiar appearance is found to be associated with an unusual taste.
Things which we see become associated, by habit, with certain tactile
sensations which we expect if we touch them; one of the horrors of a
ghost (in many ghost-stories) is that it fails to give us any sensations
of touch. Uneducated people who go abroad for the first time are so
surprised as to be incredulous when they find their native language not
understood.

And this kind of association is not confined to men; in animals also it
is very strong. A horse which has been often driven along a certain road
resists the attempt to drive him in a different direction. Domestic
animals expect food when they see the person who usually feeds them. We
know that all these rather crude expectations of uniformity are liable
to be misleading. The man who has fed the chicken every day throughout
its life at last wrings its neck instead, showing that more refined
views as to the uniformity of nature would have been useful to the
chicken.

But in spite of the misleadingness of such expectations, they
nevertheless exist. The mere fact that something has happened a certain
number of times causes animals and men to expect that it will happen
again. Thus our instincts certainly cause us to believe that the sun
will rise to-morrow, but we may be in no better a position than the
chicken which unexpectedly has its neck wrung. We have therefore to
distinguish the fact that past uniformities \emph{cause} expectations as
to the future, from the question whether there is any reasonable ground
for giving weight to such expectations after the question of their
validity has been raised.

The problem we have to discuss is whether there is any reason for
believing in what is called `the uniformity of
nature'. The belief in the uniformity of nature is the
belief that everything that has happened or will happen is an instance
of some general law to which there are no exceptions. The crude
expectations which we have been considering are all subject to
exceptions, and therefore liable to disappoint those who entertain them.
But science habitually assumes, at least as a working hypothesis, that
general rules which have exceptions can be replaced by general rules
which have no exceptions. `Unsupported bodies in air
fall' is a general rule to which balloons and
aeroplanes are exceptions. But the laws of motion and the law of
gravitation, which account for the fact that most bodies fall, also
account for the fact that balloons and aeroplanes can rise; thus the
laws of motion and the law of gravitation are not subject to these
exceptions.

The belief that the sun will rise to-morrow might be falsified if the
earth came suddenly into contact with a large body which destroyed its
rotation; but the laws of motion and the law of gravitation would not be
infringed by such an event. The business of science is to find
uniformities, such as the laws of motion and the law of gravitation, to
which, so far as our experience extends, there are no exceptions. In
this search science has been remarkably successful, and it may be
conceded that such uniformities have held hitherto. This brings us back
to the question: Have we any reason, assuming that they have always held
in the past, to suppose that they will hold in the future?

It has been argued that we have reason to know that the future will
resemble the past, because what was the future has constantly become the
past, and has always been found to resemble the past, so that we really
have experience of the future, namely of times which were formerly
future, which we may call past futures. But such an argument really begs
the very question at issue. We have experience of past futures, but not
of future futures, and the question is: Will future futures resemble
past futures? This question is not to be answered by an argument which
starts from past futures alone. We have therefore still to seek for some
principle which shall enable us to know that the future will follow the
same laws as the past.

The reference to the future in this question is not essential. The same
question arises when we apply the laws that work in our experience to
past things of which we have no experience---as, for example, in
geology, or in theories as to the origin of the Solar System. The
question we really have to ask is: `When two things have
been found to be often associated, and no instance is known of the one
occurring without the other, does the occurrence of one of the two, in a
fresh instance, give any good ground for expecting the
other?' On our answer to this question must depend the
validity of the whole of our expectations as to the future, the whole of
the results obtained by induction, and in fact practically all the
beliefs upon which our daily life is based.

It must be conceded, to begin with, that the fact that two things have
been found often together and never apart does not, by itself, suffice
to \emph{prove} demonstratively that they will be found together in the
next case we examine. The most we can hope is that the oftener things
are found together, the more probable it becomes that they will be found
together another time, and that, if they have been found together often
enough, the probability will amount \emph{almost} to certainty. It can
never quite reach certainty, because we know that in spite of frequent
repetitions there sometimes is a failure at the last, as in the case of
the chicken whose neck is wrung. Thus probability is all we ought to
seek.

It might be urged, as against the view we are advocating, that we know
all natural phenomena to be subject to the reign of law, and that
sometimes, on the basis of observation, we can see that only one law can
possibly fit the facts of the case. Now to this view there are two
answers. The first is that, even if \emph{some} law which has no
exceptions applies to our case, we can never, in practice, be sure that
we have discovered that law and not one to which there are exceptions.
The second is that the reign of law would seem to be itself only
probable, and that our belief that it will hold in the future, or in
unexamined cases in the past, is itself based upon the very principle we
are examining.

The principle we are examining may be called the \emph{principle of
induction}, and its two parts may be stated as follows:

\begin{enumerate}[label={(\emph{\alph*})}]
	\item When a thing of a certain sort A has been found to be associated
	with a thing of a certain other sort B, and has never been found
	dissociated from a thing of the sort B, the greater the number of cases
	in which A and B have been associated, the greater is the probability
	that they will be associated in a fresh case in which one of them is
	known to be present;
	\item Under the same circumstances, a sufficient number of cases of
	association will make the probability of a fresh association nearly a
	certainty, and will make it approach certainty without limit.
\end{enumerate}
As just stated, the principle applies only to the verification of our
expectation in a single fresh instance. But we want also to know that
there is a probability in favour of the general law that things of the
sort A are \emph{always} associated with things of the sort B, provided
a sufficient number of cases of association are known, and no cases of
failure of association are known. The probability of the general law is
obviously less than the probability of the particular case, since if the
general law is true, the particular case must also be true, whereas the
particular case may be true without the general law being true.
Nevertheless the probability of the general law is increased by
repetitions, just as the probability of the particular case is. We may
therefore repeat the two parts of our principle as regards the general
law, thus:

\begin{enumerate}[label={(\emph{\alph*})}]
	\item The greater the number of cases in which a thing of the sort A has
	been found associated with a thing of the sort B, the more probable it
	is (if no cases of failure of association are known) that A is always
	associated with B;
	\item Under the same circumstances, a sufficient number of cases of the
	association of A with B will make it nearly certain that A is always
	associated with B, and will make this general law approach certainty
	without limit.
\end{enumerate}
It should be noted that probability is always relative to certain data.
In our case, the data are merely the known cases of coexistence of A and
B. There may be other data, which \emph{might} be taken into account,
which would gravely alter the probability. For example, a man who had
seen a great many white swans might argue, by our principle, that on the
data it was \emph{probable} that all swans were white, and this might be
a perfectly sound argument. The argument is not disproved by the fact
that some swans are black, because a thing may very well happen in spite
of the fact that some data render it improbable. In the case of the
swans, a man might know that colour is a very variable characteristic in
many species of animals, and that, therefore, an induction as to colour
is peculiarly liable to error. But this knowledge would be a fresh
datum, by no means proving that the probability relatively to our
previous data had been wrongly estimated. The fact, therefore, that
things often fail to fulfil our expectations is no evidence that our
expectations will not \emph{probably} be fulfilled in a given case or a
given class of cases. Thus our inductive principle is at any rate not
capable of being \emph{disproved} by an appeal to experience.

The inductive principle, however, is equally incapable of being
\emph{proved} by an appeal to experience. Experience might conceivably
confirm the inductive principle as regards the cases that have been
already examined; but as regards unexamined cases, it is the inductive
principle alone that can justify any inference from what has been
examined to what has not been examined. All arguments which, on the
basis of experience, argue as to the future or the unexperienced parts
of the past or present, assume the inductive principle; hence we can
never use experience to prove the inductive principle without begging
the question. Thus we must either accept the inductive principle on the
ground of its intrinsic evidence, or forgo all justification of our
expectations about the future. If the principle is unsound, we have no
reason to expect the sun to rise to-morrow, to expect bread to be more
nourishing than a stone, or to expect that if we throw ourselves off the
roof we shall fall. When we see what looks like our best friend
approaching us, we shall have no reason to suppose that his body is not
inhabited by the mind of our worst enemy or of some total stranger. All
our conduct is based upon associations which have worked in the past,
and which we therefore regard as likely to work in the future; and this
likelihood is dependent for its validity upon the inductive principle.

The general principles of science, such as the belief in the reign of
law, and the belief that every event must have a cause, are as
completely dependent upon the inductive principle as are the beliefs of
daily life All such general principles are believed because mankind have
found innumerable instances of their truth and no instances of their
falsehood. But this affords no evidence for their truth in the future,
unless the inductive principle is assumed.

Thus all knowledge which, on a basis of experience tells us something
about what is not experienced, is based upon a belief which experience
can neither confirm nor confute, yet which, at least in its more
concrete applications, appears to be as firmly rooted in us as many of
the facts of experience. The existence and justification of such
beliefs---for the inductive principle, as we shall see, is not the only
example---raises some of the most difficult and most debated problems of
philosophy. We will, in the next chapter, consider briefly what may be
said to account for such knowledge, and what is its scope and its degree
of certainty.

\protect\hypertarget{link2HCH0007}{}{}

\pagebreak
\section{Notes for the Student}
\markboth{CHAPTER 6 NOTES}{CHAPTER 6 NOTES}
In Ch 6, we consider David Hume's \textit{problem of induction}. The problem, in a nutshell, is that \textit{any} argument from past experience to justify the principle of induction is circular. And circular arguments are bad. Let us first see why circular arguments are bad.
\subsection*{Circular Arguments}
A \textit{circular argument} is one in which the premises presuppose the desired conclusion. There are more and less straightforward cases of this. Here is an easy example:
\begin{enumerate}
	\item Landon is a logic god. \hfill Premise
	\item Therefore, Landon is a logic god. \hfill From 1
\end{enumerate}
This sort of argument is bad because it will never convince someone that \textit{denies} (or \textit{does not yet accept}) the conclusion. Here is another slightly harder example:
\begin{enumerate}
	\item What scripture tell us is true. \hfill Premise
	\item Scripture tells us that God exists. \hfill Premise
	\item Thus, God exists. \hfill From 1
\end{enumerate}
Why is this circular? Because it presupposes that scripture is trustworthy. But the trustworthiness of scripture seems to presuppose that God exists. The basis for Premise (1) is:
\begin{enumerate}
	\item God wrote scripture. \hfill Premise
	\item So, what scripture tell us is true. \hfill From 1
	\item Scripture tells us that God exists. \hfill Premise
	\item Thus, God exists. \hfill From 2, 3
\end{enumerate}
But this will not convince anyone who doubts that God exists. So circular arguments are bad because they fail to \textit{rationally} convince anyone of their conclusion.
\subsection*{The Problem of Induction}
\par The problem of induction is that any argument for the principle of induction that relies on past experience is circular. Consider your justification for a typical claim about the future:
\begin{itemize} 
	\item The sun will rise this morning (\pageref{sun}).
\end{itemize}
Your argument for this likely to be something like this:
\begin{enumerate}
	\item In every past morning, the sun has risen. \hfill Premise
	\item If in every past morning, the sun has risen, then it will this morning. \hfill Premise
	\item Thus, the sun will rise this morning. \hfill 1, 2 MP
\end{enumerate}
Now this argument is circular. Why? Because any justification for Premise (2) is circular:
\begin{enumerate}
	\item If $A$ and $B$ have always been associated, then $A$ and $B$ will be associated. \hfill Principle of Induction
	\item Thus, if in every past morning [$A$] the sun has risen [$B$], then the sun will rise this morning. \hfill From 1
\end{enumerate}
We see that the argument for a connection between past sunrises and future ones just relies on the principle of induction. So we do not have a \textit{rational} basis for accepting the principle of induction (except by assuming the principle itself, which is circular). Uh-oh!
\subsection*{Hume's Fork}
Hume's argument has an extra step (\textit{An Enquiry Concerning Human Understanding}, \S IV):
\begin{quote}
	When a man says, \textit{I have found, in all past instances, such sensible qualities conjoined}...he is not guilty of a tautology, nor are these propositions in any respect the same. You say that the one proposition is an inference from the other. But you must confess that the inference is not intuitive; neither is it demonstrative: Of what nature is it, then? To say it is experimental, is begging the question. For all inferences from experience suppose, as their foundation, that the future will resemble the past, and that similar powers will be conjoined with similar sensible qualities. If there be any suspicion that the course of nature may change, and that the past may be no rule for the future, all experience becomes useless, and can give rise to no inference or conclusion. It is impossible, therefore, that any arguments from experience can prove this resemblance of the past to the future; since all these arguments are founded on the supposition of that resemblance.
\end{quote}
Hume here introduces what is known as \textit{Hume's Fork}: this is to break knowledge into two categories: what is justifiable independently of experience (or \textit{a priori}), and what is justifiable by experience (or \textit{a posteriori}). Hume then argues that the principle of induction is \textit{not} justifiable independently of experience. This is because, in his view, the only things that are justifiable in this way are tautologies, or sentences like \textit{It is raining or it is not raining}---sentences that would be true no matter what the world is like.

The principle of induction, Hume argues, is not a tautology. It is rather an informative, interesting claim that would, in some worlds, be false. (Pause to consider what a world in which the principle of induction \textit{would} be like!) So the principle of induction is not the sort of thing that can be demonstrated \textit{a priori} (by reason alone).

On the other hand, the principle of induction cannot be supported by experience, because that would be circular (by the argument we gave in the previous section). Oh, no!

Russell's way out of this problem---one many philosophers have taken---is to argue that, contrary to Hume, the principle of induction \textit{is} the sort of principle that can be justified \textit{a priori}---independently of experience, we have reasons to accept the principle of induction. 

We will see his argument in the next chapter. Stay tuned!

\hypertarget{chapter-vii.-on-our-knowledge-of-general-principles}{%
\chapter{On Our Knowledge of General Principles}\label{chapter-vii.-on-our-knowledge-of-general-principles}}

We saw in the preceding chapter that the principle of induction, while
necessary to the validity of all arguments based on experience, is
itself not capable of being proved by experience, and yet is
unhesitatingly believed by every one, at least in all its concrete
applications. In these characteristics the principle of induction does
not stand alone. There are a number of other principles which cannot be
proved or disproved by experience, but are used in arguments which start
from what is experienced.

Some of these principles have even greater evidence than the principle
of induction, and the knowledge of them has the same degree of certainty
as the knowledge of the existence of sense-data. They constitute the
means of drawing inferences from what is given in sensation; and if what
we infer is to be true, it is just as necessary that our principles of
inference should be true as it is that our data should be true. The
principles of inference are apt to be overlooked because of their very
obviousness---the assumption involved is assented to without our
realizing that it is an assumption. But it is very important to realize
the use of principles of inference, if a correct theory of knowledge is
to be obtained; for our knowledge of them raises interesting and
difficult questions.

In all our knowledge of general principles, what actually happens is
that first of all we realize some particular application of the
principle, and then we realize that the particularity is irrelevant, and
that there is a generality which may equally truly be affirmed. This is
of course familiar in such matters as teaching arithmetic:
`two and two are four' is first learnt
in the case of some particular pair of couples, and then in some other
particular case, and so on, until at last it becomes possible to see
that it is true of any pair of couples. The same thing happens with
logical principles. Suppose two men are discussing what day of the month
it is. One of them says, `At least you will admit that
\emph{if} yesterday was the 15th to-day must be the
16th.' `Yes', says the
other, `I admit that.'
`And you know', the first continues,
`that yesterday was the 15th, because you dined with
Jones, and your diary will tell you that was on the
15th.' `Yes', says the
second; `therefore to-day \emph{is} the
16th.'

Now such an argument is not hard to follow; and if it is granted that
its premisses are true in fact, no one will deny that the conclusion
must also be true. But it depends for its truth upon an instance of a
general logical principle. The logical principle is as follows:
`Suppose it known that \emph{if} this is true, then that
is true. Suppose it also known that this \emph{is} true, then it follows
that that is true.' When it is the case that if this is
true, that is true, we shall say that this
`implies' that, and that that
`follows from' this. Thus our principle
states that if this implies that, and this is true, then that is true.
\label{mp} In other words, `anything implied by a true proposition
is true', or `whatever follows from a
true proposition is true'.

This principle is really involved---at least, concrete instances of it
are involved---in all demonstrations. Whenever one thing which we
believe is used to prove something else, which we consequently believe,
this principle is relevant. If any one asks: `Why should
I accept the results of valid arguments based on true
premisses?' we can only answer by appealing to our
principle. In fact, the truth of the principle is impossible to doubt,
and its obviousness is so great that at first sight it seems almost
trivial. Such principles, however, are not trivial to the philosopher,
for they show that we may have indubitable knowledge which is in no way
derived from objects of sense.

The above principle is merely one of a certain number of self-evident
logical principles. Some at least of these principles must be granted
before any argument or proof becomes possible. When some of them have
been granted, others can be proved, though these others, so long as they
are simple, are just as obvious as the principles taken for granted. For
no very good reason, three of these principles have been singled out by
tradition under the name of `Laws of
Thought'.

They are as follows:
\begin{enumerate}[label=(\arabic*)]
	\item {\emph{The law of identity}: `Whatever is,
is.' \label{identity}} 
	\item {\emph{The law of contradiction}: `Nothing can both
be and not be.'  \label{lnc}}
	\item {\emph{The law of excluded middle}: `Everything must
either be or not be.' \label{lem}}
\end{enumerate}
These three laws are samples of self-evident logical principles, but are
not really more fundamental or more self-evident than various other
similar principles: for instance, the one we considered just now, which
states that what follows from a true premiss is true. The name
`laws of thought' is also misleading,
for what is important is not the fact that we think in accordance with
these laws, but the fact that things behave in accordance with them; in
other words, the fact that when we think in accordance with them we
think \emph{truly}. But this is a large question, to which we must
return at a later stage.

In addition to the logical principles which enable us to prove from a
given premiss that something is \emph{certainly} true, there are other
logical principles which enable us to prove, from a given premiss, that
there is a greater or less probability that something is true. \label{induction} An
example of such principles---perhaps the most important example is the
inductive principle, which we considered in the preceding chapter.

One of the great historic controversies in philosophy is the controversy
between the two schools called respectively
`empiricists' and
`rationalists'. The empiricists---who are
best represented by the British philosophers, Locke, Berkeley, and
Hume---maintained that all our knowledge is derived from experience; the
rationalists---who are represented by the Continental philosophers of
the seventeenth century, especially Descartes and Leibniz---maintained
that, in addition to what we know by experience, there are certain
`innate ideas' and
`innate principles', which we know
independently of experience. It has now become possible to decide with
some confidence as to the truth or falsehood of these opposing schools.
\label{rationalists} It must be admitted, for the reasons already stated, that logical
principles are known to us, and cannot be themselves proved by
experience, since all proof presupposes them. In this, therefore, which
was the most important point of the controversy, the rationalists were
in the right.

On the other hand, even that part of our knowledge which is
\emph{logically} independent of experience (in the sense that experience
cannot prove it) is yet elicited and caused by experience. It is on
occasion of particular experiences that we become aware of the general
laws which their connexions exemplify. It would certainly be absurd to
suppose that there are innate principles in the sense that babies are
born with a knowledge of everything which men know and which cannot be
deduced from what is experienced. For this reason, the word
`innate' would not now be employed to
describe our knowledge of logical principles. The phrase
'\emph{a priori}' is less
objectionable, and is more usual in modern writers. Thus, while
admitting that all knowledge is elicited and caused by experience, we
shall nevertheless hold that some knowledge is \emph{a priori}, in the
sense that the experience which makes us think of it does not suffice to
prove it, but merely so directs our attention that we see its truth
without requiring any proof from experience.

There is another point of great importance, in which the empiricists
were in the right as against the rationalists. \label{exist} Nothing can be known to
\emph{exist} except by the help of experience. That is to say, if we
wish to prove that something of which we have no direct experience
exists, we must have among our premisses the existence of one or more
things of which we have direct experience. Our belief that the Emperor
of China\label{China2} exists, for example, rests upon testimony, and testimony
consists, in the last analysis, of sense-data seen or heard in reading
or being spoken to. Rationalists believed that, from general
consideration as to what must be, they could deduce the existence of
this or that in the actual world. In this belief they seem to have been
mistaken. All the knowledge that we can acquire \emph{a priori}
concerning existence seems to be hypothetical: it tells us that if one
thing exists, another must exist, or, more generally, that if one
proposition is true, another must be true. This is exemplified by the
principles we have already dealt with, such as
`\emph{if} this is true, and this implies that, then
that is true', or `\emph{if} this and
that have been repeatedly found connected, they will probably be
connected in the next instance in which one of them is
found'. Thus the scope and power of \emph{a priori}
principles is strictly limited. All knowledge that something exists must
be in part dependent on experience. When anything is known immediately,
its existence is known by experience alone; when anything is proved to
exist, without being known immediately, both experience and \emph{a
priori} principles must be required in the proof. Knowledge is called
\emph{empirical} when it rests wholly or partly upon experience. Thus
all knowledge which asserts existence is empirical, and the only \emph{a
priori} knowledge concerning existence is hypothetical, giving
connexions among things that exist or may exist, but not giving actual
existence.

\emph{A priori} knowledge is not all of the logical kind we have been
hitherto considering. Perhaps the most important example of non-logical
\emph{a priori} knowledge is knowledge as to ethical value. I am not
speaking of judgements as to what is useful or as to what is virtuous,
for such judgements do require empirical premisses; I am speaking of
judgements as to the intrinsic desirability of things. If something is
useful, it must be useful because it secures some end; the end must, if
we have gone far enough, be valuable on its own account, and not merely
because it is useful for some further end. Thus all judgements as to
what is useful depend upon judgements as to what has value on its own
account.

\label{thegood} We judge, for example, that happiness is more desirable than misery,
knowledge than ignorance, goodwill than hatred, and so on. Such
judgements must, in part at least, be immediate and \emph{a priori}.
Like our previous \emph{a priori} judgements, they may be elicited by
experience, and indeed they must be; for it seems not possible to judge
whether anything is intrinsically valuable unless we have experienced
something of the same kind. But it is fairly obvious that they cannot be
proved by experience; for the fact that a thing exists or does not exist
cannot prove either that it is good that it should exist or that it is
bad. The pursuit of this subject belongs to ethics, where the
impossibility of deducing what ought to be from what is has to be
established. In the present connexion, it is only important to realize
that knowledge as to what is intrinsically of value is \emph{a priori}
in the same sense in which logic is \emph{a priori}, namely in the sense
that the truth of such knowledge can be neither proved nor disproved by
experience.

\label{puremath} All pure mathematics is \emph{a priori}, like logic. This was
strenuously denied by the empirical philosophers, who maintained that
experience was as much the source of our knowledge of arithmetic as of
our knowledge of geography. They maintained that by the repeated
experience of seeing two things and two other things, and finding that
altogether they made four things, we were led by induction to the
conclusion that two things and two other things would \emph{always} make
four things altogether. If, however, this were the source of our
knowledge that two and two are four, we should proceed differently, in
persuading ourselves of its truth, from the way in which we do actually
proceed. In fact, a certain number of instances are needed to make us
think of two abstractly, rather than of two coins or two books or two
people, or two of any other specified kind. But as soon as we are able
to divest our thoughts of irrelevant particularity, we become able to
see the general principle that two and two are four; any one instance is
seen to be \emph{typical}, and the examination of other instances
becomes unnecessary.\footnote{Cf. 
	\href{https://plato.stanford.edu/entries/whitehead/}{A. N. Whitehead}, 
	\href{https://archive.org/details/introductiontoma00whitiala/mode/2up}
	{\emph{Introduction to Mathematics}} (Home University Library).}

The same thing is exemplified in geometry. If we want to prove some
property of \emph{all} triangles, we draw some one triangle and reason
about it; but we can avoid making use of any property which it does not
share with all other triangles, and thus, from our particular case, we
obtain a general result. We do not, in fact, feel our certainty that two
and two are four increased by fresh instances, because, as soon as we
have seen the truth of this proposition, our certainty becomes so great
as to be incapable of growing greater. Moreover, we feel some quality of
necessity about the proposition `two and two are
four', which is absent from even the best attested
empirical generalizations. Such generalizations always remain mere
facts: we feel that there might be a world in which they were false,
though in the actual world they happen to be true. In any possible
world, on the contrary, we feel that two and two would be four: this is
not a mere fact, but a necessity to which everything actual and possible
must conform.

\label{empirical} The case may be made clearer by considering a genuinely-empirical
generalization, such as `All men are
mortal.' It is plain that we believe this proposition,
in the first place, because there is no known instance of men living
beyond a certain age, and in the second place because there seem to be
physiological grounds for thinking that an organism such as a
man's body must sooner or later wear out. Neglecting the
second ground, and considering merely our experience of
men's mortality, it is plain that we should not be
content with one quite clearly understood instance of a man dying,
whereas, in the case of `two and two are
four', one instance does suffice, when carefully
considered, to persuade us that the same must happen in any other
instance. Also we can be forced to admit, on reflection, that there may
be some doubt, however slight, as to whether \emph{all} men are mortal.
This may be made plain by the attempt to imagine two different worlds,
in one of which there are men who are not mortal, while in the other two
and two make five. When Swift invites us to consider the race of
Struldbugs who never die, we are able to acquiesce in imagination. But a
world where two and two make five seems quite on a different level. We
feel that such a world, if there were one, would upset the whole fabric
of our knowledge and reduce us to utter doubt.

The fact is that, in simple mathematical judgements such as
`two and two are four', and also in many
judgements of logic, we can know the general proposition without
inferring it from instances, although some instance is usually necessary
to make clear to us what the general proposition means. This is why
there is real utility in the process of \emph{deduction}, which goes
from the general to the general, or from the general to the particular,
as well as in the process of \emph{induction}, which goes from the
particular to the particular, or from the particular to the general. It
is an old debate among philosophers whether deduction ever gives
\emph{new} knowledge. We can now see that in certain cases, at least, it
does do so. If we already know that two and two always make four, and we
know that Brown and Jones are two, and so are Robinson and Smith, we can
deduce that Brown and Jones and Robinson and Smith are four. This is new
knowledge, not contained in our premisses, because the general
proposition, `two and two are four',
never told us there were such people as Brown and Jones and Robinson and
Smith, and the particular premisses do not tell us that there were four
of them, whereas the particular proposition deduced does tell us both
these things.

But the newness of the knowledge is much less certain if we take the
stock instance of deduction that is always given in books on logic,
namely, `All men are mortal; Socrates is a man,
therefore Socrates is mortal.' In this case, what we
really know beyond reasonable doubt is that certain men, A, B, C, were
mortal, since, in fact, they have died. If Socrates is one of these men,
it is foolish to go the roundabout way through `all men
are mortal' to arrive at the conclusion that
\emph{probably} Socrates is mortal. If Socrates is not one of the men on
whom our induction is based, we shall still do better to argue straight
from our A, B, C, to Socrates, than to go round by the general
proposition, `all men are mortal'. For
the probability that Socrates is mortal is greater, on our data, than
the probability that all men are mortal. (This is obvious, because if
all men are mortal, so is Socrates; but if Socrates is mortal, it does
not follow that all men are mortal.) Hence we shall reach the conclusion
that Socrates is mortal with a greater approach to certainty if we make
our argument purely inductive than if we go by way of
`all men are mortal' and then use
deduction.

This illustrates the difference between general propositions known
\emph{a priori} such as `two and two are
four', and empirical generalizations such as
`all men are mortal'. In regard to the
former, deduction is the right mode of argument, whereas in regard to
the latter, induction is always theoretically preferable, and warrants a
greater confidence in the truth of our conclusion, because all empirical
generalizations are more uncertain than the instances of them.

We have now seen that there are propositions known \emph{a priori}, and
that among them are the propositions of logic and pure mathematics, as
well as the fundamental propositions of ethics. The question which must
next occupy us is this: How is it possible that there should be such
knowledge? And more particularly, how can there be knowledge of general
propositions in cases where we have not examined all the instances, and
indeed never can examine them all, because their number is infinite?
These questions, which were first brought prominently forward by the
German philosopher \href{https://plato.stanford.edu/entries/kant/}{Kant} 
(1724-1804), are very difficult, and historically very important.

\protect\hypertarget{link2HCH0008}{}{}

\pagebreak

\section{Notes for the Student}
\markboth{CHAPTER 7 NOTES}{CHAPTER 7 NOTES}
We saw in Ch 6 that Hume's problem of induction can be put like this:
\begin{enumerate}
	\item If the principle of induction is justified, then it is either justified \textit{a priori} or justified \textit{a posteriori}. \hfill Hume's Fork
	\item The principle of induction is not justified \textit{a priori}. \hfill Premise
	\item The principle of induction is not justified \textit{a posteriori}. \hfill Premise
	\item Thus, the principle of induction is not justified. \hfill 1-3 MT
\end{enumerate}
Premise (2) is justified by a particular view of \textit{a priori} knowledge: on Hume's view, the only sorts of claims that can be justified \textit{a priori} are trivial tautologies, like \textit{it is raining or it is not raining}. Premise (3) is justified by a ban on non-circular arguments.

Russell challenges Premise (2). Russell thinks that \textit{a priori} justification can justify all sorts of claims besides tautologies (though it can also justify these). These claims include:
\begin{enumerate}
	\item Any claim implied by a true claim is also true (\pageref{mp}).
	\item The law of identity: any thing $x$ is itself (\pageref{identity}).
	\item The law of non-contradiction: no fact both obtains and does not obtain (\pageref{lnc}).
	\item The law of excluded middle: any fact either obtains or does not obtain (\pageref{lem}).
	\item The principle of induction (\pageref{induction}).
	\item Possibly, knowledge of what is good in itself (\pageref{thegood}).\footnote{You should be aware of, and familiar with, Russell's argument that claims about instrumental goodness are dependent upon claims about intrinsic goodness (\pageref{thegood}).}
	\item All claims of pure mathematics (\pageref{puremath}).
\end{enumerate}
Equally important are the claims that \textit{cannot} be justified \textit{a priori}:
\begin{enumerate}
	\item All claims as to what exists (\pageref{exist}).
	\item Empirical generalizations, like \textit{All humans are mortal} (\pageref{empirical}).
\end{enumerate}
But fundamentally, Russell is claiming, contrary to Hume, that we can justify \textit{a priori} claims that are genuinely informative. Russell takes up, then, in the next chapter, how such knowledge is possible. But it is important to bear in mind that, in rejecting Premise (2) of Hume's argument and taking this positive view of \textit{a priori} justification, we are rejecting \textit{empiricism}, the view that all justification is \textit{a posteriori}. As Russell says:
\begin{quote}
	It must be admitted, for the reasons already stated, that logical principles are known to us, and cannot be themselves proved by experience, since all proof presupposes them. In this, therefore...the rationalists were in the right. (\pageref{rationalists})
\end{quote}
Here is where you can step in and evaluate whether Russell's argument is persuasive or not!

\hypertarget{chapter-viii.-how-a-priori-knowledge-is-possible}{%
\chapter{How \emph{A Priori} Knowledge is Possible}}\label{chapter-viii.-how-a-priori-knowledge-is-possible}

Immanuel Kant is generally regarded as the greatest of the modern
philosophers. Though he lived through the Seven Years War and the French
Revolution, he never interrupted his teaching of philosophy at
Königsberg in East Prussia. His most distinctive contribution was the
invention of what he called the
`critical' philosophy, which, assuming
as a datum that there is knowledge of various kinds, inquired how such
knowledge comes to be possible, and deduced, from the answer to this
inquiry, many metaphysical results as to the nature of the world.
Whether these results were valid may well be doubted. But Kant
undoubtedly deserves credit for two things: first, for having perceived
that we have \emph{a priori} knowledge which is not purely
`analytic', i.e. such that the opposite
would be self-contradictory, and secondly, for having made evident the
philosophical importance of the theory of knowledge.

Before the time of Kant, it was generally held that whatever knowledge
was \emph{a priori} must be `analytic'. \label{analytic}
What this word means will be best illustrated by examples. If I say,
`A bald man is a man', `A
plane figure is a figure', `A bad poet is
a poet', I make a purely analytic judgement: the subject
spoken about is given as having at least two properties, of which one is
singled out to be asserted of it. Such propositions as the above are
trivial, and would never be enunciated in real life except by an orator
preparing the way for a piece of sophistry. They are called
`analytic' because the predicate is
obtained by merely analysing the subject. Before the time of Kant it was
thought that all judgements of which we could be certain \emph{a priori}
were of this kind: that in all of them there was a predicate which was
only part of the subject of which it was asserted. If this were so, we
should be involved in a definite contradiction if we attempted to deny
anything that could be known \emph{a priori}. `A bald
man is not bald' would assert and deny baldness of the
same man, and would therefore contradict itself. Thus according to the
philosophers before Kant, the law of contradiction, which asserts that
nothing can at the same time have and not have a certain property,
sufficed to establish the truth of all \emph{a priori} knowledge.

\href{https://plato.stanford.edu/entries/hume/}{Hume} (1711-76), 
who preceded Kant, accepting the usual view as to what
makes knowledge \emph{a priori}, discovered that, in many cases which
had previously been supposed analytic, and notably in the case of cause
and effect, the connexion was really synthetic. Before Hume,
rationalists at least had supposed that the effect could be logically
deduced from the cause, if only we had sufficient knowledge. Hume
argued---correctly, as would now be generally admitted---that this could
not be done. Hence he inferred the far more doubtful proposition that
nothing could be known \emph{a priori} about the connexion of cause and
effect. Kant, who had been educated in the rationalist tradition, was
much perturbed by Hume's scepticism, and endeavoured to
find an answer to it. He perceived that not only the connexion of cause
and effect, but all the propositions of arithmetic and geometry, are
`synthetic', i.e. not analytic: in all
these propositions, no analysis of the subject will reveal the
predicate. His stock instance was the proposition 7 + 5 = 12. He pointed
out, quite truly, that 7 and 5 have to be put together to give 12: the
idea of 12 is not contained in them, nor even in the idea of adding them
together. Thus he was led to the conclusion that all pure mathematics,
though \emph{a priori}, is synthetic; and this conclusion raised a new
problem of which he endeavoured to find the solution.

The question which Kant put at the beginning of his philosophy, namely
`How is pure mathematics possible?' is
an interesting and difficult one, to which every philosophy which is not
purely sceptical must find some answer. \label{instances} The answer of the pure
empiricists, that our mathematical knowledge is derived by induction
from particular instances, we have already seen to be inadequate, for
two reasons: first, that the validity of the inductive principle itself
cannot be proved by induction; secondly, that the general propositions
of mathematics, such as `two and two always make
four', can obviously be known with certainty by
consideration of a single instance, and gain nothing by enumeration of
other cases in which they have been found to be true. Thus our knowledge
of the general propositions of mathematics (and the same applies to
logic) must be accounted for otherwise than our (merely probable)
knowledge of empirical generalizations such as `all men
are mortal'.

The problem arises through the fact that such knowledge is general,
whereas all experience is particular. It seems strange that we should
apparently be able to know some truths in advance about particular
things of which we have as yet no experience; but it cannot easily be
doubted that logic and arithmetic will apply to such things. We do not
know who will be the inhabitants of London a hundred years hence; but we
know that any two of them and any other two of them will make four of
them. This apparent power of anticipating facts about things of which we
have no experience is certainly surprising. Kant's
solution of the problem, though not valid in my opinion, is interesting.
It is, however, very difficult, and is differently understood by
different philosophers. We can, therefore, only give the merest outline
of it, and even that will be thought misleading by many exponents of
Kant's system.

\label{Kant} What Kant maintained was that in all our experience there are two
elements to be distinguished, the one due to the object (i.e. to what we
have called the `physical object'), the
other due to our own nature. We saw, in discussing matter and
sense-data, that the physical object is different from the associated
sense-data, and that the sense-data are to be regarded as resulting from
an interaction between the physical object and ourselves. So far, we are
in agreement with Kant. But what is distinctive of Kant is the way in
which he apportions the shares of ourselves and the physical object
respectively. He considers that the crude material given in
sensation---the colour, hardness, etc.---is due to the object, and that
what we supply is the arrangement in space and time, and all the
relations between sense-data which result from comparison or from
considering one as the cause of the other or in any other way. His chief
reason in favour of this view is that we seem to have \emph{a priori}
knowledge as to space and time and causality and comparison, but not as
to the actual crude material of sensation. We can be sure, he says, that
anything we shall ever experience must show the characteristics affirmed
of it in our \emph{a priori} knowledge, because these characteristics
are due to our own nature, and therefore nothing can ever come into our
experience without acquiring these characteristics.

The physical object, which he calls the `thing in
itself',\footnote{Kant's `thing in
	itself' is identical \emph{in definition} with the
	physical object, namely, it is the cause of sensations. In the
	properties deduced from the definition it is not identical, since Kant
	held (in spite of some inconsistency as regards cause) that we can know
	that none of the categories are applicable to the `thing
	in itself'.} he regards as essentially unknowable; what
can be known is the object as we have it in experience, which he calls
the `phenomenon'. The phenomenon, being a
joint product of us and the thing in itself, is sure to have those
characteristics which are due to us, and is therefore sure to conform to
our \emph{a priori} knowledge. Hence this knowledge, though true of all
actual and possible experience, must not be supposed to apply outside
experience. Thus in spite of the existence of \emph{a priori} knowledge,
we cannot know anything about the thing in itself or about what is not
an actual or possible object of experience. In this way he tries to
reconcile and harmonize the contentions of the rationalists with the
arguments of the empiricists.

Apart from minor grounds on which Kant's philosophy may
be criticized, there is one main objection which seems fatal to any
attempt to deal with the problem of \emph{a priori} knowledge by his
method. \label{certainty} The thing to be accounted for is our certainty that the facts
must always conform to logic and arithmetic. To say that logic and
arithmetic are contributed by us does not account for this. Our nature
is as much a fact of the existing world as anything, and there can be no
certainty that it will remain constant. It might happen, if Kant is
right, that to-morrow our nature would so change as to make two and two
become five. This possibility seems never to have occurred to him, yet
it is one which utterly destroys the certainty and universality which he
is anxious to vindicate for arithmetical propositions. It is true that
this possibility, formally, is inconsistent with the Kantian view that
time itself is a form imposed by the subject upon phenomena, so that our
real Self is not in time and has no to-morrow. But he will still have to
suppose that the time-order of phenomena is determined by
characteristics of what is behind phenomena, and this suffices for the
substance of our argument.

Reflection, moreover, seems to make it clear that, if there is any truth
in our arithmetical beliefs, they must apply to things equally whether
we think of them or not. Two physical objects and two other physical
objects must make four physical objects, even if physical objects cannot
be experienced. To assert this is certainly within the scope of what we
mean when we state that two and two are four. Its truth is just as
indubitable as the truth of the assertion that two phenomena and two
other phenomena make four phenomena. Thus Kant's
solution unduly limits the scope of \emph{a priori} propositions, in
addition to failing in the attempt at explaining their certainty.

Apart from the special doctrines advocated by Kant, it is very common
among philosophers to regard what is \emph{a priori} as in some sense
mental, as concerned rather with the way we must think than with any
fact of the outer world. We noted in the preceding chapter the three
principles commonly called `laws of
thought'. The view which led to their being so named is a
natural one, but there are strong reasons for thinking that it is
erroneous. Let us take as an illustration the law of contradiction. This
is commonly stated in the form `Nothing can both be and
not be', which is intended to express the fact that
nothing can at once have and not have a given quality. Thus, for
example, if a tree is a beech it cannot also be not a beech; if my table
is rectangular it cannot also be not rectangular, and so on.

Now what makes it natural to call this principle a law of \emph{thought}
is that it is by thought rather than by outward observation that we
persuade ourselves of its necessary truth. When we have seen that a tree
is a beech, we do not need to look again in order to ascertain whether
it is also not a beech; thought alone makes us know that this is
impossible. But the conclusion that the law of contradiction is a law of
\emph{thought} is nevertheless erroneous. What we believe, when we
believe the law of contradiction, is not that the mind is so made that
it must believe the law of contradiction. \emph{This} belief is a
subsequent result of psychological reflection, which presupposes the
belief in the law of contradiction. The belief in the law of
contradiction is a belief about things, not only about thoughts. It is
not, e.g., the belief that if we \emph{think} a certain tree is a beech,
we cannot at the same time \emph{think} that it is not a beech; it is
the belief that if the tree \emph{is} a beech, it cannot at the same
time \emph{be} not a beech. Thus the law of contradiction is about
things, and not merely about thoughts; and although belief in the law of
contradiction is a thought, the law of contradiction itself is not a
thought, but a fact concerning the things in the world. If this, which
we believe when we believe the law of contradiction, were not true of
the things in the world, the fact that we were compelled to \emph{think}
it true would not save the law of contradiction from being false; and
this shows that the law is not a law of \emph{thought}.

A similar argument applies to any other \emph{a priori} judgement. When
we judge that two and two are four, we are not making a judgement about
our thoughts, but about all actual or possible couples. The fact that
our minds are so constituted as to believe that two and two are four,
though it is true, is emphatically not what we assert when we assert
that two and two are four. And no fact about the constitution of our
minds could make it \emph{true} that two and two are four. Thus our
\emph{a priori} knowledge, if it is not erroneous, is not merely
knowledge about the constitution of our minds, but is applicable to
whatever the world may contain, both what is mental and what is
non-mental.

The fact seems to be that all our \emph{a priori} knowledge is concerned
with entities which do not, properly speaking, \emph{exist}, either in
the mental or in the physical world. These entities are such as can be
named by parts of speech which are not substantives; they are such
entities as qualities and relations. Suppose, for instance, that I am in
my room. I exist, and my room exists; but does
`in' exist? Yet obviously the word
`in' has a meaning; it denotes a
relation which holds between me and my room. This relation is something,
although we cannot say that it exists \emph{in the same sense} in which
I and my room exist. The relation `in'
is something which we can think about and understand, for, if we could
not understand it, we could not understand the sentence
`I am in my room'. Many philosophers,
following Kant, have maintained that relations are the work of the mind,
that things in themselves have no relations, but that the mind brings
them together in one act of thought and thus produces the relations
which it judges them to have.

This view, however, seems open to objections similar to those which we
urged before against Kant. It seems plain that it is not thought which
produces the truth of the proposition `I am in my
room'. It may be true that an earwig is in my room, even
if neither I nor the earwig nor any one else is aware of this truth; for
this truth concerns only the earwig and the room, and does not depend
upon anything else. Thus relations, as we shall see more fully in the
next chapter, must be placed in a world which is neither mental nor
physical. This world is of great importance to philosophy, and in
particular to the problems of \emph{a priori} knowledge. In the next
chapter we shall proceed to develop its nature and its bearing upon the
questions with which we have been dealing.

\protect\hypertarget{link2HCH0009}{}{}

\pagebreak
\section{Notes for the Student}
\markboth{CHAPTER 8 NOTES}{CHAPTER 8 NOTES}
As we saw in Ch 7, Russell maintains that we have \textit{a priori} knowledge that is not uninformative and tautologous. Russell clarifies this with a standard distinction between \textit{analytic} and \textit{synthetic} claims. He draws on Kant's work, \textit{The Critique of Pure Reason}. 

Now let us first understand what it means to say of a claim that it is not analytic, but rather it is synthetic. Kant explains the distinction thusly (\textit{Critique of Pure Reason}, \S IV):
\begin{quote}
	In all judgments in which the relation of a subject to the predicate is thought...this relation is possible in two different ways. Either the predicate \textit{B} belongs to the subject \textit{A} as something that is (covertly) contained in this concept \textit{A}; or \textit{B} lies entirely outside the concept \textit{A}, though to be sure it stands in connection with it. In the first case I call the judgment \textbf{analytic}, in the second \textbf{synthetic}. Analytic judgments (affirmative ones) are thus those in which the connection of the predicate is thought through identity, but those in which this connection is thought without identity are to be called synthetic judgments. 
\end{quote}
So to say that a claim is \textit{analytic} is to say that the claim is true in virtue of the \textit{meanings} of the constituent concepts. The say that a claim is \textit{synthetic} is to say that the claim is not analytic: neither concept contains the other in its meaning. Let us see an example.

Take the claim that \textit{all bachelors are unmarried}. This claim is seen to be true once we unpack what it means to be a `bachelor'. For \textit{being unmarried} is contained in the concept of \textit{being a bachelor}. A test for this is: if someone grasped the concept of \textit{being a bachelor}, then they would grasp the concept of \textit{being unmarried}. In contrast, you could grasp the concept of \textit{being the sum of seven and five} without grasping the concept of \textit{being twelve}, and so $7+5=12$ is not analytic (\textit{Critique of Pure Reason}, \S V):
\begin{quote}
	To be sure, one might initially think that the proposition ``$7 + 5 = 12$" is a merely analytic proposition that follows from the concept of a sum of seven and give in accordance with the principle of contradiction. Yet if one considers it more closely, one finds that the concept of the sum of 7 and 5 contains nothing more than the unification of both numbers in a single one, through which it is not at all thought what this single number is which comprehends the two of them. The concept of twelve is by no means already thought merely by my thinking of that unification of seven and five, and no matter how long I analyze my concept of such a possible sum I will not find twelve in it. 
\end{quote}
Kant is applying the test to see whether the concept , reflecting on this, concludes that, the concept of \textit{being the sum of seven and five}, though it may contain the concepts of \textit{being seven}, \textit{being five}, and \textit{the sum of}, does not contain the concept of \textit{being twelve} as a part.

Let us take one more example. Consider the claim \textit{all humans are mortal}. This claim is, arguably, true. But how do we justify it? Presumably, not from reflection on the mere concepts of \textit{being human} and \textit{being mortal}. One could have the concept of \textit{being human} without grasping the concept of \textit{being mortal}. This is why Kant---and Russell---classify this claim as synthetic rather than analytic. So it is an inductive generalization from experience.

Kant, reflecting on this example and others, concludes, ``\textbf{Mathematical judgments are all synthetic}." Likewise, ``\textbf{Judgments of experience, as such, are all synthetic}." 
\subsection*{Returning to Russell}
Recall Russell's examples of \textit{a priori} justifiable but synthetic claims:
\begin{enumerate}
	\item Any claim implied by a true claim is also true (\pageref{mp}).
	\item The law of identity: any thing $x$ is itself (\pageref{identity}).
	\item The law of non-contradiction: no fact both obtains and does not obtain (\pageref{lnc}).
	\item The law of excluded middle: any fact either obtains or does not obtain (\pageref{lem}).
	\item The principle of induction (\pageref{induction}).
	\item Possibly, knowledge of what is good in itself (\pageref{thegood}).\footnote{You should be aware of, and familiar with, Russell's argument that claims about instrumental goodness are dependent upon claims about intrinsic goodness (\pageref{thegood}).}
	\item All claims of pure mathematics (\pageref{puremath}).
\end{enumerate}
Like Kant, Russell believes that such claims are not true \textit{merely in virtue of what they mean}. They are not analytic. Analytic claims, rather, are like `A bald man is a man', `A plane figure is a figure', `A bad poet is a poet', and so on (\pageref{analytic}). In these cases of \textit{analytic} claims, analyzing the subject term reveals that it contains the predicate term.\footnote{In the claim `a \textit{plane figure} is a \textit{figure}', `plane figure' is the subject term and `figure' is the predicate term. Generally, in `$s$ is $P$', the subject term `$s$' is what gets ascribed the predicate term `$P$'.}

We already saw why Russell thinks that the empiricists were wrong to say that all justification comes from experience. Russell rehashes this argument briefly. First, the principle of induction can only be justified by experience in a circular way; because circular arguments are bad, this is just to say that it cannot be justified by experience; second, we have justification for logical and mathematical claims like $2+2=4$ independently of experience, unlike claims like \textit{all humans are mortal} that must be justified partly by experience (\pageref{instances}).

But how is it that we have such knowledge at all? This is a difficult question. Kant's answer---paraphrasing---was to make these claims part of the scaffolding of experience (\pageref{Kant}). Kant claimed that we could not have experiences at all without them being structured or scaffolded according to the categories of \textit{space}, \textit{time}, \textit{cause}, and \textit{comparison}. The inquiry into these categories of cognition itself, rather than what was being cognized, was what Kant called \textit{transcendental philosophy} (\textit{Critique of Pure Reason}, \S VII).

In Russell's view, this makes \textit{a priori} justification too mental. It makes \textit{a priori} justification about laws of possible thinking rather than about the world, and fails to explain why \textit{a priori} claims should be \textit{necessary} rather than claims about how our cognitive faculties \textit{contingently} are. A claim like $2+2=4$ is necessarily true, but how we are is contingent. So how we are cannot be what justifies \textit{a priori} that $2+2=4$ (\pageref{certainty}).

So Russell argues. If he is right, there is no hope for accounting for our knowledge of mathematics through a transcendental philosophy like Kant proposed. But we need to recover our knowledge of mathematics. How do we do that? Russell hints at the answer, and he takes it up in the next chapter: \textit{a priori} justification is about abstract entities existing outside space and time that exist necessarily and unchangingly. These are called \textit{universals}.

\hypertarget{chapter-ix.-the-world-of-universals}{%
\chapter{The World of Universals}\label{chapter-ix.-the-world-of-universals}}

At the end of the preceding chapter we saw that such entities as
relations appear to have a being which is in some way different from
that of physical objects, and also different from that of minds and from
that of sense-data. In the present chapter we have to consider what is
the nature of this kind of being, and also what objects there are that
have this kind of being. We will begin with the latter question.

The problem with which we are now concerned is a very old one, since it
was brought into philosophy by \href{https://plato.stanford.edu/entries/plato/}{Plato}. 
Plato's `theory of ideas' is an attempt to
solve this very problem, and in my opinion it is one of the most
successful attempts hitherto made. The theory to be advocated in what
follows is largely Plato's, with merely such
modifications as time has shown to be necessary.

The way the problem arose for Plato was more or less as follows. Let us
consider, say, such a notion as \emph{justice}. \label{tokens} If we ask ourselves what
justice is, it is natural to proceed by considering this, that, and the
other just act, with a view to discovering what they have in common.
They must all, in some sense, partake of a common nature, which will be
found in whatever is just and in nothing else. This common nature, in
virtue of which they are all just, will be justice itself, the pure
essence the admixture of which with facts of ordinary life produces the
multiplicity of just acts. Similarly with any other word which may be
applicable to common facts, such as
`whiteness' for example. The word will
be applicable to a number of particular things because they all
participate in a common nature or essence. \label{types} This pure essence is what
Plato calls an `idea' or
`form'. (It must not be supposed that
`ideas', in his sense, exist in minds,
though they may be apprehended by minds.) The
`idea' \emph{justice} is not identical
with anything that is just: it is something other than particular
things, which particular things partake of. Not being particular, it
cannot itself exist in the world of sense. Moreover it is not fleeting
or changeable like the things of sense: it is eternally itself,
immutable and indestructible.

Thus Plato is led to a supra-sensible world, more real than the common
world of sense, the unchangeable world of ideas, which alone gives to
the world of sense whatever pale reflection of reality may belong to it.
The truly real world, for Plato, is the world of ideas; for whatever we
may attempt to say about things in the world of sense, we can only
succeed in saying that they participate in such and such ideas, which,
therefore, constitute all their character. Hence it is easy to pass on
into a mysticism. We may hope, in a mystic illumination, to see the
ideas as we see objects of sense; and we may imagine that the ideas
exist in heaven. These mystical developments are very natural, but the
basis of the theory is in logic, and it is as based in logic that we
have to consider it.

The word `idea' has acquired, in the
course of time, many associations which are quite misleading when
applied to Plato's
`ideas'. We shall therefore use the word
`universal' instead of the word
`idea', to describe what Plato meant. The
essence of the sort of entity that Plato meant is that it is opposed to
the particular things that are given in sensation. We speak of whatever
is given in sensation, or is of the same nature as things given in
sensation, as a \emph{particular}; by opposition to this, a
\emph{universal} will be anything which may be shared by many
particulars, and has those characteristics which, as we saw, distinguish
justice and whiteness from just acts and white things.

When we examine common words, we find that, broadly speaking, proper
names stand for particulars, while other substantives, adjectives,
prepositions, and verbs stand for universals. Pronouns stand for
particulars, but are ambiguous: it is only by the context or the
circumstances that we know what particulars they stand for. The word
`now' stands for a particular, namely
the present moment; but like pronouns, it stands for an ambiguous
particular, because the present is always changing.

It will be seen that no sentence can be made up without at least one
word which denotes a universal. The nearest approach would be some such
statement as `I like this'. But even here
the word `like' denotes a universal,
for I may like other things, and other people may like things. Thus all
truths involve universals, and all knowledge of truths involves
acquaintance with universals.

Seeing that nearly all the words to be found in the dictionary stand for
universals, it is strange that hardly anybody except students of
philosophy ever realizes that there are such entities as universals. We
do not naturally dwell upon those words in a sentence which do not stand
for particulars; and if we are forced to dwell upon a word which stands
for a universal, we naturally think of it as standing for some one of
the particulars that come under the universal. When, for example, we
hear the sentence, `Charles I's head was
cut off', we may naturally enough think of Charles I, of
Charles I's head, and of the operation of cutting off
\emph{his} head, which are all particulars; but we do not naturally
dwell upon what is meant by the word
`head' or the word
`cut', which is a universal: We feel such
words to be incomplete and insubstantial; they seem to demand a context
before anything can be done with them. Hence we succeed in avoiding all
notice of universals as such, until the study of philosophy forces them
upon our attention.

Even among philosophers, we may say, broadly, that only those universals
which are named by adjectives or substantives have been much or often
recognized, while those named by verbs and prepositions have been
usually overlooked. This omission has had a very great effect upon
philosophy; it is hardly too much to say that most metaphysics, since
Spinoza, has been largely determined by it. The way this has occurred
is, in outline, as follows: Speaking generally, adjectives and common
nouns express qualities or properties of single things, whereas
prepositions and verbs tend to express relations between two or more
things. Thus the neglect of prepositions and verbs led to the belief
that every proposition can be regarded as attributing a property to a
single thing, rather than as expressing a relation between two or more
things. Hence it was supposed that, ultimately, there can be no such
entities as relations between things. Hence either there can be only one
thing in the universe, or, if there are many things, they cannot
possibly interact in any way, since any interaction would be a relation,
and relations are impossible.

The first of these views, advocated by \href{https://plato.stanford.edu/entries/spinoza/}{Spinoza} 
and held in our own day by \href{https://plato.stanford.edu/entries/bradley/}{Bradley} 
and many other philosophers, is called \emph{monism}; the
second, advocated by Leibniz but not very common nowadays, is called
\emph{monadism}, because each of the isolated things is called a
\emph{monad}. Both these opposing philosophies, interesting as they are,
result, in my opinion, from an undue attention to one sort of
universals, namely the sort represented by adjectives and substantives
rather than by verbs and prepositions.

As a matter of fact, if any one were anxious to deny altogether that
there are such things as universals, we should find that we cannot
strictly prove that there are such entities as \emph{qualities}, i.e.
the universals represented by adjectives and substantives, whereas we
can prove that there must be \emph{relations}, i.e. the sort of
universals generally represented by verbs and prepositions. Let us take
in illustration the universal \emph{whiteness}. If we believe that there
is such a universal, we shall say that things are white because they
have the quality of whiteness. This view, however, was strenuously
denied by Berkeley and Hume, who have been followed in this by later
empiricists. \label{resemblances} The form which their denial took was to deny that there are
such things as `abstract ideas '. When we
want to think of whiteness, they said, we form an image of some
particular white thing, and reason concerning this particular, taking
care not to deduce anything concerning it which we cannot see to be
equally true of any other white thing. As an account of our actual
mental processes, this is no doubt largely true. In geometry, for
example, when we wish to prove something about all triangles, we draw a
particular triangle and reason about it, taking care not to use any
characteristic which it does not share with other triangles. The
beginner, in order to avoid error, often finds it useful to draw several
triangles, as unlike each other as possible, in order to make sure that
his reasoning is equally applicable to all of them. \label{regress} But a difficulty
emerges as soon as we ask ourselves how we know that a thing is white or
a triangle. If we wish to avoid the universals \emph{whiteness} and
\emph{triangularity}, we shall choose some particular patch of white or
some particular triangle, and say that anything is white or a triangle
if it has the right sort of resemblance to our chosen particular. But
then the resemblance required will have to be a universal. Since there
are many white things, the resemblance must hold between many pairs of
particular white things; and this is the characteristic of a universal.
It will be useless to say that there is a different resemblance for each
pair, for then we shall have to say that these resemblances resemble
each other, and thus at last we shall be forced to admit resemblance as
a universal. The relation of resemblance, therefore, must be a true
universal. And having been forced to admit this universal, we find that
it is no longer worth while to invent difficult and unplausible theories
to avoid the admission of such universals as whiteness and
triangularity.

Berkeley and Hume failed to perceive this refutation of their rejection
of `abstract ideas', because, like their
adversaries, they only thought of \emph{qualities}, and altogether
ignored \emph{relations} as universals. We have therefore here another
respect in which the rationalists appear to have been in the right as
against the empiricists, although, owing to the neglect or denial of
relations, the deductions made by rationalists were, if anything, more
apt to be mistaken than those made by empiricists.

Having now seen that there must be such entities as universals, the next
point to be proved is that their being is not merely mental. By this is
meant that whatever being belongs to them is independent of their being
thought of or in any way apprehended by minds. We have already touched
on this subject at the end of the preceding chapter, but we must now
consider more fully what sort of being it is that belongs to universals.

Consider such a proposition as `Edinburgh is north of
London'. Here we have a relation between two places, and
it seems plain that the relation subsists independently of our knowledge
of it. When we come to know that Edinburgh is north of London, we come
to know something which has to do only with Edinburgh and London: we do
not cause the truth of the proposition by coming to know it, on the
contrary we merely apprehend a fact which was there before we knew it.
The part of the earth's surface where Edinburgh stands
would be north of the part where London stands, even if there were no
human being to know about north and south, and even if there were no
minds at all in the universe. This is, of course, denied by many
philosophers, either for Berkeley's reasons or for
Kant's. But we have already considered these reasons,
and decided that they are inadequate. We may therefore now assume it to
be true that nothing mental is presupposed in the fact that Edinburgh is
north of London. But this fact involves the relation
`north of', which is a universal; and it
would be impossible for the whole fact to involve nothing mental if the
relation `north of', which is a
constituent part of the fact, did involve anything mental. Hence we must
admit that the relation, like the terms it relates, is not dependent
upon thought, but belongs to the independent world which thought
apprehends but does not create.

This conclusion, however, is met by the difficulty that the relation
`north of' does not seem to
\emph{exist} in the same sense in which Edinburgh and London exist. If
we ask `Where and when does this relation
exist?' the answer must be \label{nowhereandnowhen} `Nowhere and
nowhen'. There is no place or time where we can find the
relation `north of'. It does not exist in
Edinburgh any more than in London, for it relates the two and is neutral
as between them. Nor can we say that it exists at any particular time.
Now everything that can be apprehended by the senses or by introspection
exists at some particular time. Hence the relation
`north of' is radically different from
such things. It is neither in space nor in time, neither material nor
mental; yet it is something.

It is largely the very peculiar kind of being that belongs to universals
which has led many people to suppose that they are really mental. We can
think \emph{of} a universal, and our thinking then exists in a perfectly
ordinary sense, like any other mental act. Suppose, for example, that we
are thinking of whiteness. Then \emph{in one sense} it may be said that
whiteness is `in our mind'. We have here
the same ambiguity as we noted in discussing Berkeley in Chapter IV. In
the strict sense, it is not whiteness that is in our mind, but the act
of thinking of whiteness. The connected ambiguity in the word
`idea', which we noted at the same time,
also causes confusion here. In one sense of this word, namely the sense
in which it denotes the \emph{object} of an act of thought, whiteness is
an `idea'. Hence, if the ambiguity is not
guarded against, we may come to think that whiteness is an
`idea' in the other sense, i.e. an act
of thought; and thus we come to think that whiteness is mental. But in
so thinking, we rob it of its essential quality of universality. One
man's act of thought is necessarily a different thing
from another man's; one man's act of
thought at one time is necessarily a different thing from the same
man's act of thought at another time. Hence, if
whiteness were the thought as opposed to its object, no two different
men could think of it, and no one man could think of it twice. That
which many different thoughts of whiteness have in common is their
\emph{object}, and this object is different from all of them. Thus
universals are not thoughts, though when known they are the objects of
thoughts.

We shall find it convenient only to speak of things \emph{existing} when
they are in time, that is to say, when we can point to some time at
which they exist (not excluding the possibility of their existing at all
times). Thus thoughts and feelings, minds and physical objects exist.
\label{subsist} But universals do not exist in this sense; we shall say that they
\emph{subsist} or \emph{have being}, where
`being' is opposed to
`existence' as being timeless. The
world of universals, therefore, may also be described as the world of
being. The world of being is unchangeable, rigid, exact, delightful to
the mathematician, the logician, the builder of metaphysical systems,
and all who love perfection more than life. The world of existence is
fleeting, vague, without sharp boundaries, without any clear plan or
arrangement, but it contains all thoughts and feelings, all the data of
sense, and all physical objects, everything that can do either good or
harm, everything that makes any difference to the value of life and the
world. According to our temperaments, we shall prefer the contemplation
of the one or of the other. The one we do not prefer will probably seem
to us a pale shadow of the one we prefer, and hardly worthy to be
regarded as in any sense real. But the truth is that both have the same
claim on our impartial attention, both are real, and both are important
to the metaphysician. Indeed no sooner have we distinguished the two
worlds than it becomes necessary to consider their relations.

But first of all we must examine our knowledge of universals. This
consideration will occupy us in the following chapter, where we shall
find that it solves the problem of \emph{a priori} knowledge, from which
we were first led to consider universals.

\protect\hypertarget{link2HCH0010}{}{}

\pagebreak
\section{Notes for the Student}
\markboth{CHAPTER 9 NOTES}{CHAPTER 9 NOTES}
As we saw in Ch  8, Russell wants \textit{a priori} justification to be both necessary and about real objects, not merely about how we are built. Given that constraint, transcendental philosophy is out. What is needed are a certain kind of real object: eternal, unchanging, necessarily existing entities that can be the subject-matter of mathematics and logic.
\subsection*{Introducing Universals}
Universals are just the eternal, unchanging, necessarily existing entities that logic and math is all about! These are the entities that Russell sought. Let us first see why we need them.

The American philosopher Charles Peirce coined the distinction between \textit{types} and \textit{tokens}. We will ignore Peirce's way of putting the distinction (you can review the text on our course website if you like). But consider the following sentence: `Red books are red things.' This sentence contains the word `red'. But it contains this same word more than once: the word `red' occurs here twice. When we mean to talk about a specific event or thing, we call this a \textit{token} event or thing. When we mean to talk about what the common feature of specific events or things, we mean to talk about the \textit{type} of which these tokens are instances. 

So in the sentence `Red books are red things', there are two tokens, or occurrences, of the word `red', but they are tokens of the same type, the word `red' itself.

Now what is this type of thing? Plato raised this issue centuries ago: in considering various kinds of just acts, one might wonder what they are all instances of (\pageref{tokens}). When we wonder about this, we are wondering about their common feature: we are asking about what makes it the case that these are all tokens of some one type like \textit{justice}.

One account of what makes specific events or things the same type, or the same sort of thing, is that the type is present in them all. If there are a number of red books, then these all have the feature of \textit{being red} or \textit{redness}. If I perform a number of just acts, they all have the feature of \textit{being just} or \textit{justice}. Now Russell says that these tokens of redness or justice, these red things and just acts---which he calls \textit{particulars}---are all said to be `red' or `just' when they share in the same \textit{universal}: universals are what, for Russell, a type is (\pageref{types}. 
\section*{Russell's Regress Argument for Universals}
Not everyone likes universals. They seem spooky to some philosophers, like Berkeley and Hume. Roughly, the worry is this: I see particular red things. But I do not see a universal \textit{redness} as some property share by all the red things that there are. 

Now the problem for those that deny the existence of universals is that there is something that all red things share. We do correctly call them all `red', after all. Berkeley and Hume tried to say that what makes them all red is that they \textit{resemble} some paradigmatic red thing (\pageref{resemblances}). Russell argues that this account of what a type is fails miserably. It suffers from an infinite regress. Here is Russell's \textit{regress argument for universals} (\pageref{regress}):
\begin{enumerate}%[Russell's Regress Argument for Universals] 
	\item If two red books $x$ and $y$ resemble each other without there being universals, then they resemble a third red book $r$. \hfill Premise
	\item If two red books $x$ and $y$ resemble a third red book $r$, then $x$ resembles $r$ and $y$ resembles $r$. \hfill Premise
	\item If $x$ resembles $r$ and $y$ resembles $r$, then \textit{resemblance} is not a universal. \hfill Premise
	\item It is not the case that \textit{resemblance} is not a universal. \hfill Premise
	\item So, it is not the case that two red books $x$ and $y$ resemble each other without there being universals. \hfill 1-4, MT
\end{enumerate}
Russell concludes that we need universals. His argument is quite famous. It is further discussed in recent work by our own Katarina Perovic in Ch 9 of Wishon and Linsky (2015).
\section*{Universals are non-mental and outside space and time}
Russell then further argues that universals are not mental as follows (Ch 9, 97-98):
\begin{enumerate}%[Universals are not mental.] 
	\item If \textit{Edinburgh is north of London} is true regardless of whether we ever thought about it, then the relation \textit{is north of} relates Edinburgh and London regardless of whether we ever thought about it. \hfill Premise
	\item If the relation \textit{is north of} relates Edinburgh and London regardless of whether we ever thought about it, then it is not mental. \hfill Premise
	\item \textit{Edinburgh is north of London} is true regardless of whether we ever thought about it. \hfill Premise
	\item So, the relation \textit{is north of} are not mental. \hfill 1-3, MP
\end{enumerate}
Russell next argues that universals are outside space and time as follows (Ch 9, 98-99):
\begin{enumerate}%[Universals are outside space and time.] 
	\item If the relation \textit{is north of} relates Edinburgh and London without being located between or in them, then it is outside space and time. \hfill Premise
	\item The relation \textit{is north of} relates Edinburgh and London without being located between or in them. \hfill Premise
	\item So, the relation \textit{is north of} is outside space and time. \hfill 1-2, MP
\end{enumerate}
Universals, in contrast with particulars, exist ``nowhere and nowhen" (\pageref{nowhereandnowhen}). 

To mark this difference, Russell introduces some terminology: he says that \textit{particulars exist} concretely, whereas \textit{universals subsist (or have being)} (\pageref{subsist}). In Ch 10, Russell considers how we come to know this ``world of universals" or ``world of being" (\pageref{subsist}).

\hypertarget{chapter-x.-on-our-knowledge-of-universals}{%
\chapter{On Our Knowledge of Universals}\label{chapter-x.-on-our-knowledge-of-universals}}

In regard to one man's knowledge at a given time,
universals, like particulars, may be divided into those known by
acquaintance, those known only by description, and those not known
either by acquaintance or by description.

Let us consider first the knowledge of universals by acquaintance. \label{sensible} It is
obvious, to begin with, that we are acquainted with such universals as
white, red, black, sweet, sour, loud, hard, etc., i.e. with qualities
which are exemplified in sense-data. When we see a white patch, we are
acquainted, in the first instance, with the particular patch; but by
seeing many white patches, we easily learn to abstract the whiteness
which they all have in common, and in learning to do this we are
learning to be acquainted with whiteness. A similar process will make us
acquainted with any other universal of the same sort. Universals of this
sort may be called `sensible qualities'.
\label{abstraction} They can be apprehended with less effort of abstraction than any others,
and they seem less removed from particulars than other universals are.

We come next to relations. \label{spacetime} The easiest relations to apprehend are those
which hold between the different parts of a single complex sense-datum.
For example, I can see at a glance the whole of the page on which I am
writing; thus the whole page is included in one sense-datum. But I
perceive that some parts of the page are to the left of other parts, and
some parts are above other parts. The process of abstraction in this
case seems to proceed somewhat as follows: I see successively a number
of sense-data in which one part is to the left of another; I perceive,
as in the case of different white patches, that all these sense-data
have something in common, and by abstraction I find that what they have
in common is a certain relation between their parts, namely the relation
which I call `being to the left of'. In
this way I become acquainted with the universal relation.

In like manner I become aware of the relation of before and after in
time. Suppose I hear a chime of bells: when the last bell of the chime
sounds, I can retain the whole chime before my mind, and I can perceive
that the earlier bells came before the later ones. Also in memory I
perceive that what I am remembering came before the present time. From
either of these sources I can abstract the universal relation of before
and after, just as I abstracted the universal relation
`being to the left of'. Thus
time-relations, like space-relations, are among those with which we are
acquainted.

\label{resemblance} Another relation with which we become acquainted in much the same way is
resemblance. If I see simultaneously two shades of green, I can see that
they resemble each other; if I also see a shade of red: at the same
time, I can see that the two greens have more resemblance to each other
than either has to the red. In this way I become acquainted with the
universal \emph{resemblance} or \emph{similarity}.

Between universals, as between particulars, there are relations of which
we may be immediately aware. We have just seen that we can perceive that
the resemblance between two shades of green is greater than the
resemblance between a shade of red and a shade of green. Here we are
dealing with a relation, namely `greater
than', between two relations. Our knowledge of such
relations, though it requires more power of abstraction than is required
for perceiving the qualities of sense-data, appears to be equally
immediate, and (at least in some cases) equally indubitable. Thus there
is immediate knowledge concerning universals as well as concerning
sense-data.

Returning now to the problem of \emph{a priori} knowledge, which we left
unsolved when we began the consideration of universals, we find
ourselves in a position to deal with it in a much more satisfactory
manner than was possible before. Let us revert to the proposition
`two and two are four'. It is fairly
obvious, in view of what has been said, that this proposition states a
relation between the universal `two'
and the universal `four'. This suggests a
proposition which we shall now endeavour to establish: namely,
\label{apriori} \emph{All} a priori \emph{knowledge deals exclusively with the relations
of universals}. This proposition is of great importance, and goes a long
way towards solving our previous difficulties concerning \emph{a priori}
knowledge.

The only case in which it might seem, at first sight, as if our
proposition were untrue, is the case in which an \emph{a priori}
proposition states that \emph{all} of one class of particulars belong to
some other class, or (what comes to the same thing) that \emph{all}
particulars having some one property also have some other. In this case
it might seem as though we were dealing with the particulars that have
the property rather than with the property. \label{couples} The proposition
`two and two are four' is really a case
in point, for this may be stated in the form `any two
and any other two are four', or `any
collection formed of two twos is a collection of four'.
If we can show that such statements as this really deal only with
universals, our proposition may be regarded as proved.

\label{discovering} One way of discovering what a proposition deals with is to ask ourselves
what words we must understand---in other words, what objects we must be
acquainted with---in order to see what the proposition means. As soon as
we see what the proposition means, even if we do not yet know whether it
is true or false, it is evident that we must have acquaintance with
whatever is really dealt with by the proposition. By applying this test,
it appears that many propositions which might seem to be concerned with
particulars are really concerned only with universals. In the special
case of `two and two are four', even when
we interpret it as meaning `any collection formed of two
twos is a collection of four', it is plain that we can
understand the proposition, i.e. we can see what it is that it asserts,
as soon as we know what is meant by
`collection' and
`two' and
`four'. It is quite unnecessary to know
all the couples in the world: if it were necessary, obviously we could
never understand the proposition, since the couples are infinitely
numerous and therefore cannot all be known to us. Thus although our
general statement \emph{implies} statements about particular couples,
\emph{as soon as we know that there are such particular couples}, yet it
does not itself assert or imply that there are such particular couples,
and thus fails to make any statement whatever about any actual
particular couple. The statement made is about
`couple', the universal, and not about
this or that couple.

Thus the statement `two and two are
four' deals exclusively with universals, and therefore
may be known by anybody who is acquainted with the universals concerned
and can perceive the relation between them which the statement asserts.
It must be taken as a fact, discovered by reflecting upon our knowledge,
that we have the power of sometimes perceiving such relations between
universals, and therefore of sometimes knowing general \emph{a priori}
propositions such as those of arithmetic and logic. The thing that
seemed mysterious, when we formerly considered such knowledge, was that
it seemed to anticipate and control experience. This, however, we can
now see to have been an error. \emph{No} fact concerning anything
capable of being experienced can be known independently of experience.
We know \emph{a priori} that two things and two other things together
make four things, but we do \emph{not} know \emph{a priori} that if
Brown and Jones are two, and Robinson and Smith are two, then Brown and
Jones and Robinson and Smith are four. The reason is that this
proposition cannot be understood at all unless we know that there are
such people as Brown and Jones and Robinson and Smith, and this we can
only know by experience. Hence, although our general proposition is
\emph{a priori}, all its applications to actual particulars involve
experience and therefore contain an empirical element. In this way what
seemed mysterious in our \emph{a priori} knowledge is seen to have been
based upon an error.

It will serve to make the point clearer if we contrast our genuine
\emph{a priori} judgement with an empirical generalization, such as
`all men are mortals'. \label{mortal} Here as before, we
can \emph{understand} what the proposition means as soon as we
understand the universals involved, namely \emph{man} and \emph{mortal}.
It is obviously unnecessary to have an individual acquaintance with the
whole human race in order to understand what our proposition means. Thus
the difference between an \emph{a priori} general proposition and an
empirical generalization does not come in the \emph{meaning} of the
proposition; it comes in the nature of the \emph{evidence} for it. In
the empirical case, the evidence consists in the particular instances.
We believe that all men are mortal because we know that there are
innumerable instances of men dying, and no instances of their living
beyond a certain age. We do not believe it because we see a connexion
between the universal \emph{man} and the universal \emph{mortal}. It is
true that if physiology can prove, assuming the general laws that govern
living bodies, that no living organism can last for ever, that gives a
connexion between \emph{man} and \emph{mortality} which would enable us
to assert our proposition without appealing to the special evidence of
\emph{men} dying. But that only means that our generalization has been
subsumed under a wider generalization, for which the evidence is still
of the same kind, though more extensive. The progress of science is
constantly producing such subsumptions, and therefore giving a
constantly wider inductive basis for scientific generalizations. But
although this gives a greater \emph{degree} of certainty, it does not
give a different \emph{kind}: the ultimate ground remains inductive,
i.e. derived from instances, and not an \emph{a priori} connexion of
universals such as we have in logic and arithmetic. \label{logic}

Two opposite points are to be observed concerning \emph{a priori}
general propositions. The first is that, if many particular instances
are known, our general proposition may be arrived at in the first
instance by induction, and the connexion of universals may be only
subsequently perceived. For example, it is known that if we draw
perpendiculars to the sides of a triangle from the opposite angles, all
three perpendiculars meet in a point. It would be quite possible to be
first led to this proposition by actually drawing perpendiculars in many
cases, and finding that they always met in a point; this experience
might lead us to look for the general proof and find it. Such cases are
common in the experience of every mathematician.

The other point is more interesting, and of more philosophical
importance. \label{single} It is, that we may sometimes know a general proposition in
cases where we do not know a single instance of it. Take such a case as
the following: We know that any two numbers can be multiplied together,
and will give a third called their \emph{product}. We know that all
pairs of integers the product of which is less than 100 have been
actually multiplied together, and the value of the product recorded in
the multiplication table. But we also know that the number of integers
is infinite, and that only a finite number of pairs of integers ever
have been or ever will be thought of by human beings. Hence it follows
that there are pairs of integers which never have been and never will be
thought of by human beings, and that all of them deal with integers the
product of which is over 100. Hence we arrive at the proposition:
`All products of two integers, which never have been and
never will be thought of by any human being, are over
100.' Here is a general proposition of which the truth
is undeniable, and yet, from the very nature of the case, we can never
give an instance; because any two numbers we may think of are excluded
by the terms of the proposition.

This possibility, of knowledge of general propositions of which no
instance can be given, is often denied, because it is not perceived that
the knowledge of such propositions only requires a knowledge of the
relations of universals, and does not require any knowledge of instances
of the universals in question. Yet the knowledge of such general
propositions is quite vital to a great deal of what is generally
admitted to be known. For example, we saw, in our early chapters, that
knowledge of physical objects, as opposed to sense-data, is only
obtained by an inference, and that they are not things with which we are
acquainted. Hence we can never know any proposition of the form
`this is a physical object', where
`this' is something immediately known.
It follows that all our knowledge concerning physical objects is such
that no actual instance can be given. We can give instances of the
associated sense-data, but we cannot give instances of the actual
physical objects. Hence our knowledge as to physical objects depends
throughout upon this possibility of general knowledge where no instance
can be given. And the same applies to our knowledge of other
people's minds, or of any other class of things of which
no instance is known to us by acquaintance.

We may now take a survey of the sources of our knowledge, as they have
appeared in the course of our analysis. We have first to distinguish
knowledge of things and knowledge of truths. In each there are two
kinds, one immediate and one derivative. \label{immediate} Our immediate knowledge of
things, which we called \emph{acquaintance}, consists of two sorts,
according as the things known are particulars or universals. Among
particulars, we have acquaintance with sense-data and (probably) with
ourselves. Among universals, there seems to be no principle by which we
can decide which can be known by acquaintance, but it is clear that
among those that can be so known are sensible qualities, relations of
space and time, similarity, and certain abstract logical universals. Our
derivative knowledge of things, which we call knowledge by
\emph{description}, always involves both acquaintance with something and
knowledge of truths. Our immediate knowledge of \emph{truths} may be
called \emph{intuitive} knowledge, and the truths so known may be called
\emph{self-evident} truths. Among such truths are included those which
merely state what is given in sense, and also certain abstract logical
and arithmetical principles, and (though with less certainty) some
ethical propositions. Our \emph{derivative} knowledge of truths consists
of everything that we can deduce from self-evident truths by the use of
self-evident principles of deduction.

If the above account is correct, all our knowledge of truths depends
upon our intuitive knowledge. It therefore becomes important to consider
the nature and scope of intuitive knowledge, in much the same way as, at
an earlier stage, we considered the nature and scope of knowledge by
acquaintance. But knowledge of truths raises a further problem, which
does not arise in regard to knowledge of things, namely the problem of
\emph{error}. Some of our beliefs turn out to be erroneous, and
therefore it becomes necessary to consider how, if at all, we can
distinguish knowledge from error. This problem does not arise with
regard to knowledge by acquaintance, for, whatever may be the object of
acquaintance, even in dreams and hallucinations, there is no error
involved so long as we do not go beyond the immediate object: error can
only arise when we regard the immediate object, i.e. the sense-datum, as
the mark of some physical object. Thus the problems connected with
knowledge of truths are more difficult than those connected with
knowledge of things. As the first of the problems connected with
knowledge of truths, let us examine the nature and scope of our
intuitive judgements.

\protect\hypertarget{link2HCH0011}{}{}

\pagebreak
\section{Notes for the Student}
\markboth{CHAPTER 10 NOTES}{CHAPTER 10 NOTES}
As we saw in Ch 9, Russell believes that there are universals, that they are non-mental, and that do not exist in space and time. Rather, they \textit{subsist} (or \textit{have being}). But how do we know such entities? This chapter is concerned to explain that.

We have perceptions of sensible qualities all the time: colors, sounds, tastes, smells, and so on. Russell wants to claim that, by a minimal effort of \textit{abstraction}, we come to recognize sensible universals (\pageref{sensible}). Moreover, we can also become aware of \textit{relations between data of sense and memory}, like spatial and temporal relations (\pageref{spacetime}). We become aware of relations of \textit{resemblance} in much the same way (\pageref{resemblance}). 

Notice that Russell is claiming that we become \textit{aware} of these universals by abstraction. Awareness, or acquaintance, is a kind of knowledge of things. Russell is saying that we have acquaintance with abstract universals much as we have acquaintance with concrete particulars. It is still a kind of direct awareness of an object. The only difference is that awareness of universals ``requires more power of abstraction," especially as the universals grow more removed from our sensory experience (\pageref{abstraction}). 

Berkeley and Hume were led to deny that there are universals precisely because they doubted whether we could enjoy awareness of such things. What do you think?
\subsection*{Russell's Account of \textit{a priori} Knowledge}
Finally, we arrive at Russell's account of \textit{a priori} knowledge:
\begin{quote}
	\textit{All} a priori \textit{knowledge deals exclusively with the relations of universals.} (\pageref{apriori})
\end{quote}
Let us consider what this means. Universals can relate particulars: the universal \textit{is north of} relates various cities, like Edinburgh and London. But universals can also related universals, as in mathematical claims: the universals \textit{two} and \textit{four} are related by the universal \textit{equals} in the claim \textit{a pair of two couples equals four things} (\pageref{couples}). Likewise, universals are related by universals in color-comparison claims: the universals \textit{redness} and \textit{blackness} are related by \textit{is lighter in color than} in the claim \textit{whatever has redness is lighter in color than whatever has blackness}. It is the universals that are related because we did not relate any particular entities: no particulars were involved in our claims at all. 

This is what Russell is getting at when he says:
\begin{quote}
	One way of discovering what a proposition deals with is to ask ourselves what words we must understand---in other words, what objects we must be acquainted with---in order to see what the proposition means. (\pageref{discovering})
\end{quote} 
He is giving you a test for whether a claim is purely \textit{a priori}, involving only relations between universals, or whether it depends on experience. If we need acquaintance with particulars, then the claim is not purely \textit{a priori}, but relies on \textit{a posteriori} knowledge, like \textit{all humans are mortal} (\pageref{mortal}). Otherwise, it is pure \textit{a priori}, as logic and math are (\pageref{logic}).

Russell adds that we can know a general claim without being able to know a single case of it. He shows this with a clever example: `all products of two integers, which never have been and never will be thought of by any human being, are over 100.' (\pageref{single})

In the next chapter, we will discuss immediate knowledge of truths, or \textit{intuitive knowledge}.

\hypertarget{chapter-xi.-on-intuitive-knowledge}{%
\chapter{On Intuitive Knowledge}\label{chapter-xi.-on-intuitive-knowledge}}

There is a common impression that everything that we believe ought to be
capable of proof, or at least of being shown to be highly probable. It
is felt by many that a belief for which no reason can be given is an
unreasonable belief. In the main, this view is just. Almost all our
common beliefs are either inferred, or capable of being inferred, from
other beliefs which may be regarded as giving the reason for them. As a
rule, the reason has been forgotten, or has even never been consciously
present to our minds. Few of us ever ask ourselves, for example, what
reason there is to suppose the food we are just going to eat will not
turn out to be poison. Yet we feel, when challenged, that a perfectly
good reason could be found, even if we are not ready with it at the
moment. And in this belief we are usually justified.

\label{socrates} But let us imagine some insistent Socrates, who, whatever reason we give
him, continues to demand a reason for the reason. We must sooner or
later, and probably before very long, be driven to a point where we
cannot find any further reason, and where it becomes almost certain that
no further reason is even theoretically discoverable. Starting with the
common beliefs of daily life, we can be driven back from point to point,
until we come to some general principle, or some instance of a general
principle, which seems luminously evident, and is not itself capable of
being deduced from anything more evident. In most questions of daily
life, such as whether our food is likely to be nourishing and not
poisonous, we shall be driven back to the inductive principle, which we
discussed in Chapter VI. But beyond that, there seems to be no further
regress. The principle itself is constantly used in our reasoning,
sometimes consciously, sometimes unconsciously; but there is no
reasoning which, starting from some simpler self-evident principle,
leads us to the principle of induction as its conclusion. And the same
holds for other logical principles. Their truth is evident to us, and we
employ them in constructing demonstrations; but they themselves, or at
least some of them, are incapable of demonstration.

Self-evidence, however, is not confined to those among general
principles which are incapable of proof. When a certain number of
logical principles have been admitted, the rest can be deduced from
them; but the propositions deduced are often just as self-evident as
those that were assumed without proof. All arithmetic, moreover, can be
deduced from the general principles of logic, yet the simple
propositions of arithmetic, such as `two and two are
four', are just as self-evident as the principles of
logic.

\label{selfevident} It would seem, also, though this is more disputable, that there are some
self-evident ethical principles, such as `we ought to
pursue what is good'.

It should be observed that, in all cases of general principles,
particular instances, dealing with familiar things, are more evident
than the general principle. For example, the law of contradiction states
that nothing can both have a certain property and not have it. This is
evident as soon as it is understood, but it is not so evident as that a
particular rose which we see cannot be both red and not red. (It is of
course possible that parts of the rose may be red and parts not red, or
that the rose may be of a shade of pink which we hardly know whether to
call red or not; but in the former case it is plain that the rose as a
whole is not red, while in the latter case the answer is theoretically
definite as soon as we have decided on a precise definition of
`red'.) It is usually through particular
instances that we come to be able to see the general principle. Only
those who are practised in dealing with abstractions can readily grasp a
general principle without the help of instances.

In addition to general principles, the other kind of self-evident truths
are those immediately derived from sensation. We will call such truths
`truths of perception', and the
judgements expressing them we will call `judgements of
perception'. But here a certain amount of care is
required in getting at the precise nature of the truths that are
self-evident. The actual sense-data are neither true nor false. A
particular patch of colour which I see, for example, simply exists: it
is not the sort of thing that is true or false. It is true that there is
such a patch, true that it has a certain shape and degree of brightness,
true that it is surrounded by certain other colours. But the patch
itself, like everything else in the world of sense, is of a radically
different kind from the things that are true or false, and therefore
cannot properly be said to be \emph{true}. Thus whatever self-evident
truths may be obtained from our senses must be different from the
sense-data from which they are obtained.

It would seem that there are two kinds of self-evident truths of
perception, though perhaps in the last analysis the two kinds may
coalesce. First, there is the kind which simply asserts the
\emph{existence} of the sense-datum, without in any way analysing it. We
see a patch of red, and we judge `there is such-and-such
a patch of red', or more strictly `there
is that'; this is one kind of intuitive judgement of
perception. The other kind arises when the object of sense is complex,
and we subject it to some degree of analysis. If, for instance, we see a
\emph{round} patch of red, we may judge `that patch of
red is round'. This is again a judgement of perception,
but it differs from our previous kind. In our present kind we have a
single sense-datum which has both colour and shape: the colour is red
and the shape is round. Our judgement analyses the datum into colour and
shape, and then recombines them by stating that the red colour is round
in shape. Another example of this kind of judgement is
`this is to the right of that', where
`this' and
`that' are seen simultaneously. In this
kind of judgement the sense-datum contains constituents which have some
relation to each other, and the judgement asserts that these
constituents have this relation.

Another class of intuitive judgements, analogous to those of sense and
yet quite distinct from them, are judgements of \emph{memory}. There is
some danger of confusion as to the nature of memory, owing to the fact
that memory of an object is apt to be accompanied by an image of the
object, and yet the image cannot be what constitutes memory. This is
easily seen by merely noticing that the image is in the present, whereas
what is remembered is known to be in the past. Moreover, we are
certainly able to some extent to compare our image with the object
remembered, so that we often know, within somewhat wide limits, how far
our image is accurate; but this would be impossible, unless the object,
as opposed to the image, were in some way before the mind. Thus the
essence of memory is not constituted by the image, but by having
immediately before the mind an object which is recognized as past. But
for the fact of memory in this sense, we should not know that there ever
was a past at all, nor should we be able to understand the word
`past', any more than a man born blind
can understand the word `light'. Thus
there must be intuitive judgements of memory, and it is upon them,
ultimately, that all our knowledge of the past depends.

The case of memory, however, raises a difficulty, for it is notoriously
fallacious, and thus throws doubt on the trustworthiness of intuitive
judgements in general. This difficulty is no light one. But let us first
narrow its scope as far as possible. Broadly speaking, memory is
trustworthy in proportion to the vividness of the experience and to its
nearness in time. If the house next door was struck by lightning half a
minute ago, my memory of what I saw and heard will be so reliable that
it would be preposterous to doubt whether there had been a flash at all.
And the same applies to less vivid experiences, so long as they are
recent. I am absolutely certain that half a minute ago I was sitting in
the same chair in which I am sitting now. Going backward over the day, I
find things of which I am quite certain, other things of which I am
almost certain, other things of which I can become certain by thought
and by calling up attendant circumstances, and some things of which I am
by no means certain. I am quite certain that I ate my breakfast this
morning, but if I were as indifferent to my breakfast as a philosopher
should be, I should be doubtful. As to the conversation at breakfast, I
can recall some of it easily, some with an effort, some only with a
large element of doubt, and some not at all. Thus there is a continual
gradation in the degree of self-evidence of what I remember, and a
corresponding gradation in the trustworthiness of my memory. \label{uncertain}

Thus the first answer to the difficulty of fallacious memory is to say
that memory has degrees of self-evidence, and that these correspond to
the degrees of its trustworthiness, reaching a limit of perfect
self-evidence and perfect trustworthiness in our memory of events which
are recent and vivid.

It would seem, however, that there are cases of very firm belief in a
memory which is wholly false. It is probable that, in these cases, what
is really remembered, in the sense of being immediately before the mind,
is something other than what is falsely believed in, though something
generally associated with it. George IV is said to have at last believed
that he was at the battle of Waterloo, because he had so often said that
he was. In this case, what was immediately remembered was his repeated
assertion; the belief in what he was asserting (if it existed) would be
produced by association with the remembered assertion, and would
therefore not be a genuine case of memory. It would seem that cases of
fallacious memory can probably all be dealt with in this way, i.e. they
can be shown to be not cases of memory in the strict sense at all.

\label{degrees} One important point about self-evidence is made clear by the case of
memory, and that is, that self-evidence has degrees: it is not a quality
which is simply present or absent, but a quality which may be more or
less present, in gradations ranging from absolute certainty down to an
almost imperceptible faintness. \label{hierarchy} Truths of perception and some of the
principles of logic have the very highest degree of self-evidence;
truths of immediate memory have an almost equally high degree. The
inductive principle has less self-evidence than some of the other
principles of logic, such as `what follows from a true
premiss must be true'. Memories have a diminishing
self-evidence as they become remoter and fainter; the truths of logic
and mathematics have (broadly speaking) less self-evidence as they
become more complicated. Judgements of intrinsic ethical or aesthetic
value are apt to have some self-evidence, but not much.

Degrees of self-evidence are important in the theory of knowledge,
since, if propositions may (as seems likely) have some degree of
self-evidence without being true, it will not be necessary to abandon
all connexion between self-evidence and truth, but merely to say that,
where there is a conflict, the more self-evident proposition is to be
retained and the less self-evident rejected.

It seems, however, highly probable that two different notions are
combined in `self-evidence' as above
explained; that one of them, which corresponds to the highest degree of
self-evidence, is really an infallible guarantee of truth, while the
other, which corresponds to all the other degrees, does not give an
infallible guarantee, but only a greater or less presumption. This,
however, is only a suggestion, which we cannot as yet develop further.
After we have dealt with the nature of truth, we shall return to the
subject of self-evidence, in connexion with the distinction between
knowledge and error.

\protect\hypertarget{link2HCH0012}{}{}

\pagebreak
\section{Notes for the Student}
\markboth{CHAPTER 11 NOTES}{CHAPTER 11 NOTES}
As we saw in Ch 5, Russell is a \textit{foundationalist} about justification; he reaffirms this view in discussing ``some insistent Socrates" (\pageref{socrates}). Foundationalists hold that all justification terminates in beliefs that are not inferred from anything, but are justified on the basis of evidence or self-evidence. For example, the belief that there is a table in this room is justified on evidence. A belief in the law of contradiction is justified on self-evidence.

Russell holds that self-evidence comes in degrees. So self-evident is not equivalent to \textit{certain}. Many uncertain beliefs will be self-evident in Russell's sense, like the belief in the principle of induction, self-evident beliefs in ethics, or beliefs based on memory (\pageref{uncertain}). But all these beliefs will be what Russell calls \textit{intuitive beliefs} (or \textit{judgments}).

The hierarchy of self-evident beliefs that Russell gives is as follows (\pageref{hierarchy}):
\begin{enumerate}
	\item Beliefs about current perceptions and principles of logic
	\item Beliefs about immediate memory
	\item Belief in the principle of induction
	\item Beliefs about less immediate memories and complicated logic and math
	\item Beliefs about intrinsic ethical and aesthetic value
\end{enumerate}
What explains this hierarchy? Self-evidence, after all, sounds like the sort of thing that a claim either has or lacks. And we tend to identify self-evident claims as certain. Russell explicitly rejects this identification. So what is going on here?

We will see a bit more fully what self-evidence comes to in Ch 13. It is clear from Ch 11 that it is supposed to be the `rock-bottom' of justification. Self-evident beliefs ``in the first and absolute sense" are beliefs that are justified by what we are acquainted with: a self-evident belief, in this high-certainty sense, is a belief such that we have acquaintance with the fact that makes it true (\pageref{selfevident}).

Here is a surprising consequence of what Russell earlier said about the privacy of sense-data to one person (\pageref{private}). Because facts involving particular things all involve things that are private---like sense-data or mental-data---self-evident claims of this sort are only evident one the basis of sense-data available to at most one person at a time. In contrast, facts involving universals may be known to multiple subjects, so self-evident \emph{a priori} claims can be known on the basis of data available to multiple people at once.

Here is an exercise for you: give an example of each kind of self-evident truth---an example of a truth which only you can know, and another which multiple people can know. This gives you a sense of how Russell views self-evidence: it sort of tracks our field of acquaintance, and as the things involved in the belief get further from our immediate acquaintance (as in memory) or are less obviously part of our field of acquaintance (as ethical properties are), the self-evidence of the belief decreases.

\hypertarget{chapter-xii.-truth-and-falsehood}{%
\chapter{Truth and Falsehood}\label{chapter-xii.-truth-and-falsehood}}
Our knowledge of truths, unlike our knowledge of things, has an
opposite, namely \emph{error}. So far as things are concerned, we may
know them or not know them, but there is no positive state of mind which
can be described as erroneous knowledge of things, so long, at any rate,
as we confine ourselves to knowledge by acquaintance. Whatever we are
acquainted with must be something; we may draw wrong inferences from our
acquaintance, but the acquaintance itself cannot be deceptive. Thus
there is no dualism as regards acquaintance. But as regards knowledge of
truths, there is a dualism. We may believe what is false as well as what
is true. We know that on very many subjects different people hold
different and incompatible opinions: hence some beliefs must be
erroneous. Since erroneous beliefs are often held just as strongly as
true beliefs, it becomes a difficult question how they are to be
distinguished from true beliefs. How are we to know, in a given case,
that our belief is not erroneous? This is a question of the very
greatest difficulty, to which no completely satisfactory answer is
possible. There is, however, a preliminary question which is rather less
difficult, and that is: What do we \emph{mean} by truth and falsehood?
It is this preliminary question which is to be considered in this
chapter. In this chapter we are not asking how we can know whether a
belief is true or false: we are asking what is meant by the question
whether a belief is true or false. It is to be hoped that a clear answer
to this question may help us to obtain an answer to the question what
beliefs are true, but for the present we ask only `What is truth?' and 
`What is falsehood?' not `What beliefs are true?' and 
`What beliefs are false?' It is very important to keep these different
questions entirely separate, since any confusion between them is 
sure to produce an answer which is not really applicable to either.

There are three points to observe in the attempt to discover the nature
of truth, three requisites which any theory must fulfil.
\begin{enumerate}
	\item[(1)] Our theory of truth must be such as to admit of its opposite, falsehood. 
	A good many philosophers have failed adequately to satisfy this condition: 
	they have constructed theories according to which all our thinking ought to 
	have been true, and have then had the greatest difficulty in finding a place 
	for falsehood. In this respect our theory of belief must differ from our theory 
	of acquaintance, since in the case of acquaintance it was not necessary to take 
	account of any opposite. 
	\item[(2)] It seems fairly evident that if there were no beliefs there could be no 
	falsehood, and no truth either, in the sense in which truth is correlative to 
	falsehood. If we imagine a world of mere matter, there would be no room for 
	falsehood in such a world, and although it would contain what may be called 
	`facts', it would not contain any truths, in the sense in which truths are things 
	of the same kind as falsehoods. In fact, truth and falsehood are properties of 
	beliefs and statements: hence a world of mere matter, since it would contain 
	no beliefs or statements, would also contain no truth or falsehood.
	\item[(3)] But, as against what we have just said, it is to be observed that the truth 
	or falsehood of a belief always depends upon something which lies outside the 
	belief itself. If I believe that Charles I died on the scaffold, I believe truly, not 
	because of any intrinsic quality of my belief, which could be discovered by 
	merely examining the belief, but because of an historical event which happened 
	two and a half centuries ago. If I believe that Charles I died in his bed, I believe 
	falsely: no degree of vividness in my belief, or of care in arriving at it, prevents 
	it from being false, again because of what happened long ago, and not because 
	of any intrinsic property of my belief. Hence, although truth and falsehood are 
	properties of beliefs, they are properties dependent upon the relations of the 
	beliefs to other things, not upon any internal quality of the beliefs.
\end{enumerate}
The third of the above requisites leads us to adopt the view---which has
on the whole been commonest among philosophers---that truth consists in
some form of correspondence between belief and fact. It is, however, by
no means an easy matter to discover a form of correspondence to which
there are no irrefutable objections. By this partly---and partly by the
feeling that, if truth consists in a correspondence of thought with
something outside thought, thought can never know when truth has been
attained---many philosophers have been led to try to find some
definition of truth which shall not consist in relation to something
wholly outside belief. The most important attempt at a definition of
this sort is the theory that truth consists in \emph{coherence}. \label{coherence} It is
said that the mark of falsehood is failure to cohere in the body of our
beliefs, and that it is the essence of a truth to form part of the
completely rounded system which is The Truth.

There is, however, a great difficulty in this view, or rather two great
difficulties. \label{multiple} The first is that there is no reason to suppose that only
\emph{one} coherent body of beliefs is possible. It may be that, with
sufficient imagination, a novelist might invent a past for the world
that would perfectly fit on to what we know, and yet be quite different
from the real past. In more scientific matters, it is certain that there
are often two or more hypotheses which account for all the known facts
on some subject, and although, in such cases, men of science endeavour
to find facts which will rule out all the hypotheses except one, there
is no reason why they should always succeed.

In philosophy, again, it seems not uncommon for two rival hypotheses to
be both able to account for all the facts. Thus, for example, it is
possible that life is one long dream, and that the outer world has only
that degree of reality that the objects of dreams have; but although
such a view does not seem inconsistent with known facts, there is no
reason to prefer it to the common-sense view, according to which other
people and things do really exist. Thus coherence as the definition of
truth fails because there is no proof that there can be only one
coherent system.

The other objection to this definition of truth is that it assumes the
meaning of `coherence' known, whereas,
in fact, `coherence' presupposes the
truth of the laws of logic. \label{laws} Two propositions are coherent when both may
be true, and are incoherent when one at least must be false. Now in
order to know whether two propositions can both be true, we must know
such truths as the law of contradiction. For example, the two
propositions, `this tree is a beech'
and `this tree is not a beech', are not
coherent, because of the law of contradiction. But if the law of
contradiction itself were subjected to the test of coherence, we should
find that, if we choose to suppose it false, nothing will any longer be
incoherent with anything else. Thus the laws of logic supply the
skeleton or framework within which the test of coherence applies, and
they themselves cannot be established by this test.

For the above two reasons, coherence cannot be accepted as giving the
\emph{meaning} of truth, though it is often a most important \emph{test}
of truth after a certain amount of truth has become known.

Hence we are driven back to \emph{correspondence with fact} as
constituting the nature of truth. It remains to define precisely what we
mean by `fact', and what is the nature of
the correspondence which must subsist between belief and fact, in order
that belief may be true.

In accordance with our three requisites, we have to seek a theory of
truth which (1) allows truth to have an opposite, namely falsehood, (2)
makes truth a property of beliefs, but (3) makes it a property wholly
dependent upon the relation of the beliefs to outside things.

The necessity of allowing for falsehood makes it impossible to regard
belief as a relation of the mind to a single object, which could be said
to be what is believed. \label{entities} If belief were so regarded, we should find that,
like acquaintance, it would not admit of the opposition of truth and
falsehood, but would have to be always true. This may be made clear by
examples. Othello believes falsely that Desdemona loves Cassio. We
cannot say that this belief consists in a relation to a single object,
`Desdemona's love for
Cassio', for if there were such an object, the belief
would be true. There is in fact no such object, and therefore Othello
cannot have any relation to such an object. Hence his belief cannot
possibly consist in a relation to this object.

It might be said that his belief is a relation to a different object,
namely `that Desdemona loves Cassio'; but
it is almost as difficult to suppose that there is such an object as
this, when Desdemona does not love Cassio, as it was to suppose that
there is `Desdemona's love for
Cassio'. Hence it will be better to seek for a theory of
belief which does not make it consist in a relation of the mind to a
single object.

It is common to think of relations as though they always held between
two terms, but in fact this is not always the case. Some relations
demand three terms, some four, and so on. Take, for instance, the
relation `between'. So long as only two
terms come in, the relation `between'
is impossible: three terms are the smallest number that render it
possible. York is between London and Edinburgh; but if London and
Edinburgh were the only places in the world, there could be nothing
which was between one place and another. Similarly \emph{jealousy}
requires three people: there can be no such relation that does not
involve three at least. Such a proposition as `A wishes
B to promote C's marriage with D'
involves a relation of four terms; that is to say, A and B and C and D
all come in, and the relation involved cannot be expressed otherwise
than in a form involving all four. Instances might be multiplied
indefinitely, but enough has been said to show that there are relations
which require more than two terms before they can occur.

The relation involved in \emph{judging} or \emph{believing} must, if
falsehood is to be duly allowed for, be taken to be a relation between
several terms, not between two. When Othello believes that Desdemona
loves Cassio, he must not have before his mind a single object,
`Desdemona's love for
Cassio', or `that Desdemona loves Cassio
', for that would require that there should be objective
falsehoods, which subsist independently of any minds; and this, though
not logically refutable, is a theory to be avoided if possible. Thus it
is easier to account for falsehood if we take judgement to be a relation
in which the mind and the various objects concerned all occur severally;
that is to say, Desdemona and loving and Cassio must all be terms in the
relation which subsists when Othello believes that Desdemona loves
Cassio. This relation, therefore, is a relation of four terms, since
Othello also is one of the terms of the relation. When we say that it is
a relation of four terms, we do not mean that Othello has a certain
relation to Desdemona, and has the same relation to loving and also to
Cassio. This may be true of some other relation than believing; but
believing, plainly, is not a relation which Othello has to \emph{each}
of the three terms concerned, but to \emph{all} of them together: there
is only one example of the relation of believing involved, but this one
example knits together four terms. Thus the actual occurrence, at the
moment when Othello is entertaining his belief, is that the relation
called `believing' is knitting together
into one complex whole the four terms Othello, Desdemona, loving, and
Cassio. \label{weaving} What is called belief or judgement is nothing but this relation
of believing or judging, which relates a mind to several things other
than itself. An \emph{act} of belief or of judgement is the occurrence
between certain terms at some particular time, of the relation of
believing or judging.

We are now in a position to understand what it is that distinguishes a
true judgement from a false one. For this purpose we will adopt certain
definitions. In every act of judgement there is a mind which judges, and
there are terms concerning which it judges. We will call the mind the
\emph{subject} in the judgement, and the remaining terms the
\emph{objects}. Thus, when Othello judges that Desdemona loves Cassio,
Othello is the subject, while the objects are Desdemona and loving and
Cassio. The subject and the objects together are called the
\emph{constituents} of the judgement. It will be observed that the
relation of judging has what is called a
`sense' or
`direction'. We may say, metaphorically,
that it puts its objects in a certain \emph{order}, which we may
indicate by means of the order of the words in the sentence. (In an
inflected language, the same thing will be indicated by inflections,
e.g. by the difference between nominative and accusative.)
Othello's judgement that Cassio loves Desdemona differs
from his judgement that Desdemona loves Cassio, in spite of the fact
that it consists of the same constituents, because the relation of
judging places the constituents in a different order in the two cases.
Similarly, if Cassio judges that Desdemona loves Othello, the
constituents of the judgement are still the same, but their order is
different. This property of having a
`sense' or
`direction' is one which the relation
of judging shares with all other relations. The
`sense' of relations is the ultimate
source of order and series and a host of mathematical concepts; but we
need not concern ourselves further with this aspect.

We spoke of the relation called
`judging' or
`believing' as knitting together into
one complex whole the subject and the objects. In this respect, judging
is exactly like every other relation. Whenever a relation holds between
two or more terms, it unites the terms into a complex whole. If Othello
loves Desdemona, there is such a complex whole as
`Othello's love for
Desdemona'. The terms united by the relation may be
themselves complex, or may be simple, but the whole which results from
their being united must be complex. Wherever there is a relation which
relates certain terms, there is a complex object formed of the union of
those terms; and conversely, wherever there is a complex object, there
is a relation which relates its constituents. When an act of believing
occurs, there is a complex, in which
`believing' is the uniting relation,
and subject and objects are arranged in a certain order by the
`sense' of the relation of believing.
Among the objects, as we saw in considering `Othello
believes that Desdemona loves Cassio', one must be a
relation---in this instance, the relation
`loving'. But this relation, as it occurs
in the act of believing, is not the relation which creates the unity of
the complex whole consisting of the subject and the objects. The
relation `loving', as it occurs in the
act of believing, is one of the objects---it is a brick in the
structure, not the cement. The cement is the relation
`believing'. When the belief is
\emph{true}, there is another complex unity, in which the relation which
was one of the objects of the belief relates the other objects. Thus,
e.g., if Othello believes \emph{truly} that Desdemona loves Cassio, then
there is a complex unity, `Desdemona's
love for Cassio', which is composed exclusively of the
\emph{objects} of the belief, in the same order as they had in the
belief, with the relation which was one of the objects occurring now as
the cement that binds together the other objects of the belief. On the
other hand, when a belief is \emph{false}, there is no such complex
unity composed only of the objects of the belief. If Othello believes
\emph{falsely} that Desdemona loves Cassio, then there is no such
complex unity as `Desdemona's love for
Cassio'.

Thus a belief is \emph{true} when it \emph{corresponds} to a certain
associated complex, and \emph{false} when it does not. \label{corresponds} Assuming, for the
sake of definiteness, that the objects of the belief are two terms and a
relation, the terms being put in a certain order by the
`sense' of the believing, then if the
two terms in that order are united by the relation into a complex, the
belief is true; if not, it is false. This constitutes the definition of
truth and falsehood that we were in search of. Judging or believing is a
certain complex unity of which a mind is a constituent; if the remaining
constituents, taken in the order which they have in the belief, form a
complex unity, then the belief is true; if not, it is false.

Thus although truth and falsehood are properties of beliefs, yet they
are in a sense extrinsic properties, for the condition of the truth of a
belief is something not involving beliefs, or (in general) any mind at
all, but only the \emph{objects} of the belief. A mind, which believes,
believes truly when there is a \emph{corresponding} complex not
involving the mind, but only its objects. This correspondence ensures
truth, and its absence entails falsehood. Hence we account
simultaneously for the two facts that beliefs (a) depend on minds for
their \emph{existence}, (b) do not depend on minds for their
\emph{truth}.

We may restate our theory as follows: If we take such a belief as
`Othello believes that Desdemona loves
Cassio', we will call Desdemona and Cassio the
\emph{object-terms}, and loving the \emph{object-relation}. If there is
a complex unity `Desdemona's love for
Cassio', consisting of the object-terms related by the
object-relation in the same order as they have in the belief, then this
complex unity is called the \emph{fact corresponding to the belief}.
Thus a belief is true when there is a corresponding fact, and is false
when there is no corresponding fact.

It will be seen that minds do not \emph{create} truth or falsehood. They
create beliefs, but when once the beliefs are created, the mind cannot
make them true or false, except in the special case where they concern
future things which are within the power of the person believing, such
as catching trains. What makes a belief true is a \emph{fact}, and this
fact does not (except in exceptional cases) in any way involve the mind
of the person who has the belief.

Having now decided what we \emph{mean} by truth and falsehood, we have
next to consider what ways there are of knowing whether this or that
belief is true or false. This consideration will occupy the next
chapter.

\protect\hypertarget{link2HCH0013}{}{}

\pagebreak
\section{Notes for the Student}
\markboth{CHAPTER 12 NOTES}{CHAPTER 12 NOTES}
	\subsection*{Defining Truth}
Defining \textit{what it is for a belief to be true} is difficult because beliefs can be true or false. You just \textit{have} acquaintance or not. But you can have beliefs that are \textit{true} or that are \textit{false}. So how do we define the truth of a belief?
\subsection*{Coherence Theory of Truth}
The first option that Russell considers is the \textit{coherence theory of truth} according to which truth is what a belief has when it \textit{coheres} with all our other beliefs (\pageref{coherence}). On this view, true beliefs cohere with \textit{other beliefs} instead of corresponding to \textit{non-belief stuff}.

Russell raises two objections to this view. First, there are many collections of coherent beliefs, but the coherence theory gives us no way to choose between them (\pageref{multiple}). 

Second, coherence theorists cannot explain what it means for beliefs to \textit{cohere} because the laws of logic are what determines whether beliefs are coherent (\pageref{laws}). But then the laws of logic have to be accepted independently of their coherence with other beliefs. The alternative is to say that the laws of logic are coherent, so that the laws of logic show that the laws of logic are coherent. This is a viciously circular account of \textit{coherence}. 
\subsection*{Correspondence Theory of Truth}
Russell accepts a \textit{correspondence theory of truth} according to which truth is what a belief has when it corresponds to a fact. He spends multiple pages in explaining this view fully.

The first thing he notes is that what beliefs correspond to are \textit{complex} in the sense that what beliefs correspond to are composed of multiple entities (\pageref{entities}). The fact of \textit{Landon being human} is a complex involving two constituents---me, and the property of being human. The fact of \textit{Landon being the brother of George} is a complex involving three constituents---me, George, and the relation of brotherhood. Beliefs do not correspond to \textit{objects} by themselves. There is no belief that corresponds to just me, for example.

Second, beliefs are woven together by us using our minds and judgments (\pageref{weaving}). This is how a complex belief can exist without there existing corresponding fact.

So a belief or judgment is a complex thing that we use our minds produce. Beliefs and judgments involve multiple constituents. These beliefs will exist independently of whether they are true or false. A belief is true if a fact exists that corresponds to it (\pageref{corresponds}).

Next we explain how the correspondence theory meets Russell's three criteria for truth:
\begin{enumerate}
	\item \textbf{Truth has to admit of its opposite, falsity.} \\ The correspondence theory meets this criterion because when a belief does not correspond to a fact, it is false.
	\item \textbf{Truth has to be such that there would be no truths if there were no minds.} \\ The correspondence theory meets this criterion because there are no beliefs if there are no minds, and beliefs are \textit{truth-bearers} (the sort of entity that has a truth-value).
	\item \textbf{Truth has to be such that it depends on external factors.} \\ The correspondence theory meets this criterion because whether a fact exists is external to a belief.
\end{enumerate}
%The correspondence theory meets these three criteria.  Because there are no beliefs if there are no minds, it meets criterion (2). And because beliefs are \textit{truth-bearers} (the sort of entity that has a truth-value), there are no truths when there are no minds, so it meets criterion (2). Finally, whether a fact exists is external to a belief, so a belief's truth-value depends on external factors, so it meets criterion (3).

\hypertarget{chapter-xiii.-knowledge-error-and-probable-opinion}{%
\chapter{Knowledge, Error, and Probable Opinion}\label{chapter-xiii.-knowledge-error-and-probable-opinion}}

The question as to what we mean by truth and falsehood, which we
considered in the preceding chapter, is of much less interest than the
question as to how we can know what is true and what is false. This
question will occupy us in the present chapter. There can be no doubt
that \emph{some} of our beliefs are erroneous; thus we are led to
inquire what certainty we can ever have that such and such a belief is
not erroneous. In other words, can we ever \emph{know} anything at all,
or do we merely sometimes by good luck believe what is true? Before we
can attack this question, we must, however, first decide what we mean by
`knowing', and this question is not so
easy as might be supposed.

At first sight we might imagine that knowledge could be defined as
`true belief'. When what we believe is
true, it might be supposed that we had achieved a knowledge of what we
believe. But this would not accord with the way in which the word is
commonly used. \label{luck} To take a very trivial instance: If a man believes that
the late Prime Minister's last name began with a B,\label{Balfour1} he
believes what is true, since the late Prime Minister was Sir Henry
Campbell Bannerman. But if he believes that Mr. Balfour was the late
Prime Minister, he will still believe that the late Prime
Minister's last name began with a B, yet this belief,
though true, would not be thought to constitute knowledge. If a
newspaper, by an intelligent anticipation, announces the result of a
battle before any telegram giving the result has been received, it may
by good fortune announce what afterwards turns out to be the right
result, and it may produce belief in some of its less experienced
readers. But in spite of the truth of their belief, they cannot be said
to have knowledge. Thus it is clear that a true belief is not knowledge
when it is deduced from a false belief.

In like manner, a true belief cannot be called knowledge when it is
deduced by a fallacious process of reasoning, even if the premisses from
which it is deduced are true. \label{fallacious} If I know that all Greeks are men and that
Socrates was a man, and I infer that Socrates was a Greek, I cannot be
said to \emph{know} that Socrates was a Greek, because, although my
premisses and my conclusion are true, the conclusion does not follow
from the premisses.

But are we to say that nothing is knowledge except what is validly
deduced from true premisses? Obviously we cannot say this. Such a
definition is at once too wide and too narrow. In the first place, it is
too wide, because it is not enough that our premisses should be
\emph{true}, they must also be \emph{known}. \label{premises} The man who believes that
Mr. Balfour was the late Prime Minister may proceed to draw valid
deductions from the true premiss that the late Prime
Minister's name began with a B,\label{Balfour2} but he cannot be said to
\emph{know} the conclusions reached by these deductions. Thus we shall
have to amend our definition by saying that knowledge is what is validly
deduced from \emph{known} premisses. This, however, is a circular
definition: it assumes that we already know what is meant by
`known premisses'. It can, therefore, at
best define one sort of knowledge, the sort we call derivative, as
opposed to intuitive knowledge. We may say:

'\emph{Derivative} knowledge is what is validly deduced
from premisses known intuitively'. In this statement
there is no formal defect, but it leaves the definition of
\emph{intuitive} knowledge still to seek.

Leaving on one side, for the moment, the question of intuitive
knowledge, let us consider the above suggested definition of derivative
knowledge. The chief objection to it is that it unduly limits knowledge.
It constantly happens that people entertain a true belief, which has
grown up in them because of some piece of intuitive knowledge from which
it is capable of being validly inferred, but from which it has not, as a
matter of fact, been inferred by any logical process.

Take, for example, the beliefs produced by reading. If the newspapers
announce the death of the King, we are fairly well justified in
believing that the King is dead, since this is the sort of announcement
which would not be made if it were false. And we are quite amply
justified in believing that the newspaper asserts that the King is dead.
But here the intuitive knowledge upon which our belief is based is
knowledge of the existence of sense-data derived from looking at the
print which gives the news. This knowledge scarcely rises into
consciousness, except in a person who cannot read easily. A child may be
aware of the shapes of the letters, and pass gradually and painfully to
a realization of their meaning. But anybody accustomed to reading passes
at once to what the letters mean, and is not aware, except on
reflection, that he has derived this knowledge from the sense-data
called seeing the printed letters. \label{reading} Thus although a valid inference from
the-letters to their meaning is possible, and \emph{could} be performed
by the reader, it is not in fact performed, since he does not in fact
perform any operation which can be called logical inference. Yet it
would be absurd to say that the reader does not \emph{know} that the
newspaper announces the King's death.

We must, therefore, admit as derivative knowledge whatever is the result
of intuitive knowledge even if by mere association, provided there
\emph{is} a valid logical connexion, and the person in question could
become aware of this connexion by reflection. There are in fact many
ways, besides logical inference, by which we pass from one belief to
another: the passage from the print to its meaning illustrates these
ways. These ways may be called `psychological
inference'. \label{psycho} We shall, then, admit such psychological
inference as a means of obtaining derivative knowledge, provided there
is a discoverable logical inference which runs parallel to the
psychological inference. This renders our definition of derivative
knowledge less precise than we could wish, since the word
`discoverable' is vague: it does not
tell us how much reflection may be needed in order to make the
discovery. But in fact `knowledge' is
not a precise conception: it merges into `probable
opinion', as we shall see more fully in the course of the
present chapter. A very precise definition, therefore, should not be
sought, since any such definition must be more or less misleading.

The chief difficulty in regard to knowledge, however, does not arise
over derivative knowledge, but over intuitive knowledge. So long as we
are dealing with derivative knowledge, we have the test of intuitive
knowledge to fall back upon. But in regard to intuitive beliefs, it is
by no means easy to discover any criterion by which to distinguish some
as true and others as erroneous. In this question it is scarcely
possible to reach any very precise result: all our knowledge of truths
is infected with some degree of doubt, and a theory which ignored this
fact would be plainly wrong. Something may be done, however, to mitigate
the difficulties of the question.

Our theory of truth, to begin with, supplies the possibility of
distinguishing certain truths as \emph{self-evident} in a sense which
ensures infallibility. When a belief is true, we said, there is a
corresponding fact, in which the several objects of the belief form a
single complex. The belief is said to constitute \emph{knowledge} of
this fact, provided it fulfils those further somewhat vague conditions
which we have been considering in the present chapter. But in regard to
any fact, besides the knowledge constituted by belief, we may also have
the kind of knowledge constituted by \emph{perception} (taking this word
in its widest possible sense). For example, if you know the hour of the
sunset, you can at that hour know the fact that the sun is setting: this
is knowledge of the fact by way of knowledge of \emph{truths}; but you
can also, if the weather is fine, look to the west and actually see the
setting sun: you then know the same fact by the way of knowledge of
\emph{things}.

Thus in regard to any complex fact, there are, theoretically, two ways
in which it may be known: (1) by means of a judgement, in which its
several parts are judged to be related as they are in fact related; (2)
by means of \emph{acquaintance} with the complex fact itself, which may
(in a large sense) be called perception, though it is by no means
confined to objects of the senses. Now it will be observed that the
second way of knowing a complex fact, the way of acquaintance, is only
possible when there really is such a fact, while the first way, like all
judgement, is liable to error. The second way gives us the complex
whole, and is therefore only possible when its parts do actually have
that relation which makes them combine to form such a complex. The first
way, on the contrary, gives us the parts and the relation severally, and
demands only the reality of the parts and the relation: the relation may
not relate those parts in that way, and yet the judgement may occur.

It will be remembered that at the end of Chapter XI we suggested that
there might be two kinds of self-evidence, one giving an absolute
guarantee of truth, the other only a partial guarantee. These two kinds
can now be distinguished.

We may say that a truth is self-evident, in the first and most absolute
sense, when we have acquaintance with the fact which corresponds to the
truth. \label{selfevidence} When Othello believes that Desdemona loves Cassio, the
corresponding fact, if his belief were true, would be
`Desdemona's love for
Cassio'. This would be a fact with which no one could
have acquaintance except Desdemona; hence in the sense of self-evidence
that we are considering, the truth that Desdemona loves Cassio (if it
were a truth) could only be self-evident to Desdemona. All mental facts,
and all facts concerning sense-data, have this same privacy: there is
only one person to whom they can be self-evident in our present sense,
since there is only one person who can be acquainted with the mental
things or the sense-data concerned. Thus no fact about any particular
existing thing can be self-evident to more than one person. On the other
hand, facts about universals do not have this privacy. Many minds may be
acquainted with the same universals; hence a relation between universals
may be known by acquaintance to many different people. In all cases
where we know by acquaintance a complex fact consisting of certain terms
in a certain relation, we say that the truth that these terms are so
related has the first or absolute kind of self-evidence, and in these
cases the judgement that the terms are so related \emph{must} be true.
Thus this sort of self-evidence is an absolute guarantee of truth. \label{infallible}

But although this sort of self-evidence is an absolute guarantee of
truth, it does not enable us to be \emph{absolutely} certain, in the
case of any given judgement, that the judgement in question is true. \label{fallible}
Suppose we first perceive the sun shining, which is a complex fact, and
thence proceed to make the judgement `the sun is
shining'. In passing from the perception to the
judgement, it is necessary to analyse the given complex fact: we have to
separate out `the sun' and
`shining' as constituents of the fact.
In this process it is possible to commit an error; hence even where a
\emph{fact} has the first or absolute kind of self-evidence, a judgement
believed to correspond to the fact is not absolutely infallible, because
it may not really correspond to the fact. But if it does correspond (in
the sense explained in the preceding chapter), then it \emph{must} be
true.

The second sort of self-evidence will be that which belongs to
judgements in the first instance, and is not derived from direct
perception of a fact as a single complex whole. This second kind of
self-evidence will have degrees, from the very highest degree down to a
bare inclination in favour of the belief. Take, for example, the case of
a horse trotting away from us along a hard road. At first our certainty
that we hear the hoofs is complete; gradually, if we listen intently,
there comes a moment when we think perhaps it was imagination or the
blind upstairs or our own heartbeats; at last we become doubtful whether
there was any noise at all; then we \emph{think} we no longer hear
anything, and at last we \emph{know} we no longer hear anything. In this
process, there is a continual gradation of self-evidence, from the
highest degree to the least, not in the sense-data themselves, but in
the judgements based on them.

Or again: Suppose we are comparing two shades of colour, one blue and
one green. We can be quite sure they are different shades of colour; but
if the green colour is gradually altered to be more and more like the
blue, becoming first a blue-green, then a greeny-blue, then blue, there
will come a moment when we are doubtful whether we can see any
difference, and then a moment when we know that we cannot see any
difference. The same thing happens in tuning a musical instrument, or in
any other case where there is a continuous gradation. Thus self-evidence
of this sort is a matter of degree; and it seems plain that the higher
degrees are more to be trusted than the lower degrees.

In derivative knowledge our ultimate premisses must have some degree of
self-evidence, and so must their connexion with the conclusions deduced
from them. \label{connected} Take for example a piece of reasoning in geometry. It is not
enough that the axioms from which we start should be self-evident: it is
necessary also that, at each step in the reasoning, the connexion of
premiss and conclusion should be self-evident. In difficult reasoning,
this connexion has often only a very small degree of self-evidence;
hence errors of reasoning are not improbable where the difficulty is
great.

From what has been said it is evident that, both as regards intuitive
knowledge and as regards derivative knowledge, if we assume that
intuitive knowledge is trustworthy in proportion to the degree of its
self-evidence, there will be a gradation in trustworthiness, from the
existence of noteworthy sense-data and the simpler truths of logic and
arithmetic, which may be taken as quite certain, down to judgements
which seem only just more probable than their opposites. \label{firm} What we firmly
believe, if it is true, is called \emph{knowledge}, provided it is
either intuitive or inferred (logically or psychologically) from
intuitive knowledge from which it follows logically. What we firmly
believe, if it is not true, is called \emph{error}. \label{hesitatingly} What we firmly
believe, if it is neither knowledge nor error, and also what we believe
hesitatingly, because it is, or is derived from, something which has not
the highest degree of self-evidence, may be called \emph{probable
opinion}. Thus the greater part of what would commonly pass as knowledge
is more or less probable opinion. \label{probable}

In regard to probable opinion, we can derive great assistance from
\emph{coherence}, which we rejected as the \emph{definition} of truth,
but may often use as a \emph{criterion}. A body of individually probable
opinions, if they are mutually coherent, become more probable than any
one of them would be individually. It is in this way that many
scientific hypotheses acquire their probability. They fit into a
coherent system of probable opinions, and thus become more probable than
they would be in isolation. The same thing applies to general
philosophical hypotheses. Often in a single case such hypotheses may
seem highly doubtful, while yet, when we consider the order and
coherence which they introduce into a mass of probable opinion, they
become pretty nearly certain. This applies, in particular, to such
matters as the distinction between dreams and waking life. If our
dreams, night after night, were as coherent one with another as our
days, we should hardly know whether to believe the dreams or the waking
life. As it is, the test of coherence condemns the dreams and confirms
the waking life. But this test, though it increases probability where it
is successful, never gives absolute certainty, unless there is certainty
already at some point in the coherent system. Thus the mere organization
of probable opinion will never, by itself, transform it into indubitable
knowledge.

\protect\hypertarget{link2HCH0014}{}{}

\pagebreak
	\section{Notes for the Student}
	\markboth{CHAPTER 13 NOTES}{CHAPTER 13 NOTES}
In Ch 12 we defined truth as correspondence of a belief to a fact. Now we want to develop a three-fold classification of beliefs: they are either \textit{knowledge}, \textit{error}, or \textit{probable opinion}.
\subsection*{Derivative/Inferred Knowledge}
First, Russell argues that true belief is not \textit{sufficient} for knowledge because someone that only believes truly that $p$ \textit{by luck} or \textit{coincidence} is not said to know that $p$ (\pageref{luck}). A belief that $p$ is not knowledge when it is based on another false belief or a fallacious inference (\pageref{fallacious}). Plato takes up a similar point in the \textit{Theaetetus}, which is a dialogue about the definition of \textit{knowledge}. When Theaetetus suggests that knowledge is true belief, Socrates objects that a jury might truly believe that someone is guilty of a crime, and even convict them of it, by being persuaded by a good prosecution; but this would not give them knowledge of that person's committing the crime (200d-201c). The result of all this is:
	\[ s \text{ believes that } p \text{ and } p \text{ is true} \neq s \text{ knows that } p. \]
So we cannot define knowledge too loosely so that \textit{any} true belief is knowledge. Indeed, even a belief which is validly deduced from other true beliefs does not count as knowledge. We further need to have knowledge of the premises of our argument (\pageref{premises}).

Second, we cannot define knowledge too strictly so that \textit{only} true beliefs that are validly deduced from other true beliefs. The problem with this is that we hold many beliefs that we \textit{logically and practically could} justify properly, but we do not do so. For example, when we form a belief based on the content of what we have read, we neglect to justify that the words on the page, the sense-data, report the information on which our belief is based: we neglect that the sense-data are a medium of communicating content when we read, but we could justify this if we wanted to (\pageref{reading}). But we are often justified in forming beliefs on the basis of what is read without explicitly deducing that the sense-data that we read communicates some content. In such cases and others like them, including practically every case of testimony (which is where most of our knowledge comes from), we have derivative knowledge, but we did not make a logical inference. Rather, we make a \textit{psychological inference}, which occurs when we associate one belief with another and we have the capacity (or are in a position) to give a logical inference (\pageref{psycho}).

In contrast, in a \textit{logical inference} we infer some claim with a logical argument---much as we have done repeatedly in argument constructions on your previous assignments.
\subsection*{Intuitive Knowledge}
Most of the remaining chapter deals with the difficulty of determining \textit{when} we have intuitive knowledge. This is incredibly difficult to determine because practically any belief \textit{could} be incorrect. We are all fallible, and we lack infallible means of coming to know (\pageref{fallible}).

Acquaintance is our fall-back. Russell thinks that you can be acquainted with facts---facts of perception, facts of memory, facts of introspection, and facts of abstraction involving universals. When you believe that $p$, and you have acquaintance with a fact that makes $p$ true, you have knowledge that $p$; this is as close to \textit{infallible} knowledge as you can get (\pageref{infallible}). When you have a belief that $p$ which is based on your acquaintance with a fact corresponding to $p$, there must be such a fact, which makes error \textit{really} unlikely: beliefs that are so-supported are what Russell calls \textit{self-evident} (\pageref{selfevidence}). He writes:
\begin{quote}
	In all cases where we know by acquaintance a complex fact consisting of certain terms in a certain relation, we say that the truth that these terms are so related has the first or absolute kind of self-evidence, and in these cases the judgment that the terms are so related \textit{must} be true. Thus this sort of self-evidence is an absolute guarantee of truth. (\pageref{selfevidence})
\end{quote}
Importantly, Russell denies that self-evident beliefs are \textit{certain} because self-evident beliefs are formed through judgment, and judgments are formed by analyses of experience that can be mistaken (\pageref{fallible}). So the truth corresponding to the fact is self-evident in the sense that guarantees its truth, whereas the belief corresponding to the fact has, in cases where we are acquainted with the relevant fact, some extremely high degree of self-evidence, but not \textit{certainty} because error can always creep in when we \textit{form} judgments (\pageref{fallible}).

What is intuitive knowledge, then? \textit{Intuitive knowledge} is what we have when we believe that $p$ where $p$ is true and self-evident in the infallible sense, or else to such a high degree of self-evidence that we believe it quite ``firmly" (\pageref{firm}).
\subsection*{Defining Knowledge, Error, and Probable Opinion}
Now that we have a notion of intuitive knowledge, and of self-evidence of intuited beliefs that comes in degrees, we can say that \textit{derivative/inferred knowledge} is what is inferred psychologically or logically from what is self-evident, \textit{where the logical connection between what is inferred and what is intuitively known obtains} (\pageref{connected}). 

So speaking generally, \textit{knowledge} is a feature of true, self-evident beliefs---where self-evident here means that they are intuitive or inferred logically or inferred psychologically.

In contrast, when we have a firm belief that is false, it is \textit{error}. Falsity suffices for error.

But \textit{probable opinion} is what we believe firmly that is neither knowledge nor error, or else is believed ``hesitatingly" because we are unsure about the self-evidence of the claim or the self-evidence of what we infer it from (\pageref{hesitatingly}). Russell thinks that most of our beliefs are probable opinion (\pageref{probable}). This is partly because knowledge demands a lot.

It may be helpful to have a list of criteria for knowledge. A belief $b$ is knowledge when:
\begin{enumerate}
	\item the belief $b$ is true, 
	\item the belief $b$ is self-evident in the highest degree, and 
	\item either the belief $b$ is intuitive, or else 
	\item the belief $b$ is inferred (psychologically or logically) from known beliefs, and the belief $b$ is logically entailed by the belief from which it is inferred. 
\end{enumerate}

\hypertarget{chapter-xiv.-the-limits-of-philosophical-knowledge}{%
\chapter{The Limits of Philosophical Knowledge}\label{chapter-xiv.-the-limits-of-philosophical-knowledge}}

In all that we have said hitherto concerning philosophy, we have
scarcely touched on many matters that occupy a great space in the
writings of most philosophers. Most philosophers---or, at any rate, very
many---profess to be able to prove, by \emph{a priori} metaphysical
reasoning, such things as the fundamental dogmas of religion, the
essential rationality of the universe, the illusoriness of matter, the
unreality of all evil, and so on. There can be no doubt that the hope of
finding reason to believe such theses as these has been the chief
inspiration of many life-long students of philosophy. This hope, I
believe, is vain. \label{vain} It would seem that knowledge concerning the universe
as a whole is not to be obtained by metaphysics, and that the proposed
proofs that, in virtue of the laws of logic such and such things
\emph{must} exist and such and such others cannot, are not capable of
surviving a critical scrutiny. In this chapter we shall briefly consider
the kind of way in which such reasoning is attempted, with a view to
discovering whether we can hope that it may be valid.

The great representative, in modern times, of the kind of view which we
wish to examine, was \href{https://plato.stanford.edu/entries/hegel/}{Hegel} (1770-1831). \label{hegel} Hegel's
philosophy is very difficult, and commentators differ as to the true
interpretation of it. According to the interpretation I shall adopt,
which is that of many, if not most, of the commentators and has the
merit of giving an interesting and important type of philosophy, his
main thesis is that everything short of the Whole is obviously
fragmentary, and obviously incapable of existing without the complement
supplied by the rest of the world. Just as a comparative anatomist, from
a single bone, sees what kind of animal the whole must have been, so the
metaphysician, according to Hegel, sees, from any one piece of reality,
what the whole of reality must be---at least in its large outlines.
Every apparently separate piece of reality has, as it were, hooks which
grapple it to the next piece; the next piece, in turn, has fresh hooks,
and so on, until the whole universe is reconstructed. This essential
incompleteness appears, according to Hegel, equally in the world of
thought and in the world of things. In the world of thought, if we take
any idea which is abstract or incomplete, we find, on examination, that
if we forget its incompleteness, we become involved in contradictions;
these contradictions turn the idea in question into its opposite, or
antithesis; and in order to escape, we have to find a new, less
incomplete idea, which is the synthesis of our original idea and its
antithesis. This new idea, though less incomplete than the idea we
started with, will be found, nevertheless, to be still not wholly
complete, but to pass into its antithesis, with which it must be
combined in a new synthesis. In this way Hegel advances until he reaches
the `Absolute Idea', which, according to
him, has no incompleteness, no opposite, and no need of further
development. The Absolute Idea, therefore, is adequate to describe
Absolute Reality; but all lower ideas only describe reality as it
appears to a partial view, not as it is to one who simultaneously
surveys the Whole. \label{harmonious} Thus Hegel reaches the conclusion that Absolute
Reality forms one single harmonious system, not in space or time, not in
any degree evil, wholly rational, and wholly spiritual. Any appearance
to the contrary, in the world we know, can be proved logically---so he
believes---to be entirely due to our fragmentary piecemeal view of the
universe. If we saw the universe whole, as we may suppose God sees it,
space and time and matter and evil and all striving and struggling would
disappear, and we should see instead an eternal perfect unchanging
spiritual unity.

In this conception, there is undeniably something sublime, something to
which we could wish to yield assent. Nevertheless, when the arguments in
support of it are carefully examined, they appear to involve much
confusion and many unwarrantable assumptions. The fundamental tenet upon
which the system is built up is that what is incomplete must be not
self-subsistent, but must need the support of other things before it can
exist. It is held that whatever has relations to things outside itself
must contain some reference to those outside things in its own nature,
and could not, therefore, be what it is if those outside things did not
exist. A man's nature, for example, is constituted by
his memories and the rest of his knowledge, by his loves and hatreds,
and so on; thus, but for the objects which he knows or loves or hates,
he could not be what he is. He is essentially and obviously a fragment:
taken as the sum-total of reality he would be self-contradictory.

This whole point of view, however, turns upon the notion of the
`nature' of a thing, which seems to
mean `all the truths about the thing'. It
is of course the case that a truth which connects one thing with another
thing could not subsist if the other thing did not subsist. But a truth
about a thing is not part of the thing itself, although it must,
according to the above usage, be part of the
`nature' of the thing. \label{nature} If we mean by a
thing's `nature' all
the truths about the thing, then plainly we cannot know a
thing's `nature' unless
we know all the thing's relations to all the other
things in the universe. But if the word
`nature' is used in this sense, we
shall have to hold that the thing may be known when its
`nature' is not known, or at any rate
is not known completely. There is a confusion, when this use of the word
`nature' is employed, between knowledge
of things and knowledge of truths. We may have knowledge of a thing by
acquaintance even if we know very few propositions about
it---theoretically we need not know any propositions about it. Thus,
acquaintance with a thing does not involve knowledge of its
`nature' in the above sense. And
although acquaintance with a thing is involved in our knowing any one
proposition about a thing, knowledge of its
`nature', in the above sense, is not
involved. Hence, (1) acquaintance with a thing does not logically
involve a knowledge of its relations, and (2) a knowledge of some of its
relations does not involve a knowledge of all of its relations nor a
knowledge of its `nature' in the above
sense. I may be acquainted, for example, with my toothache, and this
knowledge may be as complete as knowledge by acquaintance ever can be,
without knowing all that the dentist (who is not acquainted with it) can
tell me about its cause, and without therefore knowing its
`nature' in the above sense. Thus the
fact that a thing has relations does not prove that its relations are
logically necessary. That is to say, from the mere fact that it is the
thing it is we cannot deduce that it must have the various relations
which in fact it has. This only \emph{seems} to follow because we know
it already.

It follows that we cannot prove that the universe as a whole forms a
single harmonious system such as Hegel believes that it forms. And if we
cannot prove this, we also cannot prove the unreality of space and time
and matter and evil, for this is deduced by Hegel from the fragmentary
and relational character of these things. \label{piecemeal} Thus we are left to the
piecemeal investigation of the world, and are unable to know the
characters of those parts of the universe that are remote from our
experience. This result, disappointing as it is to those whose hopes
have been raised by the systems of philosophers, is in harmony with the
inductive and scientific temper of our age, and is borne out by the
whole examination of human knowledge which has occupied our previous
chapters.

Most of the great ambitious attempts of metaphysicians have proceeded by
the attempt to prove that such and such apparent features of the actual
world were self-contradictory, and therefore could not be real. The
whole tendency of modern thought, however, is more and more in the
direction of showing that the supposed contradictions were illusory, and
that very little can be proved \emph{a priori} from considerations of
what \emph{must} be. A good illustration of this is afforded by space
and time. Space and time appear to be infinite in extent, and infinitely
divisible. If we travel along a straight line in either direction, it is
difficult to believe that we shall finally reach a last point, beyond
which there is nothing, not even empty space. Similarly, if in
imagination we travel backwards or forwards in time, it is difficult to
believe that we shall reach a first or last time, with not even empty
time beyond it. Thus space and time appear to be infinite in extent.

Again, if we take any two points on a line, it seems evident that there
must be other points between them however small the distance between
them may be: every distance can be halved, and the halves can be halved
again, and so on \emph{ad infinitum}. In time, similarly, however little
time may elapse between two moments, it seems evident that there will be
other moments between them. Thus space and time appear to be infinitely
divisible. But as against these apparent facts---infinite extent and
infinite divisibility---philosophers have advanced arguments tending to
show that there could be no infinite collections of things, and that
therefore the number of points in space, or of instants in time, must be
finite. Thus a contradiction emerged between the apparent nature of
space and time and the supposed impossibility of infinite collections.

Kant, who first emphasized this contradiction, deduced the impossibility
of space and time, which he declared to be merely subjective; and since
his time very many philosophers have believed that space and time are
mere appearance, not characteristic of the world as it really is. Now,
however, owing to the labours of the mathematicians, notably Georg
Cantor, it has appeared that the impossibility of infinite collections
was a mistake. They are not in fact self-contradictory, but only
contradictory of certain rather obstinate mental prejudices. Hence the
reasons for regarding space and time as unreal have become inoperative,
and one of the great sources of metaphysical constructions is dried up.

The mathematicians, however, have not been content with showing that
space as it is commonly supposed to be is possible; they have shown also
that many other forms of space are equally possible, so far as logic can
show. Some of Euclid's axioms, which appear to common
sense to be necessary, and were formerly supposed to be necessary by
philosophers, are now known to derive their appearance of necessity from
our mere familiarity with actual space, and not from any \emph{a priori}
logical foundation. By imagining worlds in which these axioms are false,
the mathematicians have used logic to loosen the prejudices of common
sense, and to show the possibility of spaces differing---some more, some
less---from that in which we live. And some of these spaces differ so
little from Euclidean space, where distances such as we can measure are
concerned, that it is impossible to discover by observation whether our
actual space is strictly Euclidean or of one of these other kinds. Thus
the position is completely reversed. Formerly it appeared that
experience left only one kind of space to logic, and logic showed this
one kind to be impossible. Now, logic presents many kinds of space as
possible apart from experience, and experience only partially decides
between them. Thus, while our knowledge of what is has become less than
it was formerly supposed to be, our knowledge of what may be is
enormously increased. Instead of being shut in within narrow walls, of
which every nook and cranny could be explored, we find ourselves in an
open world of free possibilities, where much remains unknown because
there is so much to know. \label{possibilities}

What has happened in the case of space and time has happened, to some
extent, in other directions as well. The attempt to prescribe to the
universe by means of \emph{a priori} principles has broken down; logic,
instead of being, as formerly, the bar to possibilities, has become the
great liberator of the imagination, presenting innumerable alternatives
which are closed to unreflective common sense, and leaving to experience
the task of deciding, where decision is possible, between the many
worlds which logic offers for our choice. \label{choice} Thus knowledge as to what
exists becomes limited to what we can learn from experience---not to
what we can actually experience, for, as we have seen, there is much
knowledge by description concerning things of which we have no direct
experience. But in all cases of knowledge by description, we need some
connexion of universals, enabling us, from such and such a datum, to
infer an object of a certain sort as implied by our datum. Thus in
regard to physical objects, for example, the principle that sense-data
are signs of physical objects is itself a connexion of universals; and
it is only in virtue of this principle that experience enables us to
acquire knowledge concerning physical objects. The same applies to the
law of causality, or, to descend to what is less general, to such
principles as the law of gravitation.

Principles such as the law of gravitation are proved, or rather are
rendered highly probable, by a combination of experience with some
wholly \emph{a priori} principle, such as the principle of induction.
Thus our intuitive knowledge, which is the source of all our other
knowledge of truths, is of two sorts: pure empirical knowledge, which
tells us of the existence and some of the properties of particular
things with which we are acquainted, and pure \emph{a priori} knowledge,
which gives us connexions between universals, and enables us to draw
inferences from the particular facts given in empirical knowledge. Our
derivative knowledge always depends upon some pure \emph{a priori}
knowledge and usually also depends upon some pure empirical knowledge.

Philosophical knowledge, if what has been said above is true, does not
differ essentially from scientific knowledge; there is no special source
of wisdom which is open to philosophy but not to science, and the
results obtained by philosophy are not radically different from those
obtained from science. The essential characteristic of philosophy, which
makes it a study distinct from science, is criticism. \label{criticism} It examines
critically the principles employed in science and in daily life; it
searches out any inconsistencies there may be in these principles, and
it only accepts them when, as the result of a critical inquiry, no
reason for rejecting them has appeared. If, as many philosophers have
believed, the principles underlying the sciences were capable, when
disengaged from irrelevant detail, of giving us knowledge concerning the
universe as a whole, such knowledge would have the same claim on our
belief as scientific knowledge has; but our inquiry has not revealed any
such knowledge, and therefore, as regards the special doctrines of the
bolder metaphysicians, has had a mainly negative result. But as regards
what would be commonly accepted as knowledge, our result is in the main
positive: we have seldom found reason to reject such knowledge as the
result of our criticism, and we have seen no reason to suppose man
incapable of the kind of knowledge which he is generally believed to
possess.

When, however, we speak of philosophy as a \emph{criticism} of
knowledge, it is necessary to impose a certain limitation. If we adopt
the attitude of the complete sceptic, placing ourselves wholly outside
all knowledge, and asking, from this outside position, to be compelled
to return within the circle of knowledge, we are demanding what is
impossible, and our scepticism can never be refuted. For all refutation
must begin with some piece of knowledge which the disputants share; from
blank doubt, no argument can begin. Hence the criticism of knowledge
which philosophy employs must not be of this destructive kind, if any
result is to be achieved. Against this absolute scepticism, no
\emph{logical} argument can be advanced. \label{skeptic} But it is not difficult to see
that scepticism of this kind is unreasonable.
Descartes' `methodical
doubt', with which modern philosophy began, is not of
this kind, but is rather the kind of criticism which we are asserting to
be the essence of philosophy. His `methodical
doubt' consisted in doubting whatever seemed doubtful;
in pausing, with each apparent piece of knowledge, to ask himself
whether, on reflection, he could feel certain that he really knew it.
This is the kind of criticism which constitutes philosophy. Some
knowledge, such as knowledge of the existence of our sense-data, appears
quite indubitable, however calmly and thoroughly we reflect upon it. In
regard to such knowledge, philosophical criticism does not require that
we should abstain from belief. But there are beliefs---such, for
example, as the belief that physical objects exactly resemble our
sense-data---which are entertained until we begin to reflect, but are
found to melt away when subjected to a close inquiry. Such beliefs
philosophy will bid us reject, unless some new line of argument is found
to support them. But to reject the beliefs which do not appear open to
any objections, however closely we examine them, is not reasonable, and
is not what philosophy advocates.

The criticism aimed at, in a word, is not that which, without reason,
determines to reject, but that which considers each piece of apparent
knowledge on its merits, and retains whatever still appears to be
knowledge when this consideration is completed. That some risk of error
remains must be admitted, since human beings are fallible. Philosophy
may claim justly that it diminishes the risk of error, and that in some
cases it renders the risk so small as to be practically negligible. To
do more than this is not possible in a world where mistakes must occur;
and more than this no prudent advocate of philosophy would claim to have
performed.

\protect\hypertarget{link2HCH0015}{}{}

\pagebreak
\section{Notes for the Student}
\markboth{CHAPTER 14 NOTES}{CHAPTER 14 NOTES}
Philosophers have traditionally attempted to establish ethical truths or claims about the whole universe. The tradition is widespread, ongoing, and highly varied. For example:
\begin{itemize}
	\item Parmenides (400s BCE) in his poem \textit{On Nature} aims to establish that ``What Is" (the universe, presumably) is one interconnected, continuous, eternal, unchanging whole.
	\item Plato (427-347 BCE) in his \textit{Timaeus} that the whole universe is a mathematically and divinely ordered structure that is the best it could be given material constraints.
	\item Margaret Cavendish (1623-1673 CE) in her \textit{Philosophical Letters} argued that God was beyond human intelligibility and is simply to be admired, adored, and worshiped.
	\item Philippa Foot (1920-2010 CE) in her \textit{Natural Goodness} argued that inability to reproduce was a defect with respect to certain goods that humans as a species pursue.
	\item Gertrude Elizabeth Anscombe (1919-2001) in her ``Contraception and Chastity" argued that there is no such thing as casual sex because all sexual acts are morally significant.
	\item Judith Jarvis Thomson (1929--) in her ``A Defense of Abortion" argues that prohibiting abortion is morally wrong.
\end{itemize}
Russell rejects all such attempts to establish ethical and universe-wide claims regardless of the claim. Such argumentation goes beyond what philosophizing can establish (\pageref{vain}). 

Russell takes Hegel as a representative of the traditional sort of philosophizing (\pageref{hegel}). Hegel aims to show that the world is ``one single harmonious system, not in space and time, not in any degree evil, wholly rational, and wholly spiritual." (\pageref{harmonious})

The key claim that Hegel uses to establish this is that every entity has a \textit{nature} that makes reference to other entities in the universe, so that no entity in the universe could be what it is without logically depending on other things (\pageref{nature}). You could support this with a kind of Leibniz-style argument: just as any identical entities must have all the same properties, so must any identical entities have all the same relations that they have to everything else. So we cannot say `Consider Landon, but with different parents' because if you take away my relation of \textit{being the child of Peter and Catherine}, then I am no longer \textit{Landon}, but someone else. So, Hegel infers, all relations are essential to me being, well, \textit{me}.

Russell flatly rejects the view that an entity is the sort of thing it is because of the relations it has (\pageref{nature}). All of you know who I am, and were acquainted with me, without having any knowledge of my parents. So it is hard to see how you all needed knowledge of my nature to have knowledge of me. And if you can logically separate knowledge of me from some of the relations that I have, then it is hard to believe that I have a nature that relates me to everything in the universe which you need to know before you can establish anything.

Russell suspects that such logically illicit moves are involved in all attempts to establish ethical truths or claims about the whole universe by philosophizing. What is left when we reject such ambitions is a ``piecemeal investigation of the world" (\pageref{piecemeal}). 

If Russell is right that this holds general of philosophizing like Hegel's, then the result is that philosophy is limited in what it can do. It cannot close off possibilities or provide us ethical and metaphysical comfort. Instead, philosophizing opens up logical possibilities:
\begin{quote}
	Thus, while our knowledge of what is has become less than it was formerly supposed to be, our knowledge of what may be is enormously increased. Instead of being shut in within narrow walls, of which every nook and cranny could be explored, we find ourselves in an open world of free possibilities, where much remains unknown because there is so much to know. (\pageref{possibilities})
\end{quote} 
And so Russell finds his view of philosophical knowledge and its limits \textit{liberating}:
\begin{quote}
	The attempt to prescribe to the universe by means of \textit{a priori} principles has broken down; logic, instead of being, as formerly, the bar to possibilities, has become the great liberator of the imagination, presenting innumerable alternatives which are closed to unreflective common sense, and leaving to experience the task of deciding, where decision is possible, between the many worlds which logic offers for our choice. (\pageref{choice})
\end{quote}
How does philosophy open possibilities to us? It does so by \textit{criticism}: whereas philosophers like Hegel took their views to be established by special faculties or insights, Russell holds that philosophical knowledge does not differ from any other, and that philosophical knowledge differs only in its unrestricted critical examination of ordinary and scientific knowledge, and in its search of global coherence among, and rational scrutiny, of our beliefs (\pageref{criticism}).

Note that this implies that philosophizing cannot answer radical or global skepticism. If someone demands an argument for \textit{every} premise we hold, we cannot begin an argument that would convince the global skeptic that we know anything at all (\pageref{skeptic}).

\hypertarget{chapter-xv.-the-value-of-philosophy}{%
\chapter{The Value of Philosophy}\label{chapter-xv.-the-value-of-philosophy}}
Having now come to the end of our brief and very incomplete review of
the problems of philosophy, it will be well to consider, in conclusion,
what is the value of philosophy and why it ought to be studied. It is
the more necessary to consider this question, in view of the fact that
many men, under the influence of science or of practical affairs, \label{practical} are
inclined to doubt whether philosophy is anything better than innocent
but useless trifling, hair-splitting distinctions, and controversies on
matters concerning which knowledge is impossible.

This view of philosophy appears to result, partly from a wrong
conception of the ends of life, partly from a wrong conception of the
kind of goods which philosophy strives to achieve. Physical science,
through the medium of inventions, is useful to innumerable people who
are wholly ignorant of it; thus the study of physical science is to be
recommended, not only, or primarily, because of the effect on the
student, but rather because of the effect on mankind in general. This
utility does not belong to philosophy. If the study of philosophy has
any value at all for others than students of philosophy, it must be only
indirectly, through its effects upon the lives of those who study it. It
is in these effects, therefore, if anywhere, that the value of
philosophy must be primarily sought. \label{indirectly}

But further, if we are not to fail in our endeavour to determine the
value of philosophy, we must first free our minds from the prejudices of
what are wrongly called `practical'
men. \label{materialonly} The `practical' man, as this word
is often used, is one who recognizes only material needs, who realizes
that men must have food for the body, but is oblivious of the necessity
of providing food for the mind. If all men were well off, if poverty and
disease had been reduced to their lowest possible point, there would
still remain much to be done to produce a valuable society; and even in
the existing world the goods of the mind are at least as important as
the goods of the body. It is exclusively among the goods of the mind
that the value of philosophy is to be found; and only those who are not
indifferent to these goods can be persuaded that the study of philosophy
is not a waste of time. \label{waste}

Philosophy, like all other studies, aims primarily at knowledge. The
knowledge it aims at is the kind of knowledge which gives unity and
system to the body of the sciences, and the kind which results from a
critical examination of the grounds of our convictions, prejudices, and
beliefs. But it cannot be maintained that philosophy has had any very
great measure of success in its attempts to provide definite answers to
its questions. If you ask a mathematician, a mineralogist, a historian,
or any other man of learning, what definite body of truths has been
ascertained by his science, his answer will last as long as you are
willing to listen. But if you put the same question to a philosopher, he
will, if he is candid, have to confess that his study has not achieved
positive results such as have been achieved by other sciences. It is
true that this is partly accounted for by the fact that, as soon as
definite knowledge concerning any subject becomes possible, this subject
ceases to be called philosophy, and becomes a separate science. The
whole study of the heavens, which now belongs to astronomy, was once
included in philosophy; Newton's great work was called
`the mathematical principles of natural
philosophy'. Similarly, the study of the human mind,
which was, until very lately, a part of philosophy, has now been separated from philosophy
and has become the science of psychology. Thus, to a great extent, the
uncertainty of philosophy is more apparent than real: \label{apparent} those questions
which are already capable of definite answers are placed in the
sciences, while those only to which, at present, no definite answer can
be given, remain to form the residue which is called philosophy.

This is, however, only a part of the truth concerning the uncertainty of
philosophy. There are many questions---and among them those that are of
the profoundest interest to our spiritual life---which, so far as we can
see, must remain insoluble to the human intellect unless its powers
become of quite a different order from what they are now. Has the
universe any unity of plan or purpose, or is it a fortuitous concourse
of atoms? Is consciousness a permanent part of the universe, giving hope
of indefinite growth in wisdom, or is it a transitory accident on a
small planet on which life must ultimately become impossible? Are good
and evil of importance to the universe or only to man? Such questions
are asked by philosophy, and variously answered by various philosophers.
But it would seem that, whether answers be otherwise discoverable or
not, the answers suggested by philosophy are none of them demonstrably
true. Yet, however slight may be the hope of discovering an answer, it
is part of the business of philosophy to continue the consideration of
such questions, to make us aware of their importance, to examine all the
approaches to them, and to keep alive that speculative interest in the
universe which is apt to be killed by confining ourselves to definitely
ascertainable knowledge.

Many philosophers, it is true, have held that philosophy could establish
the truth of certain answers to such fundamental questions. They have
supposed that what is of most importance in religious beliefs could be
proved by strict demonstration to be true. In order to judge of such
attempts, it is necessary to take a survey of human knowledge, and to
form an opinion as to its methods and its limitations. On such a subject
it would be unwise to pronounce dogmatically; but if the investigations
of our previous chapters have not led us astray, we shall be compelled
to renounce the hope of finding philosophical proofs of religious
beliefs. We cannot, therefore, include as part of the value of
philosophy any definite set of answers to such questions. Hence, once
more, the value of philosophy must not depend upon any supposed body of
definitely ascertainable knowledge to be acquired by those who study it.

The value of philosophy is, in fact, to be sought largely in its very
uncertainty. \label{prison} The man who has no tincture of philosophy goes through life
imprisoned in the prejudices derived from common sense, from the
habitual beliefs of his age or his nation, and from convictions which
have grown up in his mind without the co-operation or consent of his
deliberate reason. To such a man the world tends to become definite,
finite, obvious; common objects rouse no questions, and unfamiliar
possibilities are contemptuously rejected. As soon as we begin to
philosophize, on the contrary, we find, as we saw in our opening
chapters, that even the most everyday things lead to problems to which
only very incomplete answers can be given. Philosophy, though unable to
tell us with certainty what is the true answer to the doubts which it
raises, is able to suggest many possibilities which enlarge our thoughts
and free them from the tyranny of custom. Thus, while diminishing our
feeling of certainty as to what things are, it greatly increases our
knowledge as to what they may be; it removes the somewhat arrogant
dogmatism of those who have never travelled into the region of
liberating doubt, and it keeps alive our sense of wonder by showing
familiar things in an unfamiliar aspect. \label{antidogma}

Apart from its utility in showing unsuspected possibilities, philosophy
has a value---perhaps its chief value---through the greatness of the
objects which it contemplates, and the freedom from narrow and personal
aims resulting from this contemplation. The life of the instinctive man
is shut up within the circle of his private interests: family and
friends may be included, but the outer world is not regarded except as
it may help or hinder what comes within the circle of instinctive
wishes. In such a life there is something feverish and confined, in
comparison with which the philosophic life is calm and free. The private
world of instinctive interests is a small one, set in the midst of a
great and powerful world which must, sooner or later, lay our private
world in ruins. Unless we can so enlarge our interests as to include the
whole outer world, we remain like a garrison in a beleagured fortress,
knowing that the enemy prevents escape and that ultimate surrender is
inevitable. In such a life there is no peace, but a constant strife
between the insistence of desire and the powerlessness of will. In one
way or another, if our life is to be great and free, we must escape this
prison and this strife.

One way of escape is by philosophic contemplation. Philosophic
contemplation does not, in its widest survey, divide the universe into
two hostile camps---friends and foes, helpful and hostile, good and
bad---it views the whole impartially. Philosophic contemplation, when it
is unalloyed, does not aim at proving that the rest of the universe is
akin to man. All acquisition of knowledge is an enlargement of the Self,
but this enlargement is best attained when it is not directly sought. It
is obtained when the desire for knowledge is alone operative, by a study
which does not wish in advance that its objects should have this or that
character, but adapts the Self to the characters which it finds in its
objects. This enlargement of Self is not obtained when, taking the Self
as it is, we try to show that the world is so similar to this Self that
knowledge of it is possible without any admission of what seems alien.
The desire to prove this is a form of self-assertion and, like all
self-assertion, it is an obstacle to the growth of Self which it
desires, and of which the Self knows that it is capable. Self-assertion,
in philosophic speculation as elsewhere, views the world as a means to
its own ends; thus it makes the world of less account than Self, and the
Self sets bounds to the greatness of its goods. In contemplation, on the
contrary, we start from the not-Self, and through its greatness the
boundaries of Self are enlarged; through the infinity of the universe
the mind which contemplates it achieves some share in infinity.

For this reason greatness of soul is not fostered by those philosophies
which assimilate the universe to Man. Knowledge is a form of union of
Self and not-Self; like all union, it is impaired by dominion, and
therefore by any attempt to force the universe into conformity with what
we find in ourselves. There is a widespread philosophical tendency
towards the view which tells us that Man is the measure of all things,
that truth is man-made, that space and time and the world of universals
are properties of the mind, and that, if there be anything not created
by the mind, it is unknowable and of no account for us. This view, if
our previous discussions were correct, is untrue; but in addition to
being untrue, it has the effect of robbing philosophic contemplation of
all that gives it value, since it fetters contemplation to Self. What it
calls knowledge is not a union with the not-Self, but a set of
prejudices, habits, and desires, making an impenetrable veil between us
and the world beyond. The man who finds pleasure in such a theory of
knowledge is like the man who never leaves the domestic circle for fear
his word might not be law.

The true philosophic contemplation, on the contrary, finds its
satisfaction in every enlargement of the not-Self, in everything that
magnifies the objects contemplated, and thereby the subject
contemplating. Everything, in contemplation, that is personal or
private, everything that depends upon habit, self-interest, or desire,
distorts the object, and hence impairs the union which the intellect
seeks. By thus making a barrier between subject and object, such
personal and private things become a prison to the intellect. The free
intellect will see as God might see, without a \emph{here} and
\emph{now}, without hopes and fears, without the trammels of customary
beliefs and traditional prejudices, calmly, dispassionately, in the sole
and exclusive desire of knowledge---knowledge as impersonal, as purely
contemplative, as it is possible for man to attain. Hence also the free
intellect will value more the abstract and universal knowledge into
which the accidents of private history do not enter, than the knowledge
brought by the senses, and dependent, as such knowledge must be, upon an
exclusive and personal point of view and a body whose sense-organs
distort as much as they reveal. \label{impersonal}

The mind which has become accustomed to the freedom and impartiality of
philosophic contemplation will preserve something of the same freedom
and impartiality in the world of action and emotion. \label{serenity} It will view its
purposes and desires as parts of the whole, with the absence of
insistence that results from seeing them as infinitesimal fragments in a
world of which all the rest is unaffected by any one
man's deeds. The impartiality which, in contemplation,
is the unalloyed desire for truth, is the very same quality of mind
which, in action, is justice, and in emotion is that universal love
which can be given to all, and not only to those who are judged useful
or admirable. Thus contemplation enlarges not only the objects of our
thoughts, but also the objects of our actions and our affections: it
makes us citizens of the universe, not only of one walled city at war
with all the rest. In this citizenship of the universe consists
man's true freedom, and his liberation from the thraldom
of narrow hopes and fears.

Thus, to sum up our discussion of the value of philosophy; \label{philosophy} Philosophy is
to be studied, not for the sake of any definite answers to its
questions, since no definite answers can, as a rule, be known to be
true, but rather for the sake of the questions themselves; because these
questions enlarge our conception of what is possible, enrich our
intellectual imagination and diminish the dogmatic assurance which
closes the mind against speculation; but above all because, through the
greatness of the universe which philosophy contemplates, the mind also
is rendered great, and becomes capable of that union with the universe
which constitutes its highest good.

\protect\hypertarget{link2H_4_0017}{}{}

\pagebreak
\section{Notes for the Student}
\markboth{CHAPTER 15 NOTES}{CHAPTER 15 NOTES}
As we saw in Ch 14, Russell holds that philosophy cannot establish ethical truths. Nonetheless, he claims that philosophy has value. He is especially focuses on prejudice against philosophy by persons more interested in ``science or practical affairs" (\pageref{practical}).

Russell suggests that such persons have a wrong view of both \textit{the ends of life} and \textit{the goods philosophizing brings}. Empirical sciences have the advantage that they deliver goods that everyone enjoys, even those that are not engaged in empirical science, like roads and bridges, electronics, and so on. But philosophy does not deliver these sorts of goods that those not engaged in philosophy can enjoy. Only those engaged in philosophizing \textit{directly} enjoy its benefits, although its practitioners can \textit{indirectly} benefit those not engaged in philosophy:
\begin{quote}
	Physical science, through the medium of inventions, is useful to innumerable people who are wholly ignorant of it...This utility does not belong to philosophy. If the study of philosophy has any value at all for others than students of philosophy, it must be only indirectly, through its effects upon the lives of those who study it. (\pageref{indirectly})%\footnote{A severe typo in the Hackett reprinting (and most others) occurs in the quoted passage. It reads in part, ``Th\textbf{u}s value does not belong to philosophy." This \textit{should} read, ``Th\textbf{i}s value does not belong to philosophy." The misprinted first sentence implies that philosophy has no utility, which is contrary Russell's view. For a discussion of the typo, confer \textit{Acquaintance, Knowledge, and Logic: New Essays on Bertrand Russell's The Problems of Philosophy}, edited by Donovan Wishon and Bernard Linsky, 2015: CSLI lectures; no. 214.}
\end{quote}
The other difficulty that philosophy faces is it produces no material goods. Russell's ``practical' man" holds that all goods are material goods (\pageref{materialonly}). Most folks disagree with this view when it is put in these terms, though it still exerts an unconscious influence on most people. And if someone does claim that all goods are material goods, then Russell has an argument against the view that material goods are the only goods (\pageref{waste}):
\begin{enumerate}%[There are non-material goods that humans need.]
	\item If all humans were materially well-off with minimal poverty and disease, then much remains to be done to produce a valuable society. \hfill Premise
	\item If (1), then there are non-material goods that humans need. \hfill Premise
	\item So, there are non-material goods that humans need. \hfill 1, 2 MP
\end{enumerate}
You should think about this whenever an elected official or bureaucrat considers cutting funding for non-STEM programs. Some trade-off must be made between material and non-material goods, and nobody would deny that material goods like food, water, health, and shelter take priority over anything else. But these goods also take priority over electronics or roads and bridges. But imagine a society without poets, historians, artists, musicians, philosophers, writers, and so on, and I think that you will agree that cutting non-STEM programs is a great loss to a valuable society, and should only be done in great need.

On the other hand, Russell's argument has not shown that philosophy has value, but only that there are non-material goods. Maybe poetry, history, art, music, and writing produce all the non-material goods we need. Why would a society with poets, historians, artists, musicians, writers, and so on, be missing some other non-material good? Why is philosophy valuable? The rest of Ch 15 deals with this question of what makes philosophy valuable.

Russell's answer is that philosophy does the all of the following (\pageref{philosophy}):
\begin{enumerate}
	\item Philosophy engages with questions of importance to our spiritual life.
	\item Philosophizing enlarges and enriches our conceptions of what is possible.
	\item Philosophizing diminishes our dogmatic assurances that hinder open inquiry.
	\item Philosophers contemplate impersonal, enormous objects, which is morally improving.
\end{enumerate}
Philosophy has also birthed a number of other inquiries, like astronomy and psychology, so that the uncertainty of philosophy is largely only apparent (\pageref{apparent}). Philosophy's open-ended inquiry has birthed other disciplines that become empirical sciences and thereby cease to be philosophy. So philosophy is not useless trifling: its children testify to that.

Philosophy engages with spiritual questions that, whether answerable or not, animate intellectual and speculative interests that keep inquiry alive and combat dogmatic assurances (\pageref{antidogma}). Russell describes these assurances as a kind of mental prison (\pageref{prison}).

Philosophy has the useful effect of destroying these mental prisons of habit, custom, culture, and so on. It frees us from what is and allows us to consider alternative possibilities. It also is highly impersonal, which widens our concerns beyond what is immediately present to us (\pageref{impersonal}). It is easy to move from one task to another in life in a rather feverish way, but when you think about the vastness and uncaring nature of the universe, the inevitability of death of you and all your loved ones, it can produce a kind of calmness and serenity that practical affairs cannot touch (\pageref{serenity}). (These sorts of thoughts may lead to religious or existential crises, but that is OK---philosophy deals with those, too!)

\hypertarget{bibliographical-note}{%
	\chapter*{Bibliographical Note}\label{bibliographical-note}}

The student who wishes to acquire an elementary knowledge of philosophy
will find it both easier and more profitable to read some of the works
of the great philosophers than to attempt to derive an all-round view
from handbooks. The following are specially recommended:

\begin{quote}
	Plato: \href{https://archive.org/details/republicofpl1897plat}{Republic}, especially Books VI and VII.
	
	Descartes: \href{https://archive.org/details/methodmeditation00descuoft}{Meditations}.
	
	Spinoza: \href{https://archive.org/details/spinozahandbookt00pictuoft}{Ethics}.
	
	Leibniz: \href{https://archive.org/details/cu31924016874038}{The Monadology}.
	
	Berkeley: \href{https://archive.org/details/threedialoguesbe00berkiala}{Three Dialogues between Hylas and Philonous}.
	
	Hume: \href{https://archive.org/details/enquiryconcernin01hume}{Enquiry concerning Human Understanding}.
	
	Kant: \href{https://archive.org/details/kantsprolegome00kant}{Prolegomena to any Future Metaphysic}.
\end{quote}

\backmatter
% bibliography, glossary and index would go here.

\end{document}